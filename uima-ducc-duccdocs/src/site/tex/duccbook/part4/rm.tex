% Create well-known link to this spot for HTML version
\ifpdf
\else
\HCode{<a name='DUCC_RM'></a>}
\fi
\chapter{Resource Management}
\label{chap:rm}
    \section{Overview}

    The DUCC Resource Manager is responsible for allocating cluster resources among the various 
    requests for work in the system. DUCC recognizes several classes of work: 

    \begin{description}
        \item[Managed Jobs]
            Managed jobs are Java applications implemented in the UIMA framework. 
            They are scaled out by DUCC using UIMA-AS. Managed jobs are executed as some 
            number of discrete processes distributed over the cluster resources. All processes of all jobs 
            are by definition preemptable; the number of processes is allowed to increase and decrease 
            over time in order to provide all users access to the computing resources. 
        \item[Services]
            Services are long-running processes which perform some function on behalf of 
            jobs or other services. Most DUCC services are UIMA-AS services and are managed the 
            same as managed jobs. From a scheduling point of view, there is no difference between 
            services and managed jobs. 

        \item{Reservations}
            A reservation provides persistent, dedicated use of some portion of the 
            resources to a specific user. A reservation may be for an entire machine, or it may be for 
            some portion of a machine. Machines are subdivided according to the amount of memory 
            installed on the machine. 
            The work that DUCC is designed to support is extremely memory-intensive. In most cases 
            resources are significantly more constrained by memory than by CPU processing poser. The entire 
            resource pool in a DUCC cluster therefore consists of the total memory of all the processors in the           
            cluster. 

        \item{Arbitrary Processes}
            \todo Define these.
      \end{description}
          
    In order to apportion the cumulative memory resource among requests, the Resource Manager 
    defines some minimum unit of memory and allocates machines such that a "fair" number of 
    "memory units" are awarded to every user of the system. This minimum quantity is called a share 
    quantum, or simply, a share. The scheduling goal is to award an equitable number of memory 
    shares to every user of the system. 

    The Resource Manager awards shares according to a fair share policy. The memory shares in a 
    system are divided equally among all the users who have work in the system. Once an allocation 
    is assigned to a user, that user's jobs are then also assigned an equal number of shares, out of the 
    user's allocation. Finally, the Resource Manager maps the share allotments to physical resources. 
    
    To map a share allotment to physical resources, the Resource Manager considers the amount of 
    memory that each job declares it requires for each process. That per-process memory requirement 
    is translated into the minimum number of collocated quantum shares required for the process to 
    run. 
    
    For example, suppose the share quantum is 15GB. A job that declares it requires 14GB per process 
    is assigned one quantum share per process. If that job is assigned 20 shares, it will be allocated 20 
    processes across the cluster. A job that declares 28GB per process would be assigned two quanta 
    per process. If that job is assigned 20 shares, it is allocated 10 processes across the cluster. Both     
    jobs occupy the same amount of memory; they consume the same level of system resources. The 
    second job does so in half as many processes however. 
    
    The output of each scheduling cycle is always in terms of processes, where each process is allowed 
    to occupy some number of shares. The DUCC agents implement a mechanism to ensure that no 
    user's job processes exceed their allocated memory assignments. 
    
    Some work may be deemed to be more "important" than other work. To accommodate this, DUCC 
    allows jobs to be submitted with an indication of their relative importance: more important jobs are 
    assigned a higher "weight"; less important jobs are assigned a lower weight. During the fair share 
    calculations, jobs with higher weights are assigned more shares proportional to their weights; jobs 
    with lower weights are assigned proportionally fewer shares. Jobs with equal weights are assigned 
    an equal number of shares. This weighed adjustment of fair-share assignments is called weighted 
    fair share. 
    
    The abstraction used to organized jobs by importance is the job class or simply class. As jobs enter 
    the system they are grouped with other jobs of the same importance and assigned to a common 
    class. The class and its attributes are described in subsequent sections. 
    
    The scheduler executes in two phases: 
    \begin{enumerate}
        \item The How-Much phase: every job is assigned some number of shares, which is converted to the
          number of processes of the declared size.
        \item The What-Of phase: physical machines are found which can accommodate the number of
          processes allocated by the How-Much phase. Jobs are mapped to physical machines such that
          the total declared per-process amount of memory does not exceed the physical memory on the
          machine.  
    \end{enumerate}
      
    The How-Much phase is itself subdivided into three phases:
    \begin{enumerate}
        \item Class counts:Apply weighed fair-share to all the job classes that have jobs assigned to
          them. This apportions all shares in the system among all the classes according to their
          weights.  

        \item User counts: For each class, collect all the users with jobs submitted to that
          class, and apply fair-share (with equal weights) to equally divide all the class shares among
          the users. This apportions all shares assigned to the class among the users in this class.  A
          user may have jobs in more than one class, in which case that user's fair share is calculated
          independently within each class.
          
        \item Job counts: For each user (independently within each class), collect all the jobs
          assigned to that user and apply fair-share to equally divide all the user's shares among
          their jobs. This apportions all shares given to this user for each class among the user's
          jobs in that class.  Reservations are relatively simple. If the number of shares or
          machines requested is available or can be made available through preemption of fair-share
          jobs, the reservation is satisfied and resources are allocated. If not, the reservation
          fails. In the case where preemptions are required, the reservation is delayed until all
          necessary resources have been freed.
    \end{enumerate}

    \section{Scheduling Policies}

    The Resource Manager implements three coexistent scheduling policies. 
    \begin{description}
        \item[FAIR\_SHARE] This is the weighted-fair-share policy described in detail above.

        \item[FIXED\_SHARE] The FIXED\_SHARE policy is used to reserve a portion of a machine. The
          allocation is treated as a reservation in that it is permanently allocated (until it is
          canceled) and it cannot be preempted by any other request.

          A fixed-share request specifies a number of processes of a given size, for example, "10 
          processes of 32GB each". The ten processes may or may not be collocated on the same 
          machine. Note that the resource manager attempts to minimize fragmentation so if there is a 
          very large machine with few allocations, it is likely that there will be some collocation of the 
          assigned processes. 
          
          A fixed-share allocation may be thought of a reservation for a "partial" machine. 

        \item[RESERVE] The RESERVE policy is used to reserve a full machine. It always returns an
          allocation for an entire machine. The reservation is permanent (until it is canceled) and
          it cannot be preempted by any other request.

          It is possible to configure the scheduling policy so that a reservation returns any machine in 
          the cluster that is available, or to restrict it to machines of the size specified in the reservation 
          request. 
    \end{description}
    
    \section{Priority vs Weight}

    It is possible that the various policies may interfere with each other. It is also possible that
    the fair share weights are not sufficient to guarantee sufficient resources are allocated to
    high importance jobs. Priorities are used to resolve these conflicts

    Simply: priority is used to specify the order of evaluation of the job classes. Weight is used
    to specify the importance (or weights) of the job classes for use by the weighted fair-share
    scheduling policy.

    \paragraph{Priority.} It is possible that conflicts may arise in scheduling policies. For example, it may be
    desired that reservations be fulfilled before any fair-share jobs are scheduled. It may be
    desired that some types of jobs are so important that when they enter the system all other
    fair-share jobs be evicted. Other such examples can be found.
    
    To help resolve this, the Resource Manager allows job classes to be prioritized. Priority is
    used to determine the order of evaluation of the scheduling classes.
    
    When a scheduling cycle starts, the scheduling classes are ordered from "best" to "worst" priority. 
    The scheduler then attempts to allocate ALL of the system's resources to the "best" priority class. 
    If any resources are left, the scheduler goes on to the next class and so on, until either all the 
    resources are exhausted or there is no more work to schedule. 
    
    It is possible to have multiple job classes of the same priority. What this means is that resources 
    are allocated for the set of job classes from the same set of resources. Resources for higher priority 
    classes will have already been allocated, resources for lower priority classes may never become 
    available. 
    
    To constrain high priority jobs from completely monopolizing the system, class caps may be 
    assigned. Higher priority guarantees that some resources will be available (or made available) but 
    doesn't that that all resources necessarily be used. 

    \paragraph{Weight.} Weight is used to determine the relative importance of jobs in a set of job classes of 
    the same priority when doing fair-share allocation. All job classes of the same priority are assigned 
    shares from the full set of available resources according to their weights using weighted fair-share. 
    Weights are used only for fair-share allocation. 
    
    Class caps may also be used to insure that very high importance jobs cannot fully monopolize all of 
    the resources in the system. 

    \section{Node Pools}
    It may be desired or necessary to constrain certain types of resource allocations to a specific
    subset of the resources. Some nodes may have special hardware, or perhaps it is desired to
    prevent certain types of jobs from being scheduled on some specific set of machines. Nodepools
    are designed to provide this function.

    Nodepools impose hierarchical partitioning on the set of available machines. A nodepool is a
    subset of the full set of machines in the cluster. Nodepools may not overlap. A nodepool may
    itself contain non-overlapping subpools. The highest level nodepool is called the "global"
    nodepool. If a job class does not have an associated nodepool, the global nodepool is implicitly
    associated with he class.

    Nodepools are associated with job classes. During scheduling, a job may be assigned resources
    from its associated nodepool, or from any of the subpools which divide the associated nodepool.
    The scheduler attempts to fully exhaust resources in the associated nodepool before allocating
    within the subpools, and during eviction, attempts to first evict from the subpools. The
    scheduler insures that the nodepool mechanism does not disrupt fair-share allocation.

    If it is desired that jobs assigned to some subpool take priority over jobs that have spilled
    over from the "superpool", then the class associated with the subpool should be given greater
    weight, or greater priority, as appropriate. (See the Weight vs Priority discussion.)

    There is no explicit priority associated with nodepools. However, it is possible to assign a
    "preference" to a specific nodepool, if it is desired that those nodes be chosen first when the
    are available. Use the nodepool configurations "order" directive to do this.

    \section{Job Classes}
    \label{sec:rm.job-classes}
    The primary abstraction to control and configure the scheduler is the class. A class is simply a set 
    of rules used to parameterize how resources are assigned to jobs. Every job that enters the system is 
    associated with one job class. 
    
    The job class defines the following rules: 
    
    \begin{description}
        \item[Priority] This is the order of evaluation and assignment of resources to this class. See
          the discussion ofPriority vs Weight for details. 

        \item[Weight] This defines the "importance" of jobs in this class and is used in the weighted
          fair-share calculations. 

        \item[Scheduling Policy] This defines the policy, fair share, fixed share, or reserve used to
          schedule the jobs in this class.

        \item[Caps] Class caps limit the total resources assigned to a class. This is designed to prevent
          high importance and high priority job classes from fully monopolizing the resources. It can be
          used to limit the total resources available to lower importance and lower priority classes.

        \item[Nodepool] A class may be associated with exactly one nodepool. Jobs submitted to the class
          are assigned only resources which lie in that nodepool, or in any of the subpools defined
          within that nodepool.

        \item[Prediction] For the type of work that DUCC is designed to run, new processes typically take
          a great deal of time initializing. It is not unusual to experience 30 minutes or more of
          initialization before work items start to be processed.

          When a job is expanding (i.e. the number of assigned processes is allowed to dynamically 
          increase), it may be that the job will complete before the new processes can be assigned and 
          the work items within them complete initialization. In this situation it is wasteful to allow the 
          job to expand, even if its fair-share is greater than the number of processes it currently has 
          assigned. 
          
          By enabling prediction, the scheduler will consider the average initialization time for processes 
          in this job, current rate of work completion, and predict the number of processes needed to 
          complete the job in the optimal amount of time. If this number is less than the job's fair, share, 
          the fair share is capped by the predicted needs. 
          
        \item[Prediction Fudge] When doing prediction, it may be desired to look some time into the
          future past initialization times to predict if the job will end soon after it is expanded. The
          prediction fudge specifies a time past the expected initialization time that is used to
          predict the number of future shares needed.

        \item[Initialization cap] Because of the long initialization time of processes in most DUCC jobs,
          process failure during the initialization phase can be very expensive in terms of wasted
          resources. If a process is going to fail because of bugs, missing services, or any other
          reason, it is best to catch it early.

          The initialization cap is used to limit the number of processes assigned to a job until it is 
          known that at least one process has successfully passed from initialization to running. As soon 
          as this occurs the scheduler will proceed to assign the job its full fair-share of resources. 

        \item[Expand By Doubling] Even after initialization has succeeded, it may be desired to throttle
          the rate of expansion of a job into new processes.

          When expand-by-doubling is enabled, the scheduler allocates either twice the number of 
          resources a job currently has, or its fair-share of resources, whichever is smallest. 

        \item[Maximum Shares] This is for FIXED\_SHARE policies only. Because fixed share allocations are
          not preemptable, it may be desirable to limit the number of shares that any given request is
          allowed to receive.

        \item[Enforce Memory] This is for RESERVE policies only. It may be desired to allow a
          reservation request receive any machine in the cluster, regardless of its memory capacity. It
          may also be desired to require that an exact size be specified (to ensure the right size of
          machine is allocated). The enforce memory rule allows installations to create reservation
          classes for either policy.
    \end{description}
        
