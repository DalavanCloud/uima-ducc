\section{Overview}

DUCC is a multi-user, multi-system distributed application.  First-time installation is performed in
two stages:

\begin{itemize}
    \item Single-user installation: This provides single-user, single-system installation for testing,
      and verification. Simple development of small applications on small systems such as laptops or
      office workstations is possible after Single-user Installation.
      
    \item Multi-user installation: This provides secure multi-user capabilities and configuration
      for multi-system clusters.
\end{itemize}

First-time users must perform single-user installation and verification on a single system.  Once
this configuration is working and verified, it is straightforward to upgrade to a multi-user
configuration.

DUCC is distributed as a compressed tar file.  The instructions below assume installation from one
of this distribution media.  If building from source, the build creates this file in your svn
trunk/target directory. The distribution file is in the form
\begin{verbatim}
   apache-uima-ducc-[version]
\end{verbatim}
where [version] is the DUCC version.  For example, \distro.  This document will refer to the distribution
file as the ``<distribution.file>''.

\subsection{Software Prerequisites}
Both single and multi-user configurations have the following software pre-requisites:

\begin{itemize}
  \item A userid {\em ducc}, and group {\em ducc}.  User {\em ducc} must the the only member of group {\em ducc}.
  \item Reasonably current Linux.  DUCC has been tested on SUSE Linux distributions.
  \item IBM or Oracle Java JRE 1.6 or greater.
  \item Python 2.x, where 'x' is 4 or greater.  DUCC has not been tested on Python 3.x.
\end{itemize}
  
Multi-user installation has additional requirements:

\begin{itemize}
  \item All systems must have a shared filesystem and common user space 
  \item Passwordless ssh must be installed for user {\em ducc} on all systems.
  \item Root access is required to install a small setuid-root program on each system.
\end{itemize}
  
In order to build DUCC from source the following software is also required:
\begin{itemize}
    \item A Subversion client, from \url{http://subversion.apache.org/packages.html}
    \item Apache Maven, from \url{http://maven.apache.org/index.html}
\end{itemize}

The DUCC web server optionally supports direct ``jconsole'' attach to DUCC job processes.  To install
this, the following is required:
\begin{itemize}
    \item Apache Ant, any reasonably current version.
\end{itemize}
    
More detailed one-time setup instructions for source-level builds via subversion can be found here:
\url{http://uima.apache.org/one-time-setup.html\#svn-setup}

\subsection{Documentation}
After single-user installation, the DUCC documentation is found (in both PDF and HTML format) in the directory 
ducc\_runtime/docs.

\subsection{Building DUCC}
If you are installing from a binary distribution, continue to Initial  Installation and Verification.

Installation from source involves extracting the code from Subversion and running a Maven build.
\begin{enumerate}
  \item svn checkout \url{https://svn.apache.org/repos/asf/uima/sandbox/uima-ducc/trunk}
  \item cd trunk
  \item mvn install
\end{enumerate}
  
When the Maven install is complete, a binary distribution file will be placed into your source tree
in the subdirectory trunk/target.

\subsection{Single-user  Installation and Verification}

Single-user installation sets up an initial, working configuration on a single system.  No security
is established, and all jobs run as user ducc.  Note that all installation must be done as user ducc.

Verification submits a very simple UIMA pipeline for execution under DUCC.  Once this is shown to be
working, one may proceed to upgrade to full installation.


\subsection{Minimal Hardware Requirements for single-user Installation}
\begin{itemize}
    \item One Intel-based or IBM Power-based system.  (More systems may be added during multi-user
      installation, described below.)
    \item 8GB of memory.  16GB or more is preferable for developing and testing applications beyond
      the non-trivial.  
    \item 1GB disk space to hold the DUCC runtime, system logs, and job logs.  More is
      usually needed for larger installations.  
    \end{itemize}

\subsection{Single-user System Installation}
    \begin{enumerate}
      \item Expand the distribution file:
\begin{verbatim}
tar -zxf <distribution.file>
\end{verbatim}

        This creates a directory with the same name as ``<distribution.file'', without the trailing ``.tgz''.
  
        This directory contains the full DUCC runtime in a subdirectory called \duccruntime.  (Note:
        the version may be different according the the actual version of DUCC being installed.)

      \item You may use the \duccruntime ``in place'' but it is highly recommended that you move it
        into a standard location; for example, ducc's HOME directory:
\begin{verbatim}
mv apache-uima-ducc-0.7.3-SNAPSHOT/\duccruntime \$HOME
\end{verbatim}

        We refer to this directory, regardless of its location, as \duccruntime. For simplicity,
        this document assumes it is moved to ducc's \$HOME/\duccruntime.

      \item Change directories into the admin subdirectory of  \duccruntime: 
\begin{verbatim}
cd \$HOME/\duccruntime/admin
\end{verbatim}

        \item Run the post-installation script: 
\begin{verbatim}
./ducc\_post\_install
\end{verbatim}
          
\end{enumerate}

That's it, DUCC is installed and ready to run. (If errors were displayed during ducc\_post\_install
they must be corrected before continuing.)

The post-installation script performs these tasks:
\begin{enumerate}
    \item Verifies that the correct level of Java and Python are installed and available.
    \item Creates a default nodelist, \duccruntime/resources/ducc.nodes, containing the name of the node you are installing on.
    \item Establishes a nodepool for the DUCC Job Driver (JD) as the node you are installing from.
    \item Defines the �ducc head node� to be to node you are installing from.
    \item Sets up the default https keystore for the webserver.
    \item Installs the DUCC documentation �ducc book� into the DUCC webserver root.
    \item Builds and installs the C program, �ducc\_ling�, into the default location.
    \item Insures that the (supplied) ActiveMQ broker is runnable.
\end{enumerate}


\subsection{Initial System Verification}

Here we start the basic installation, submit a simple UIMA-AS job, verify that it ran, and stop
DUCC.  Once this is confirmed working DUCC is ready to use in an unsecured, single-user mode on a
single system.

To run the verification, issue these commands.
\begin{enumerate}
  \item cd \duccruntime/admin 
  \item ./check\_ducc
  
    Examine the output of check\_ducc.  If any errors are shown, correct the errors and rerun
    check\_ducc until there are no errors.  
  \item Finally, start ducc: ./start\_ducc
  \end{enumerate}
  
  Start\_ducc will perform a number of consistency checks and print the versions of the components.
  It then starts the ActiveMQ broker, the DUCC control processes, and a single DUCC agent on the
  local node. You will see some startup messages similar to the following:

\begin{verbatim}
ENV: Java is configured as: /share/jdk1.6/bin/java
ENV: java full version "1.6.0_14-b08"
MEM: memory is 15 gB
ENV: system is Linux version 2.6.32-220.el6.x86_64 (mockbuild@x86-004.build.bos.redhat.com) (gcc version 4.4.5 20110214 (Red Hat 4.4.5-6) (GCC) ) #1 SMP Wed Nov 9 08:03:13 EST 2011
ENV:         uima-ducc-rm.jar:     0.7.3-SNAPSHOT  compiled at None
ENV:         uima-ducc-pm.jar:     0.7.3-SNAPSHOT  compiled at None
ENV: uima-ducc-orchestrator.jar:     0.7.3-SNAPSHOT  compiled at None
ENV:         uima-ducc-sm.jar:     0.7.3-SNAPSHOT  compiled at None
ENV:        uima-ducc-web.jar:     0.7.3-SNAPSHOT  compiled at None
ENV:        uima-ducc-cli.jar:     0.7.3-SNAPSHOT  compiled at None
ENV:      uima-ducc-agent.jar:     0.7.3-SNAPSHOT  compiled at None
ENV:     uima-ducc-common.jar:     0.7.3-SNAPSHOT  compiled at None
ENV:         uima-ducc-jd.jar:     0.7.3-SNAPSHOT  compiled at None
broker host ducchead.biz.org
[] INFO: Loading '/home/challngr/.activemqrc'
[] INFO: Using java '/share/jdk1.6/bin/java'
[] INFO: Starting - inspect logfiles specified in logging.properties and log4j.properties to get details
[] INFO: pidfile created : '/home/challngr/\duccruntime/activemq/data/activemq-ducchead.biz.org.pid' (pid '14138')
[] Started AMQ broker
Waiting for broker 0
Waiting for broker 1
ActiveMQ is found on configured host and port: ducchead.biz.org:61616
Starting warm
local Starting rm
ducchead.biz.org PID 14198
local Starting pm
ducchead.biz.org PID 14223
local Starting sm
ducchead.biz.org PID 14248
local Starting or
ducchead.biz.org PID 14275
ducchead.biz.org Starting ws
ducchead.biz.org PID 14300
********** Starting agents from file /home/challngr/\duccruntime/resources/ducc.nodes
ducchead.biz.org
    ducc_ling OK
    DUCC Agent started PID 14325
bash-4.1$
\end{verbatim}

  Now open a browser and go to the DUCC webserver�s url, http://<hostname>:42133 where <hostname> is
  the name of the host where DUCC is started.  Navigate to the �Reservations� page via the links in
  the upper-left corner.  You should see the DUCC JobDriver reservation in state
  WaitingForResources.  In a few minutes this should change to Assigned.  (This usually takes 3-4
  minutes in the default configuration.) Now jobs can be submitted.

  To submit a job,
  \begin{enumerate}
    \item cd \duccruntime/examples/simple
    \item \duccruntime/bin/ducc\_submit �specification 1.job
    \end{enumerate}
    
    Open the browser in the DUCC jobs page.  You should see the job progress through a series of
    transitions: Waiting For Driver, Waiting For Services, Waiting For Resources, Initializing, and
    finally, Running.  You�ll see the number of work items submitted (15) and the number of work
    items completed grow from 0 to 15.  Finally, the job will move into Completing and then
    Completed..

    DUCC creates a log directory in your HOME directory under 
\begin{verbatim}
\$HOME/ducc/logs/job-id
\end{verbatim}

    In this directory, you will find a log for the sample job�s JobDriver (JD), JobProcess (JP), and
    a number of other files relating to the job.

    This is a good time to explore the DUCC web pages.  Notice that the job id is a link to a set of
    pages with details about the execution of the job.

    Notice also, in the upper-right corner is a link to the full DUCC documentation, the �DuccBook�.

    Finally, stop DUCC:
    \begin{enumerate}
      \item cd \duccruntime/admin
      \item./stop\_ducc -a
      \end{enumerate}
      
      Once the system is verified and the sample job completes correctly, proceed to Multi-User
      Installation and Verification to set up multiple-user support and optionally, multi-node
      operation.

\subsection{Logs}
    The DUCC system logs are written to the directory
\begin{verbatim}
	\duccruntime/logs
\end{verbatim}

    In that directory are found logs for each of the DUCC components plus one for each node DUCC is
    installed on.

    DUCC job/user logs are written by default to the user�s HOME directory under
\begin{verbatim}
$HOME/ducc/logs/<jobid>
\end{verbatim}

\subsection{Multi-User Installation and Verification}
  
    Multi-user installation consists of two steps over and above single-user installation:
    \begin{enumerate}
        \item Install and configure the setuid-root program, ducc-ling.  This small program allows DUCC
          jobs to be run as the submitting user rather than user ducc.

        \item Optionally update the configuration to include additional nodes.
     \end{enumerate}

     Multi-user installation has these pre-requisites (DUCC will not work on multiple nodes
     unless these steps are taken):
     \begin{itemize}

         \item All systems in the DUCC cluster must have a shared filesystem and shared user space (user
           directories are shared over NFS or an equivalent networked file system, across the systems, a
           user's id is the same).

         \item Passwordless ssh must be installed for user ducc on all systems.
           
         \item Root access is (briefly) required to install a small setuid-root program on each system.
      \end{itemize}

\subsection{Ducc\_ling Installation}
    ducc\_ling is a setuid-root program whose function is to execute user tasks under the identity of
    the user.  This must be installed correctly; incorrect installation can prevent jobs from running as
    their submitters, and in the worse case, can introduce security problems into the system.

    ducc\_ling must be installed on LOCAL disk on every system in the DUCC cluster, to avoid
    shared-filesystem access to it.  The path to ducc\_ling must be the same on each system.  For
    example, one could install it to /local/ducc/bin on local disk on every system.

    To install ducc\_ling (these instructions assume it is installed into /local/ducc/bin):
    As ducc, insure ducc\_ling is built correctly for your architecture:
    \begin{enumerate}
        \item cd \duccruntime/duccling/src
        \item make clean all
     \end{enumerate}
        
     Now, as root, move ducc\_ling to a secure location and grant authorization to run tasks under
     different users� identities:
     \begin{enumerate}
        \setcounter{enumi}{2}
         \item mkdir /local/ducc
         \item mkdir /local/ducc/bin
         \item chown ducc.ducc /local/ducc
         \item chown ducc.ducc /local/ducc/bin
         \item chmod 700 /local/ducc
         \item chmod 700 /local/ducc/bin
         \item cp \duccruntime/duccling/src/ducc\_ling /local/ducc/bin
         \item chown root.ducc /local/ducc/bin/ducc\_ling
         \item chmod 4750 /local/ducc/bin/ducc\_ling
      \end{enumerate}
         
      Finally, update the configuration to use this ducc\_ling instead of the default ducc\_ling:
      \begin{enumerate}
        \setcounter{enumi}{11}
        \item Edit \duccruntime/resources/ducc.properties and change this line:         
\begin{verbatim}
 ducc.agent.launcher.ducc\_spawn\_path=\ducchome/admin/ducc\_ling
\end{verbatim}
          to this line (Using the actual location of the updated ducc\_ling, if different from /local/ducc/bin):
\begin{verbatim}
ducc.agent.launcher.ducc\_spawn\_path=/local/ducc/bin/ducc\_ling
\end{verbatim}
        \end{enumerate}


        What these steps do:
      \begin{enumerate}
          \item Steps 1 and 2 compile ducc\_ling for your current machine architecture and operating system level.
          \item Steps 3and 4: Create directory /local/ducc/bin
          \item Steps 5 and 6: Set ownership of /local/ducc and /local/ducc/bin to user ducc,
            group ducc
          \item Steps 7 and 8: Set permissions for /local/ducc and /local/ducc/bin so only user
            ducc may access the contents of these directories
          \item Step 9: Copy the ducc\_ling program created in initial installation into /local/ducc/bin
          \item Step 10: set ownership of /local/ducc/bin/ducc\_ling to root, and group ownership to ducc
          \item Step 11: Establish the �setuid� bit, which allows user ducc to execute ducc\_ling with root priveleges.
          \item Step 12: Update the ducc configuration file, ducc.properties to point to the secured ducc\_ling.
       \end{enumerate}
          
       When invoked by the DUCC agents, ducc\_ling redirects process stdout and stderr to the user�s DUCC log directory with the user�s ownership, switches it�s identity to the user, and �exec�s itself into the user�s process, in a safe and secure manner.


\subsection{Set up the full nodelists (optional)}
To add additional nodes to the ducc cluster, DUCC needs to know what nodes to start its Agent
processes on.  These nodes are listed in the file
\begin{verbatim}
ducc_runtime/resources/ducc.nodes.  
\end{verbatim}

During initial installation, this file was initialized with the node DUCC is installed on.
Additional nodes may be added to the file using a text editor to increase the size of the DUCC
cluster.

\subsection{Full DUCC Verification}

This is identical to initial verification, with the one difference that the job �1.job� should be
submitted as any user other than ducc.  Watch the webserver and insure that you see the job execute
under the correct identity.  Once this completes, DUCC is installed and verified.

