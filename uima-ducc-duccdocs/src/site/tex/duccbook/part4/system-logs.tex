% 
% Licensed to the Apache Software Foundation (ASF) under one
% or more contributor license agreements.  See the NOTICE file
% distributed with this work for additional information
% regarding copyright ownership.  The ASF licenses this file
% to you under the Apache License, Version 2.0 (the
% "License"); you may not use this file except in compliance
% with the License.  You may obtain a copy of the License at
% 
%   http://www.apache.org/licenses/LICENSE-2.0
% 
% Unless required by applicable law or agreed to in writing,
% software distributed under the License is distributed on an
% "AS IS" BASIS, WITHOUT WARRANTIES OR CONDITIONS OF ANY
% KIND, either express or implied.  See the License for the
% specific language governing permissions and limitations
% under the License.
% 
% Create well-known link to this spot for HTML version
    \section{Overview}

    This chapter provides an overview of the DUCC process logs and how to interpret the
    entries therin.

    Each of the DUCC ``head node'' processes writes a detailed log of its operation to
    the directory \ducchome/logs.  The logs are managed by Apache log4j.  All logs are
    managed by a single log4j configuration file
\begin{verbatim}
        DUCC_HOME/resources/log4j.xml
\end{verbatim}

    The logs are set to roll after some reaching a given size and the number of generations
    is limited to prevent overrunning disk space.  In general the log level is set to
    provide sufficient diagnostic output to resolve most issues.

    Each DUCC component writes its own log as defined in the following table:

    \begin{tabular} {| l | l |}
       \hline
          Component & Log Name \\
      \hline
      \hline
          Resource Manager & rm.log \\
      \hline
          Service Manager & sm.log \\
      \hline
          Orchestrator & or.log \\
      \hline
          Process Manager & pm.log \\
      \hline
          Agent & {\em [hostname].agent.log } \\
      \hline
    \end{tabular}
    
    Because there may be many agents, the agent log is prefixed with the name of the host for
    each running agent.

\section{Resource Manager Log (rm.log)}
    To be filled in.

\section{Service  Manager Log (sm.log)}
    To be filled in.

\section{ (Orchestrator Log or.log)}
    To be filled in.

\section{Process Manager Log (pm.log)}
    To be filled in.

\section{Agent log Log (hostname.agent.log)}
    To be filled in.
