
\section{Administrative Commands}

\subsection{start-ducc}

    \subsubsection{{\em Description}}
    The command \ducchome/admin/start\_ducc is used to start DUCC processes. If run with no parameters
    it takes the following actions:
    \begin{itemize}
      \item Starts the management processes Resource Manager, Orchestrator, Process Manager,      
      Services Manager, and Web Server on the local node (where start\_ducc is executed.       
      \item Starts an agent process on every node named in the default node list. 
    \end{itemize}

    \subsubsection{{\em Usage}}

    \begin{description}
      \item[start\_ducc {[options]}] \hfill \\ 
        If no options are given, all DUCC processes are started, using the default node list, 
        ducc\_runtime/resources/ducc.nodes. This is the equivalent of 
\begin{verbatim}
start\_ducc -n $DUCC\_HOME/resources/ducc.nodes -m 
\end{verbatim}
      \end{description}
      
      \subsubsection{{\em Options: }}
      \begin{description}

        \item[-n, --nodelist {[nodefile] }] \hfill \\
          Start agents on the nodes in the nodefile. Multiple nodefiles may be specified: 
\begin{verbatim}
start\_ducc -n foo.nodes -n bar.nodes -n baz.nodes 
\end{verbatim}
          

        \item[-m, --management] \hfill \\
          Start the management processes (rm, sm, pm, orchestrator) on the local node. The 
          webserver is started on the local node, or the node configured in ducc.properties. 

        \item[-c, --component {[component] }] \hfill \\
          Start a specific DUCC component, optionally on a specific node. If the component 
          name is qualified with a nodename, the component is started on that node. To qualify 
          a component name with a destination node, use the notation component@nodename. 
          Multiple components may be specified: 
\begin{verbatim}
start\_ducc -c sm -c pm -c rm -c or@bj22 -c agent@n1 -c agent@n2 
\end{verbatim}
          
          Components include: 
          \begin{description}
            \item[rm] The Resource Manager
            \item[or]The Orchestrator
            \item[pm]The Process Manager
            \item[sm]The Service Manager
            \item[ws]The Web Server
            \item[agent]Node Agents
          \end{description}

      \end{description}

      \subsubsection{{\em Notes: }}
      A different nodelist may be used to specify where Agents are started. As well multiple node 
      lists may be specified, in which case Agents are started on all the nodes in the multiple node 
      lists. 
      
      To start only agents, run start\_ducc specifying a nodelist explicitly. When started like this, the 
      management daemons are not started unless explicitly requested. 
      
      To start only management processes, run start\_ducc with the -m or --management flags. When 
      started like the the agents are not started unless explicitly requested. 
      
      To start a specific management process, run start\_ducc with the -c component parameter, 
      specify the component that should be started. 
  
      \subsubsection{{\em Examples: }}

      Start all DUCC processes, using custom nodelists: 
\begin{verbatim}
        start\_ducc -m -n foo.nodes -n bar.nodes 
\end{verbatim}
      
      Start just management processes: 
\begin{verbatim}
        start\_ducc -m 
\end{verbatim}
      
      Start just agents on a specific set of nodes: 
\begin{verbatim}
        start\_ducc -n foo.nodes -n bar.nodes 
\end{verbatim}
      
      Start and agent on a specific node: 
\begin{verbatim}
        start\_ducc -c agent@a.specific.node 
\end{verbatim}
      
      Start the webserver on node 'bingle': 
\begin{verbatim}
        start\_ducc -c ws@bingle 
\end{verbatim}

    
\subsection{stop-ducc}

    \subsubsection{{\em Description:}}
    Stop\_ducc is used to stop DUCC processes. If run with no parameters it takes the following 
    actions:
    \todo Garbled by maven or docbook, update this

    \subsubsection{\em Usage:}

    \begin{description}
      \item[stoph\_ducc {[options]}] \hfill \\ 
        If no options are given, help text is presented. At least one option is required, to avoid 
        accidental cluster shutdown. 
      \end{description}
    

      \subsubsection{Options:}
        \begin{description}

          \item[-a --all] \hfill \\
            Stop all the DUCC processes, including agents and management processes. This 
            broadcasts a "shutdown" command to all DUCC processes. Shutdown is normally 
            performed gracefully will all process including job processes given time to save state. 
            All user processes, both jobs and services, are sent shutdown signals. Job and service 
            processes which do not shutdown within a designated grace period are then forcibly 
            terminated with kill -9. 
            
          \item[-n, --nodelist [nodefile]] \hfill \\
            Only the DUCC agents in the designated nodelists are shutdown. The processes are sent 
            kill -INT signals which triggers the Java shutdown hooks and enables graceful shutdown. 
            All user processes on the indicated nodes, both jobs and services, are sent "shutdown" 
            signals and are given a minute to shutdown gracefully. Job and service processes which do 
            not shutdown within a designated grace period are then forcibly terminated with kill -9. 
            
\begin{verbatim}
stop\_ducc -n foo.nodes -n bar.nodes -n baz.nodes 
\end{verbatim}

          \item[-m, --management] \hfill \\
            Stop only the management processes rm, pm, or, sm, and ws. All agents are left running; 
            all job drivers are left running, all job processes are left running. 

          \item[-c, --component [component]] \hfill \\
            Stop a specific DUCC component. 

            This may be used to stop an errant management component and subsequently restart it 
            (with start\_ducc). 
            
            This may also be used to stop a specific agent and the job and services processes it is
            managing, without the need to specify a nodelist.  
            
            Examples: 

            Stop agents on nodes n1 and n2:
\begin{verbatim}
stop_ducc -c agent@n1 -c agent@n2 
\end{verbatim}
            
            Stop and restart the rm: 
\begin{verbatim}
stop_ducc -c rm 
start_ducc -c rmc 
\end{verbatim}
            
            Components include: 
            \begin{description}
              \item[rm] The Resource Manager.                 
              \item[or] The Orchestrator.                 
              \item[pm] The Process Manager.                 
              \item[sm] The Service Manager.                 
              \item[ws] The Web Server.                 
              \item[agent\@node] Node Agent on the specified node.
              \end{description}
              
            \item[-k, --kill] \hfill \\
              Use this to forcibly kill a component using kill -9. This should only be used if the -a option 
              does not work. This normally has the same effect as check\_ducc -k, with the difference that 
              check\_ducc indiscriminately kills all the DUCC processes it can find, whereas stop\_ducc-k 
              can be directed to a specific instance of a component. 

       \end{description}
            

          


\subsection{check\_ducc}
\label{subsec:admin.check-ducc}
    \subsubsection{{\em Description:}}

    Check\_ducc is used to find and report on DUCC processes. It can be used to find processes 
    owned by ducc (management processes, agents, and job processes), or ducc jobs owned by 
    users. 
    
    Check\_ducc can also be used to clean up errant DUCC processes when stop\_ducc is unable 
    to do so. The difference is that stop\_ducc generally tries more gracefully stop processes. 
    check\_ducc is used as a last resort, or if a fast but graceless shutdown is desired. 
    
    \subsubsection{\em{Usage: }}

        \begin{description} 
          \item[check\_ducc {[options]}]
              If no options are given this is the equivalent of: 
\begin{verbatim}
check_ducc -n ../resources/ducc.nodes 
\end{verbatim}
              
              This searches for all the processes owned by user ducc on all the nodes in ducc.nodes. User 
              processes are not searched for. 
        \end{description}
            
     \subsubsection{\em{Options:}}
         \begin{description}
           \item[-n --nodelist {[nodefile]}]
             Only the nodes specified in the nodefile are searched. The option may be specified 
             multiple times for multiple nodefiles. Note that the "local" node is always checked as well. 
\begin{verbatim}
check_ducc -n nlist1 -n nlist2 
\end{verbatim}
             
           \item[-u --user {[userid]}]
             The userid specifies the user whose processes check\_ducc searches for. If not specified, the 
             user executing check\_ducc is used. If the user is specified as 'all' then all ducc processes 
             belonging to all users are searched for. 
\begin{verbatim}
check_ducc -u billy 
\end{verbatim}
               
           \item[-p --pids]
               
               Rewrite the PID file. The PID file contains the process ids of all known DUCC 
               management and agent processes. The PID file is normally managed by start\_ducc and 
               stop\_ducc and is stored in ducc\_runtime/state/ducc.pids. 
               
               Occasionally the PID file can become partially or fully corrupted; for example, if a DUCC 
               process dies spontaneously. Use check\_ducc -p to search the cluster for processes and 
               refresh the PID file. 
               
             \item[-r --reap]

               Reap user processes. This uses kill -9 and ducc\_ling to forcibly terminate user processes. 
               Only processes specified by '-u' or '--userid' are targeted. If the user "all" is specified, then 
               all user processes are terminated. The intent of this is to easily find and terminate "rogue" 
               user processes that do not terminate. 
               
               Use this option with care. It does not distinguish user processes by specific job id. Every 
               process started by DUCC owned by the designated user is killed. 

\begin{verbatim}
check_ducc -u billy -u bobby -r 
\end{verbatim}
           \end{description}               
       
\subsection{verify\_ducc}

    \subsubsection{{\em Description:}}
    verify\_ducc performs a number of internal consistency checks to insure the DUCC installation 
    is complete and has no obvious configuration errors. The following checks are performed: 
    \begin{itemize}
      \item Insure ducc\_ling is installed in the configured location and has correct permission and ownership on all nodes. 
      \item Insure ActiveMQ is installed and configured in a way compatible with the ActiveMQ URL in ducc.properties. 
      \item Insure all nodelists exist and are readable. 
      \item Insure all nodes can be reached via ssh. 
      \item Insure all nodes are running identical versions of DUCC. 
      \item Insure java is installed in the location configured in ducc.properties on all nodes. 
      \item Print the version of java on all nodes. 
      \item Print the version of operating system on all nodes. 
      \item Print the amount of RAM on all nodes. 
      \item Insure all configured nodepools in ducc.classes exist and reference nodes are configured 
      \end{itemize}
      
      \subsubsection{{\em Usage:}}

         \begin{description}
           \item{verify\_ducc {[options]}} \hfill \\
             If no options are given, the nodes in ducc\_runtime/resources/ducc.nodes are used 
             and the default ActiveMQ broker location of ~ducc/activemq is used. 
         \end{description}
           
      \subsubsection{{\em Options:}}
          \begin{description}
            \item[-b {[broker\_install\_dir]}] \hfill \\
          This specifies the name of the ActiveMQ broker configuration file that you are using. 
\begin{verbatim}
verify_ducc -b /home/challngr/amqbroker/amq/conf/activemq-nojournal5.xml 
\end{verbatim}
          
          \item[-n {[nodelist]}] \hfill \\
            This specifies the nodelist against which the DUCC installation is verified. This nodelist             
            should be the same nodelist that DUCC will be started with. 
          \end{description}
        
      \subsubsection{{\em Notes:}}

      It may take a couple attempts to get verify\_ducc to run without error. It is important that all 
      problems reported by verify\_ducc are handled before trying to start DUCC the first time. 
      
      It is recommended that verify\_ducc be run after any update to the DUCC configuration, most 
      importantly, the addition of nodes. Verify\_ducc checks the ducc\_ling configuration on the new 
      nodes as well as verifies network and ssh connectivity. 


