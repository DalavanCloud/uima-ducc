\section{ducc.properties}
\label{sec:ducc.properties}
    The primary configuration file is called ducc.properties and always resides in the directory
    ducc\_runtime/resources.

    Some of the properties in ducc.properties are intended as the "glue" that brings the various 
    DUCC components together and lets then run as a coherent whole. These types of properties should 
    be modified only by developers of DUCC itself. In the description below these properties are 
    classified as "Private". 

    Some of the properties are tuning parameters: timeouts, heartbeat intervals, and so on. These
    may be modified by DUCC administrators, but only after experience is gained with DUCC, and only
    to solve specific performance problems. The default tuning parameters have been chosen by the
    DUCC system developers to provide "best" operation under most reasonable situations. In the
    description below these properties are classified as "Tuning".

    Some of the properties describe the local cluster configuration: the location of the ActiveMQ
    broker, the location of the Java JRE, port numbers, etc. These should be modified by the DUCC
    administrators to configure DUCC to each individual installation. In the description below these
    properties are classified as "Local".
    
\subsection{General DUCC Properties}
    \begin{description}

       \item[ducc.authentication.implementer] \hfill \\
         This specifies the class used for WebServer session authentication.  If unconfigured,
         the Web Server enforces no authentication.
         \begin{description}
           \item[Default] (unconfigured)
           \item[Type] Local
         \end{description}

       \item[ducc.admin.endpoint] \hfill \\
         This is the JMS endpoint name used for DUCC administration messages. 
         \begin{description}
           \item[Default] ducc.admin.channel 
           \item[Type] Private 
         \end{description}

       \item[ducc.admin.endpoint.type] \hfill \\
         This is the JMS message type used for DUCC administration requests. If changed DUCC 
         admin may not work. 
         \begin{description}
           \item[Default] topic 
           \item[Type] Private
         \end{description} 
           
       \item[ducc.broker.automanage] \hfill \\
         If set to ``true'', DUCC will start and stop the ActiveMq broker as part of its normal start/stop
         scripting.  
         \begin{description}
           \item[Default] true
           \item[Type] Tuning
         \end{description} 

       \item[ducc.broker.home] \hfill \\
         For DUCC auto-managed brokers only, this names the location where ActiveMq is installed
         installed.  

         Note that the DUCC installation includes a default ActiveMq.
         \begin{description}
           \item[Default] \duccruntime/activemq 
           \item[Type] Tuning
         \end{description} 
           
       \item[ducc.broker.memory.options] \hfill \\
         For DUCC auto-managed brokers only, this names the ActiveMq configuration file.  The configuration
         file is assumed to reside in the directory specified by {\em ducc.broker.home}, so the path must be relative
         to that location.
         \begin{description}
           \item[Default] conf/activemq-nojournal5.xml
           \item[Type] Tuning
         \end{description} 
           
           
       \item[ducc.broker.decoration] \hfill \\
         The property is used by the DUCC Job Driver processes to modify the ActiveMq broker URL
         when connecting to the Job Processes.

         The supplied default is used to disable broker connection timeouts.  From the ActiveMQ
         documentation: "The maximum inactivity duration (before which the socket is considered
         dead) in milliseconds. On some platforms it can take a long time for a socket to appear to
         die, so we allow the broker to kill connections if they are inactive for a period of
         time. Use by some transports to enable a keep alive heart beat feature. Set to a value
         less-than-or-equal0 to disable inactivity monitoring. Declare the wire protocol used to
         communicate with ActiveMQ."
         
         This decoration is used to keep the broker connection alive while a JVM is in a
         long garbage collection. The applications that DUCC is designed to support can
         spend significant time in garbage collection, which can cause spurious timeouts. By
         default the DUCC configuration disables the timeout by setting it to 0.       

         \begin{description}
           \item[Default] wireFormat.maxInactivityDuration=0 
           \item[Type] Local 
         \end{description}

       \item[ducc.broker.hostname] \hfill \\
         This declares the node where the ActiveMQ broker resides. It MUST be updated to 
         the actual node where the broker is running as part of DUCC installation. The default value 
         will not work.          
         \begin{description}               
           \item[Default] \$\{ducc.head\}.  The default is defined in the ducc property, {\em ducc.head}.
             If you want to run the ActiveMq broker on the ``ducc head'', this parameter need not
             be changed.
           \item[Type] Local 
         \end{description}

       \item[ducc.broker.jmx.port] \hfill \\
         This is the port used to make JMX connections to the broker.  This should only
         be changed by administrators familiar with ActiveMq configuration.         
         \begin{description}         
           \item[Default] 1100                      
           \item[Type] Local 
         \end{description}

       \item[ducc.broker.memory.options] \hfill \\
         For DUCC auto-managed brokers only, this sets the {\tt -Xmx} heap size for the broker.
         \begin{description}
           \item[Default] -Xmx2G
           \item[Type] Tuning
         \end{description} 
           

       \item[ucc.broker.name] \hfill \\
         This is the internal name of the broker, used to locate Broker's MBean in JMX Registry. 
         It is NOT related to any node name. When using the ActiveMQ distribution supplied with 
         DUCC it should always be set to "localhost".  The default should be changed only by
         administrators familiar with ActiveMq configuration.
         \begin{description}
           \item[Default] localhost 
           \item[Type] Local              
         \end{description}


       \item[ducc.broker.port] \hfill \\
         This declares the port on which the ActiveMQ broker is listening for
         messages. It MAY be updated as part of DUCC installation. ActiveMQ ships with port
         61616 as the default port, and DUCC uses that default.         
         \begin{description}
           \item[Default] 61617              
           \item[Type] Local 
         \end{description}
             

       \item[ducc.broker.protocol] \hfill \\
         Declare the wire protocol used to communicate with ActiveMQ. 
         \begin{description}
           \item[Default] tcp 
           \item[Type] Private 
         \end{description}


       \item[ducc.broker.server.url.decoration] \hfill \\
         For DUCC auto-managed brokers only, this configures ActiveMq Server url decoration.
         
         \begin{description}
           \item[Default] transport.soWriteTimeout=45000
           \item[Type] Tuning
         \end{description} 

       \item[ducc.cli.httpclient.sotimeout] \hfill \\
         This is the timeout used by the CLI to communicate with DUCC, in millisseconds. If no 
         response is heard within this time, the request times out and is aborted. When set to 0 (the 
         default), the request never times out. 
         \begin{description}
           \item[Default] 0 
           \item[Type] Tuning 
          \end{description}

       \item[ducc.cluster.name] \hfill \\
         This is a string used in the Web Server banner to identify the local cluster. It is used
         for informational purposes only and may be set to anything desired.
         \begin{description}
           \item[Default] Apache UIMA-DUCC
           \item[Type] Local 
         \end{description}
          
       \item[ducc.head] \hfill \\
         This property declares the node where the DUCC adminstrative processes run (Orchestrator,
         Resource Manager, Process Manager, Service Manager).  This property is required and MUST be
         configured in new installation.  The installation script
         \hyperref[subsec:install.single-user]{ducc\_post\_install} initializes this property to the
         node the script is executed on.
         \begin{description}
           \item[Default] There is no default, this must be configured during system installation.
           \item[Type] Local 
         \end{description}

       \item[ducc.jms.provider] \hfill \\
         Declare the type of middleware providing the JMS service used by DUCC.
         \begin{description}
           \item[Default] activemq 
           \item[Type]Private 
         \end{description}

       \item[ducc.jmx.port] \hfill \\
         Every process started by DUCC has JMX enabled by default. When more than one process 
         runs on the same machine this can cause port conflicts. The property "ducc.jmx.port" is 
         used as the base port for JMX. If the port is busy, it is incremented internally until a free 
         port is found. 
         
         The web server's \hyperref[sec:system-details.daemons]{"System $->$ Daemons"} tab is used
         to find the JMX URL that gets assigned to each of the DUCC management processes. The web
         server's \hyperref[sec:ws-job-details]{Job details} page for each job is used to find the
         JMX URL that is assigned to each JP.
         
         \begin{description}
           \item[Default] 2099 
           \item[Type] Private 
         \end{description}

       \item[ducc.jvm] \hfill \\
         This specifies the full path to the JVM to be used by the DUCC processes. This MUST be
         configured.  The installation script
         \hyperref[subsec:install.single-user]{ducc\_post\_install} initializes this property to 
         full path to ``java'' in the installer's environment.  (If the ``java'' command cannot
         be found, ducc\_post\_install exits with error.)
         \begin{description}
           \item[Default] None.  Must be configured during installation.
           \item[Type] Local 
         \end{description}

       \item[ducc.node.min.swap.threshold] \hfill \\
         Specify a minimum amount of free swap space available on a node.
         If an agent detects free swap space dipping below the value defined
         below, it will find the fattest (in terms of memory) process in its
         inventory and kill it. The value of the parameter below is expressed
         in Megabytes.

         If set to 0, the threshold is disabled.
         \begin{description}
           \item[Default] 0
           \item[Type] Tuning
         \end{description}


       \item[ducc.agent.jvm.args] \hfill \\
         This specifies the list of arguments passed to the JVM when spawing the Agent. 
         \begin{description}           
           \item[Default] Xmx100M 
           \item[Type] Tuning 
         \end{description}


       \item[ducc.driver.jvm.args] \hfill \\
         If enabled, the arguments here are automatically added to the JVM arguments specified for 
         the Job Driver process. 
         \begin{description}
           \item[Default] (unconfigured) 
           \item[Type] Local 
         \end{description}

       \item[ducc.orchestrator.jvm.args] \hfill \\
         This specifies the list of arguments passed to the JVM when spawing the Orchestrator. 
         \begin{description}
           \item[Default] Xmx1G 
           \item[Type] Tuning 
         \end{description}


       \item[ducc.pm.jvm.args] \hfill \\
         This specifies the list of arguments passed to the JVM when spawing the Process Manager. 
         \begin{description}
           \item[Default] Xmx1G 
           \item[Type] Tuning 
         \end{description}

       \item[ducc.process.jvm.args] \hfill \\
         If enabled, the arguments here are added by DUCC to the JVM arguments in the user's job 
         processes. 
         \begin{description}
           \item[Default] (unconfigured) 
           \item[Type] Private 
         \end{description}
                   
       \item[ducc.rm.jvm.args] \hfill \\
         This specifies the list of arguments passed to the JVM when spawing the Resource 
         Manager. 
         \begin{description}           
           \item[Default] Xmx1G 
           \item[Type] Tuning 
         \end{description}

       \item[ducc.sm.jvm.args] \hfill \\
         This specifies the list of arguments passed to the JVM when spawing the Service Manager. 
         \begin{description}
           \item[Default] Xmx1G 
           \item[Type] Tuning 
         \end{description}

       \item[ducc.ws.jvm.args] \hfill \\
         specifies the list of arguments passed to the JVM when spawing the Webserver. 
         \begin{description}
           \item[Default] Xmx8G 
           \item[Type] Tuning 
         \end{description}

       \item[ducc.locale.language] \hfill \\
         Establish the language for national language support of messages. Currently only "en" is 
         supported. 
         \begin{description}
           \item[Default] en 
           \item[Type] Private 
         \end{description}
           
       \item[ducc.locale.country] \hfill \\
         Establish the country for National Language Support of messages. Currently only "us" is 
         supported. 
         \begin{description}
           \item[Default] us 
           \item[Type] Private 
         \end{description}


       \item[ducc.runmode] \hfill \\
         When set to "Test" this property bypasses userid and authentication checks. It is intended 
         for use ONLY by DUCC developers. It allows developers of DUCC to simulate a multiuser 
         environment without the need for root privileges. 
         
         Note: WARNING! Enabling this feature in a production DUCC system is a potentially serious
         security breach. It should only be set by DUCC developers running with an un-privileged
         ducc\_ling.
         \begin{description}
           \item[Default] Unconfigured. When unconfigured, test mode is DISABLED.
           \item[Type] Local 
         \end{description}


        \item[ducc.signature.required] \hfill \\
          When set, the CLI signs each request so the Orchestrator can be sure the requestor is 
          actually who he claims to be. 
          \begin{description}            
            \item[Default] on             
            \item[Type] Tuning 
          \end{description}


       \item[ducc.submit.threads.limit] \hfill \\
         This enforces a maximum number of threads per job, amortized over all the processes. No 
         job will have more threads than this dispatched. This limit is disabled by default. 

         The value represents the size of the underlying CAS pool in the Job Driver and therefore
         is related to the size of the Job Driver heap and the real memory consumed by JD.  If
         the JD is consuming too much memory, try setting or reducing this value.
         
         \begin{description}
           \item[Default] (unconfigured) 
           \item[Type] Local 
         \end{description}

       \item[ducc.submit.environment.propagated] \hfill \\
         This specifies the environmental variables whose values will be merged with the
         user-specified environment option on job, process and service submissions.  A typical setting
         might be: USER HOME

         \begin{description}
           \item[Default] (unconfigured) 
           \item[Type] Local 
         \end{description}
                                                                        
      \end{description}  
        

\subsection{Web Server Properties}

    \begin{description}
        \item[ducc.ws.configuration.class] \hfill \\
          The name of the pluggable java class used to implement the Web Server. 
          \begin{description}
            \item[Default Value] org.apache.uima.ducc.ws.config.WebServerConfiguration 
            \item[Type] Private 
          \end{description}
        
        \item[ducc.ws.node] \hfill \\
          This is the name of the node the web server is started on. If not specified, the web server is 
          started on {\tt \$\{ducc\.head\}}.
          \begin{description}
            \item[Default Value] (unconfigured) 
            \item[Type] Local 
          \end{description}
            

        \item[ducc.ws.ipaddress] \hfill \\
          In multi-homed systems it may be necessary to specify to which of the multiple addresses 
          the Web Server listens for requests. This property is an IP address that specifies to which 
          address the Web Server listens. 
          \begin{description}
            \item[Default Value] (unconfigured) 
            \item[Type] Local 
          \end{description}
              
        \item[ducc.ws.port] \hfill \\
          This is the port on which the DUCC Web Server listens for requests. 
          \begin{description}
            \item[Default Value] 42133 
            \item[Type] Local 
          \end{description}

        \item[ducc.ws.port.ssl] \hfill \\
          This is the port that the Web Server uses for SSL requests (such as authentication). 
          \begin{description}
            \item[Default Value] 42155 
            \item[Type] Local 
          \end{description}

        \item[ducc.ws.port.ssl.pw] \hfill \\
          This is the password used to generate the Web Server's keystore used for HTTPS requests.  Usually
          this keystore is created at initial installation time using \hyperref[subsec:install.single-user]{ducc\_post\_install.}
          \begin{description}
            \item[Default Value] quackquack 
            \item[Type] Local
          \end{description}
                    
        \item[ducc.ws.session.minutes] \hfill \\
          Once authenticated, this property determines the lifetime of the authenticated session to the 
          Web Server. 
          \begin{description}
            \item[Default Value] 60 
            \item[Type] Tuning
          \end{description}

        \item[ducc.ws.max.history.entries] \hfill \\
          DUCC maintains a history of all jobs.  The state of jobs, both old and current are shown
          in the \hyperref[]{Webserver's Jobs Page}.  To avoid overloading this page and the Web Server, the maximum
          number of entries that can be shown is regulated by this parameter.
          \begin{description}
            \item[Default Value] 4096
            \item[Type] Tuning
          \end{description}

        \item[ducc.ws.authentication.pw] \hfill \\
          If Web Server authentication is not locally enabled (see {\em ducc.authentication.implementer}),
          this property sets a default login password for Web Server sessions.

          If not configured, no password is required for Web Server authentication.
          \begin{description}
            \item[Default Value] (not configured)
            \item[Type] Local
          \end{description}

        \item[ducc.ws.automatic.cancel.minutes] \hfill \ Optionally configure the webserver job
          automatic cancel timeout. To disable this feature specify 0.  This is employed when a user
          specifies {\em$--$wait\_for\_completion} flag on job submission, in which case the job
          monitor program must visit 
\begin{verbatim}
   http://<host>:<port>/ducc-servlet/proxy-job-status?id=<job-id>
\end{verbatim}
          within this expiry time.  Otherwise the job will be automatically canceled.

          This provides a safeguard against runaway jobs or managed reservations, if the
          submitter gets disconnected from DUCC in some way.

          If the feature is disabled by specifing ``0'', no work is canceled even if the
          monitor itself disappears.

          \begin{description}
            \item[Default Value] 10
            \item[Type] Tuning
          \end{description}

        \item[ducc.ws.jsp.compilation.directory] \hfill \\
          This specifies the temporary used by the Web Server's JSP engine to compile its JSPs.
          \begin{description}
            \item[Default Value] /tmp/ducc/jsp
            \item[Type] Tuning
          \end{description}

      \end{description}  
            
    
\subsection{Job Driver Properties}
    \begin{description}
        \item[ducc.jd.configuration.class] \hfill \\
          The name of the pluggable java class used to implement the Job Driver. 
          \begin{description}
            \item[Default Value] org.apache.uima.ducc.jd.config.JobDriverConfiguration 
            \item[Type] Private 
          \end{description}
          
        \item[ducc.jd.state.update.endpoint] \hfill \\
          This is the JMS endpoint name by the Job Driver to send state to the Orchestrator. 
          \begin{description}
            \item[Default Value] ducc.jd.state               
            \item[Type] Private 
          \end{description}
            

        \item[ducc.jd.state.update.endpoint.type] \hfill \\
          This is the JMS message type used to send state to the Orchestrator. 
          \begin{description}            
            \item[Default Value] topic 
            \item[Type] Private 
          \end{description}
          

        \item[ducc.jd.state.publish.rate] \hfill \\
          The frequency in milliseconds that JD publishes its state to the Orchestrator. A higher rate 
          may slightly increase system response but will increase network load. A lower rate will 
          somewhat decrease system response and lower network load. 
          \begin{description}
            \item[Default Value] 15000 
            \item[Type] Tuning 
          \end{description}

        \item[ducc.jd.queue.prefix] \hfill \\
          This is a human-readable string used to form queue names for the JMS queues used to pass 
          CASs from the Job Driver to the Job Processes. 
          \begin{description}
            \item[Default Value] ducc.jd.queue. 
          \i  tem[Type] Private 
          \end{description}
          
        \item[ducc.jd.host.class] \hfill \\
          This is the scheduling class used to request a reservation from the Resource Manager for 
          the machine that will be used to run the Job Driver processes. This class must also be 
          configured in ducc.classes with scheduling policy RESERVE. 
          \begin{description}
            \item[Default Value] JobDriver 
            \item[Type] Tuning 
          \end{description}
          
        \item[ducc.jd.host.description] \hfill \\
          This is a name to be associated with the reservation that is made for the Job Driver Node. 
          It can be any string and is displayed in the Reservations page on the Web Server. 
          \begin{description}
            \item[Default Value] Job Driver 
            \item[Type] Tuning 
          \end{description}

        \item[ducc.jd.memory.size] \hfill \\
          This is the amount of memory that is requested in the Job Driver reservation. It is used 
          in conjunction with the configuration of the class specified for the job driver (by default, 
          JobDriver) to schedule a node. The default configuration for this class uses a node pool 
          instead of memory to allocate the Job Driver node so by default, this parameter is ignored. 
          \begin{description}
            \item[Default Value] 8GB 
            \item[Type] Tuning 
          \end{description}
          

        \item[ducc.jd.number.of.machines] \hfill \\
          This is the number of machines to request for Job Driver nodes. This may be increased if 
          there are many jobs in the system and the load on the JD node is high enough to slow the 
          JD processes. 
          \begin{description}
            \item[Default Value] 1 
            \item[Type] Tuning 
          \end{description}
          

        \item[ducc.jd.host.user] \hfill \\
          This is the userid that is associated with the Job Driver reservation. It does not need to be 
          a "real" userid as the actual owner of the reservation is user "ducc". It is primarily used as 
          annotation of the reservation in the Web Server and logs. 
          \begin{description}
            \item[Default Value] System 
            \item[Type] Tuning 
          \end{description}


        \item[ducc.jd.share.quantum] \hfill \\
          When CGroups are enabled, this is the RSS, in MB, that is reserved for each JD process, and enforced
          by the CGroup support.  Larger JDs are permitted, but the CGroup support will force the excess
          RSS onto swap.  This potentially slows the performance of that JD, but preserves the resources
          for other, better-behaved, JDs.
          \begin{description}
            \item[Default Value] 400
            \item[Type] Tuning 
          \end{description}

      \end{description}
      
  



\subsection{Service Manager Properties}
    \begin{description}

      \item[ducc.sm.configuration.class] \hfill \\
        This is the name of the pluggable java class used to implement the Service Manager. 
        \begin{description}
          \item[Default Value] org.apache.uima.ducc.sm.config.JobDriverConfiguration 
          \item[Type] Private 
        \end{description}
        
      \item[ducc.sm.state.update.endpoint] \hfill \\
        This is the JMS endpoint name used for state messages sent by the Service Manager. 
        \begin{description}
          \item[Default Value] ducc.sm.state 
          \item[Type] Private
        \end{description}
        
      \item[ducc.sm.state.update.endpoint.type] \hfill \\
        This is the JMS message type used for state messages sent by the Service Manager. 
        \begin{description}
          \item[Default Value] topic 
          \item[Type] Private 
        \end{description}          
        
      \item[ducc.sm.meta.ping.rate] \hfill \\
        This is the time, in milliseconds, between pings by the Servic Manager to each known, 
        running service. 
        \begin{description}          
          \item[Default Value] 60000 
          \item[Typ] Tuning
        \end{description} 
        
      \item[ducc.sm.meta.ping.stability] \hfill \\
        This is the number consecutive pings that may be missed before a service is considered 
        unavailable. 
        \begin{description}
          \item[Default Value] 10 
          \item[Type] Tuning 
        \end{description}

      \item[ducc.sm.meta.ping.timeout] \hfill \\
        This is the time in milliseconds the SM waits for a response to a ping. If the service does 
        not respond within this time the ping is accounted for as a "missed" ping. 
        \begin{description}
          \item[Default Value] 5000 
          \item[Type] Tuning 
        \end{description}
        
      \item[ducc.sm.http.port] \hfill \\
        This is the HTTP port used by the Service Manager to field requests from the CLI / API. 
        \begin{description}          
          \item[Default Value] 19989 
          \item[Type] Local 
        \end{description}
        
      \item[ducc.sm.http.node] \hfill \\
        This is the node where the Service Manager runs. It MUST be configured as part of DUCC 
        setup. The {\em ducc\_post\_install} procedures initialize this to {\em \$\{ducc.head\}}.
        \begin{description}
          \item[Default Value] \$\{ducc.head\}
          \item[Type] Local 
        \end{description}
        
      \item[ducc.sm.default.linger] \hfill \\
        This is the length of time, in milliseconds, that the SM allows a service to remain alive after 
        all referenceing jobs have exited. If no new job enters the system by the time this time has 
        expired, the SM stops the service. 
        \begin{description}
          \item[Default Value] 30000
          \item[Type] Tuning 
        \end{description}
        
      \item[ducc.sm.instance.failure.max] \hfill \\
        This is the maximum number of consecutive failures of a service instance permitted before DUCC
        stops creating new instances.  In the case of submitted services, the instance is no longer
        restarted and is cleaned up.  In the case of registered services, no more instances are started
        and the {\em autostart} flag is turned off.  The next manual {\em start} command resets the
        count to 0.
        \begin{description}
          \item[Default Value] 5
          \item[Type] Tuning 
        \end{description}
        
      \end{description}
      

\subsection{Orchestrator Properties}
    \begin{description}
      \item[ducc.orchestrator. configuration.class] \hfill \\
        This is the name of the pluggable java class used to implement the DUCC Orchestrator. 
        \begin{description}
          \item[Default Value] org.apache.uima.ducc.orchestrator.config.OrchestratorConfiguration 
          \item[Type] Private
        \end{description} 
        
      \item[ducc.orchestrator.checkpoint] \hfill \\
        This controls Orchestrator state checkpointing. If set off, no state is saved across restarts of 
        the Orchestrator except for the current job numbering. This should generally be left on. 
        \begin{description}          
          \item[Default Value] on 
          \item[Type] Private 
        \end{description}
        
      \item[ducc.orchestrator.start.type] \hfill \\
        This indicates the level of recovery to be taken on restarting a system. In general, if DUCC 
        is fully shutdown, only cold and warm starts make sense because the Job Processes and 
        Job Drivers are terminated during the shutdown. However if a management process died 
        or was terminated by the administrators, most work can be recovered without interruption, 
        allowing for a hot start. There are three level of startup: 
        \begin{description}
            \item[cold] All reservations are canceled, all currently running jobs (if any) are
              terminated. All services are terminated. The system starts with no jobs, reservations, or
              services active.

            \item[warm] All reservations are restored. All currently running jobs (if any) are 
              terminated. All services are started or restarted as indicated by their state when 
              the system went down. The system starts with no jobs active, but reservations and 
              services are preserved. 

            \item[hot] All reservations are restored. The system attempts to reattach to all jobs that are 
              still running. The system attempts to reattach to any services that are still running. 
              Any services that need to be restarted are restarted.
            \end{description}

          \begin{description}
          \item[Default Value] warm 
          \item[Type] Tuning 
        \end{description}
        

      \item[ducc.orchestrator.state.endpoint] \hfill \\
        This is the name of the JMS endpoint through which the Orchestrator broadcasts its full 
        state messages. These messages include full job information and can be large. This state is 
        used by the Process Manager and the Webserver. 
        \begin{description}
          \item[Default Value] ducc.orchestrator.request?requestTimeout=180000
          \item[Type] Private 
        \end{description}

      \item[ducc.orchestrator.state.update.endpoint.type] \hfill \\
        This is the JMS endpoint type used for the "full" state messages sent by the Orchestrator. 
        \begin{description}
          \item[Default Value] topic 
          \item[Type] Private
        \end{description} 
        
      \item[ducc.orchestrator.state.publish.rate] \hfill \\
        This is the frequency in milliseconds that the Orchestrator publishes its non-abbreviated 
        state. 
        \begin{description}
          \item[Default Value] 15000 
          \item[Type] Private 
        \end{description}
        
      \item[ducc.orchestrator.abbreviated.state.endpoint] \hfill \\
        This is the name of the JMS endpoint through which the Orchestrator broadcasts its 
        abbreviated state. This state state is used by the Resource Manager and Service Manager. 
        \begin{description}
          \item[Default Value] ducc.orchestrator.abbreviated.state 
          \item[Type] Private
        \end{description} 
        
      \item[ducc.orchestrator.abbreviated.state.update.endpoint.type] \hfill \\
        This is the JMS endpoint type used for the "abbreviated" state messages sent by the 
        Orchestrator. 
        \begin{description}
          \item[Default Value] topic 
          \item[Type] Private
        \end{description} 
        
      \item[ducc.orchestrator.abbreviated.state.publish.rate] \hfill \\
        This is the frequency in milliseconds that the Orchestrator publishes its abbreviated state. 
        \begin{description}
          \item[Default Valu] 15000 
          \item[Type] Private
        \end{description} 

      \item[ducc.orchestrator.maintenance.rate] \hfill \\
        This is the frequency in milliseconds that the Orchestrator checks and updates history and 
        state. 
        \begin{description}
          \item[Default Value] 60000 
          \item[Type] Tuning 
        \end{description}
        
      \item[ducc.orchestrator.job.factory.classpath.order] \hfill \\
        When the DUCC Agent spawns a process it must set the process's Java CLASSPATH. 
        This CLASSPATH must contain a minimum set of entries, which are supplied by the 
        Agent. However, users may want their own CLASSPATH to take precedence; for 
        example, they may have a different version of some .jar file. In this case the user's 
        CLASSPATH should be set before DUCC's. To control this, set this tuning parameter to 
        one of two values: 
        \begin{itemize}
          \item[] user-before-ducc 
          \item[]ducc-before-user
        \end{itemize}
 
        \begin{description}
          \item[Default Value] user-before-ducc 
          \item[Type] Tuning 
        \end{description}


      \item[ducc.orchestrator.http.port] \hfill \\
        This is the HTTP port used by the Orchestrator to field requests from the CLI / API. 
        \begin{description}          
          \item[Default Value] 19988
          \item[Type] Local 
        \end{description}
        
      \item[ducc.orchestrator.http.node] \hfill \\
        This is the node where the Orchestrator runs. It MUST be configured as part of DUCC 
        setup. The {\em ducc\_post\_install} procedures initialize this to {\em \$\{ducc.head\}}.
        \begin{description}
          \item[Default Value] \$\{ducc.head\}
          \item[Type] Local 
        \end{description}
        
      \item[ducc.orchestrator.unmanaged.reservations.accepted] \hfill \\
        This flag controls whether the Orchestrator will accept requests for
        unmanaged reservations (true) or deny request for unmanaged reservations
        (false).
        \begin{description}
          \item[Default Value] true
          \item[Type] Local 
        \end{description}      
      \end{description}

\subsection{Resource Manager Properties}

    \begin{description}
        \item[ducc.rm.configuration.class] \hfill \\
          This is the name of the pluggable java class used to implement the DUCC Resource 
          Manager. 
          \begin{description}
            \item[Default Value] org.apache.uima.ducc.rm.config.ResourceManagerConfiguration 
            \item[Type] Private 
          \end{description}
          
        \item[ducc.rm.state.update.endpoint] \hfill \\
          This is the name of the JMS endpoint through which the Resource Manager broadcasts its 
          abbreviated state. 
          \begin{description}
            \item[Default Value] ducc.rm.state              
            \item[Type] Private
          \end{description} 

        \item[ducc.rm.state.update.endpoint.type] \hfill \\
          This is the JMS endpoint type used for state messages sent by the Resource Manager.
          \begin{description}            
            \item[Default Value] topic 
            \item[Type] Private 
          \end{description}
          
        \item[ducc.rm.state.publish.rate] \hfill \\
          This is the rate, in milliseconds, at which the Resource Manager publishes its state to the 
          Orchestrator. 
          \begin{description}
            \item[Default Value] 60000 
            \item[Type] Tuning
          \end{description} 
          
        \item[ducc.rm.fast.recovery] \hfill \\
          If enabled, RM tries to start as soon as it recoveres state from an OR publication,
          instread of waiting for {\em init.stability} for nodes to check in. 

          \begin{description}
            \item[Default Value] false
            \item[Type] Tuning
          \end{description} 
          
        \item[ducc.rm.share.quantum] \hfill \\
          The share quantum is the smallest amount of RAM that is schedulable for jobs, in GB. 
          Jobs are scheduled based entirely on their memory requirements. Memory is allocated in 
          multiples of the share quantum. 

          See the \hyperref[chap:rm]{Resource Management} section for more information on the share quantum.
          \begin{description}
            \item[Default Value] 1
            \item[Type] Tuning 
          \end{description}

        \item[ducc.rm.scheduler] \hfill \\
          The component that implements the scheduling algorithm is pluggable. This specifies the 
          name of that class. 
          \begin{description}
            \item[Default Value] org.apache.uima.ducc.rm.scheduler.NodepoolScheduler 
            \item[Type] Private
          \end{description} 
          
        \item[ducc.rm.class.definitions] \hfill \\
          This specifies the name of the file that contains the site's class definitions. This file is 
          described in detail the section on ducc.properties [77]. 
          \begin{description}
            \item[Default Value] ducc.classes 
            \item[Type] Tuning 
          \end{description}
          
        \item[ducc.rm.default.tasks] \hfill \\
          In order to calculate the number of processes to allocate to a job, the scheduler must know 
          how many tasks or work items the job will execute. If the job does not declare that number, 
          default.tasks is used. 
          \begin{description}
            \item[Default Value] 10 
            \item[Type] Tuning 
          \end{description}
          
        \item[ducc.rm.default.memory] \hfill \\
          If a job does not declare the amount of memory each process requires, the scheduler uses 
          default.memory for scheduling. The unit is GB. 

          Note that the Agents enforce the declared memory, so if a process understates its 
          requirements it will generally be killed. 
          \begin{description}
            \item[Default Value] 15 
            \item[Type] Tuning 
          \end{description}
          
        \item[ducc.rm.default.threads] \hfill \\
          Each job process will be dispatched with some number of threads such that DUCC will 
          dispatch work items to these threads. The scheduler uses this number to calculate the 
          number of processes that must be allocated. 

          The maximum number of processes a job requites is determined by the formula:           
          $num\_processes = ciel(num\_work\_items / num\_threads)$ 
          
          Thus, a job that declares 100 work items and 4 threads is assigned a maximum of          
          $ciel(100/4) = 25 processes$
          
          \begin{description}
            \item[Default Value] 4 
            \item[Type] Tuning 
          \end{description}
          
        \item[ducc.rm.node.stability] \hfill \\
          The RM receives regular "heartbeats" from the DUCC agents in order to know what 
          nodes are available for scheduling. The node.stability property configures the number of 
          consecutive heartbeats that may be missed before the Resource Manager considers the 
          node to be inoperative. 

          If a node becomes inoperative, the Resource Manager deallocates all processes on that 
          node and attempts to reallocate them on other nodes. The node is marked offline and is 
          unusable until its heartbeats start up again. 
          
          The default configuration declares the agent heartbeats to occur at 1 minute intervals. 
          Therefore heartbeats must be missed for five minutes before the Resource Manager takes 
          corrective action. 
          \begin{description}
            \item[Default Value] 5 
            \item[Type] Tuning 
          \end{description}
          

        \item[ducc.rm.init.stability] \hfill \\
          During DUCC initialization the Resource Manager must wait some period of time for 
          all the nodes in the cluster to check-in via their "heartbeats". If the RM were to start 
          scheduling too soon there would be a period of significant "churn" as the perceived cluster 
          configurations changes rapidly. As well, it would be impossible to recover work in a warm 
          or hot start if the affected nodes had not yet checked in. 
          
          The init.stability property indicates how many heartbeat intervals the RM must wait before 
          it starts scheduling after initialization. 
          \begin{description}            
            \item[Default Value] 2
            \item[Type] Tuning 
          \end{description}
          
        \item[ducc.rm.eviction.policy] \hfill \\
          The alternative value is SHRINK\_BY\_MACHINE. 

          The eviction.policy is a heuristic to choose which processes of a job to preempt because of 
          competition from other jobs. 
          
          The SHRINK\_BY\_INVESTMENT policy attempts to preempt processes such that the 
          
          least amount of work is lost. It chooses candidates for eviction in order of: 
          \begin{enumerate}
            \item Processes still initializing, with the smallest time spent in the initializing step. 
            \item Processes whose currently active work items have been executing for the shortest 
              time.
            \end{enumerate}
            The SHRINK\_BY\_MACHINE policy attempts to preempt processes so as to minimize 
            fragmentation on machines with large memories that can contain multiple job processes. 
            No consideration of execution time or initialization time is made.             
          \begin{description}
            \item[Default Value] SHRINK\_BY\_INVESTMENT 
            \item[Type] Tuning 
          \end{description}
          

        \item[ducc.rm.initialization.cap] \hfill \\
          The type of jobs supported by DUCC generally have very long and often fragile 
          initialization periods. Errors in the applications and other problems such is missing or 
          errant services can cause processes to fail during this phase. 
          
          To avoid preempting running jobs and allocating a large number of resources to jobs only 
          to fail during initialization, the Resource Manager schedules a small number of processes 
          until it is determined that the initialization phase will succeed. 
          
          The initialization.cap determines the maximum number of processes allocated to a job 
          until at least one process successfully initializes. Once any process initializes the Resource 
          Manager will proceed to allocate the job its full fair share of processes. 
          
          The initialization cap can be overridden on a class basis by configuration via           
          \hyperref[sec:ducc.classes]{ducc.classes}.

          \begin{description}
            \item[Default Value] 2             
            \item[Type] Tuning 
          \end{description}
          

        \item[ducc.rm.expand.by.doubling] \hfill \\
          When a job expands because its fair share has increased, or it has completed initialization, 
          it may be desired to govern the rate of expansion. If expand.by.doubling is set to "true", 
          rather than allocate the full fair share of processes, the number of processes is doubled 
          each scheduling cycle, up to the maximum allowed. 

          Expand.by.doubling can be overridden on a class basis by configuration via 
          \hyperref[sec:ducc.classes]{ducc.classes}.

          \begin{description}
            \item[Default Value] true 
            \item[Type] Tuning 
          \end{description}
          

        \item[ducc.rm.prediction] \hfill \\
          Because initialization time may be very long, it may be the case that a job that might be 
          eligible for expansion will be able to complete in the currently assigned shares before any 
          new processes are able to complete their initialization. In this case expansion results in 
          waste of resources and potential eviction of processes that need not be evicted. 
          
          The Resource Manager monitors the rate of task completion and attempts to predict the 
          maximum number of processes that will be needed at a time in the future based on the 
          known process initialization time. If it is determined that expansion is unnecessary then it 
          is not done for the job. 
          
          Prediction can be overridden on a class basis by configuration via
          \hyperref[sec:ducc.classes]{ducc.classes}.
          \begin{description}
            \item[Default Value] true 
            \item[Type] Tuning 
          \end{description}
          

        \item[ducc.rm.prediction.fudge] \hfill \\
          When ducc.rm.prediction is enabled, the known initialization time of a job's processes plus 
          some "fudge" factor is used to predict the number of future resources needed. The "fudge" 
          is specified in milliseconds. 
          
          The default "fudge" is very conservative. Experience and site policy should be used to set a 
          more practical number. 

          Prediction.fudge can be overridden on a class basis by configuration via 
          \hyperref[sec:ducc.classes]{ducc.classes}.

          \begin{description}
          \item[Default Value] 120000
          \item[Type] Tuning 
          \end{description}
          
        \item[ducc.rm.defragmentation] \hfill \\
          In certain configurations and under certain loads the resource allocations can get
          ``fragmented'' so that sufficient resources exists for new work, but only piecemeal, and
          thus they cannot be allocated.  The Resource Manager will perform a limited defragmentation
          by searching for ``rich'' jobs (jobs with lots of resources) and evicting one or two
          procsses in order to make spece for new jobs.  Sufficient space is cleared only to
          allow as much new work as possible to ``get a foot in the door'' and get an initial
          resource allocation.

          Local installations may override this behaviour and prevent defragmentation altogether
          with this property.
          \begin{description}
          \item[Default Value] true
          \item[Type] Tuning 
          \end{description}
          

        \item[ducc.rm.defragmentation.threshold] \hfill \\
          If {\em ducc.rm.defragmentation} is enable, limited defragmentation of resources is
          performed by the Resource Manager to create sufficient space to schedule work 
          that has insufficient resources (new jobs, for example.).  The term
          {\em insufficient} is defined as ``needing more processes than the defragmentation
          threshold, but currently having fewer processes than the defragmentation
          threshold.''  These are called ``needy'' jobs.  Additionally, the Resource Manager
          will never evict processes from ``needy'' jobs for the purpose of defragmentation.

          This property allows installations to customize the value used to determine if a
          job is ``needy''.  Jobs with fewer processes than this are potentially needed, and
          jobs with more processes are never needy.
          \begin{description}
          \item[Default Value] 2
          \item[Type] Tuning 
          \end{description}
          

        \end{description}
      


\subsection{Agent Properties}

    \begin{description}

        \item[ducc.agent.configuration.class] \hfill \\
          This is the name of the pluggable java class used to implement the DUCC Agents. 
          \begin{description}
            \item[Default Value] org.apache.uima.ducc.nodeagent.config.AgentConfiguration 
            \item[Type] Private 
          \end{description}
          
        \item[ducc.agent.request.endpoint] \hfill \\
          This is the JMS endpoint through which agents receive state from the Process Manager. 
          \begin{description}
            \item[Default Value] ducc.agent 
            \item[Type] Private 
          \end{description}
          
        \item[ducc.agent.request.endpoint.type] \hfill \\
          This is the JMS endpoint type used for state messages sent by the Process Manager. 
          \begin{description}
            \item[Default Value] topic 
            \item[Type] Private 
          \end{description}
          
        \item[ducc.agent.managed.process.state.update.endpoint] \hfill \\
          This is the JMS endpoint used to communicate from the managed process to the Agent 
          (Job Process). 
          \begin{description}
            \item[Default Value] ducc.managed.process. state.update 
            \item[Type] Private
          \end{description} 
          
        \item[ducc.agent.managed.process.state.update.endpoint.type] \hfill \\
          This is the JMS endpoint type used to communicate from the managed process (Job 
          Process) to the Agent. 
          \begin{description}
            \item[Default Value] socket 
            \item[Type] Private 
          \end{description}
          
        \item[ducc.agent.managed.process. state.update.endpoint.params] \hfill \\
          These are configuration parameters for the Agent-to-JP communication socket. These 
          should only be modified by DUCC developers. 
          \begin{description}
            \item[Default Value] transferExchange=true\&sync=false 
            \item[Type] Private
          \end{description} 
          
        \item[ducc.agent.node.metrics.endpoint] \hfill \\
          This is the JMS endpoint used to send node metrics updates to listeners. Listeners 
          are usually the Resource Manager and Web Server. These messages serve as node 
          "heartbeats". As well, the node metrics heartbeats contain the amount of RAM on the node 
          and the number of processors. 
          \begin{description}
            \item[Default Value] ducc.node.metrics 
            \item[Type] Private 
          \end{description}
          
        \item[ducc.agent.node.metrics.endpoint.type] \hfill \\
          This is the JMS endpoint type used to send node metrics updates from the agents. 
          \begin{description}
            \item[Default Value] topic 
            \item[Type] Private 
          \end{description}
         
        \item[ducc.agent.node.metrics.publish.rate] \hfill \\
          This is the rate at which node metrics updates are published in milliseconds.   Every
          agent publishes its metrics at this rate.  On large clusters, a high rate (small 
          value for the rate) can be a burdon on the network.

          Note: the Resource Manager uses the data in the node metrics for scheduling.
          \begin{description}
            \item[Default Value] 60000 
            \item[Type] Tuning 
          \end{description}
          
        \item[ducc.agent.node.inventory.endpoint] \hfill \\
          This is the JMS endpoint used to send node inventory messages to listeners. Listeners are 
          usually the Orchestrator and Web Server. Information in these messages include a map of 
          processes being managed on the node. 
          \begin{description}
            \item[Default Value] ducc.node.inventory 
            \item[Type] Private 
          \end{description}
          
        \item[ducc.agent.node.inventory.endpoint.type] \hfill \\
          This is the JMS endpoint type used to send node inventory updates from the agents. 
          \begin{description}
            \item[Default Value] topic 
            \item[Type] Private 
          \end{description}
          
        \item[ducc.agent.node.inventory.publish.rate] \hfill \\
          This is the rate at which node inventory updates are published in milliseconds. 

          If the inventory has not changed since the last update the agent bypasses sending the 
          update, up to a maximum of ducc.agent.node.inventory.publish.rate.skip times. 
          \begin{description}
            \item[Default Value] 10000 
            \item[Type] Tuning 
          \end{description}
          
        \item[ducc.agent.node.inventory.publish.rate.skip] \hfill \\
          This is the number of times the agent will bypass publishing its node inventory if the 
          inventory has not changed. 
          \begin{description}
            \item[Default Value] 30 
            \item[Type] Tuning 
          \end{description}
          
        \item[ducc.agent.launcher.thread.pool.size] \hfill \\
          This is establishes the size of the agent's threadpool used to manage spawned processes. 
          \begin{description}
            \item[Default Value] 10 
            \item[Type] Tuning 
          \end{description}
          
        \item[ducc.agent.launcher.use.ducc.spawn] \hfill \\
          This specifies whether to launch job processes via ducc\_ling. When set to false the process 
          is launched directly as a child of the agent. Log indirection is not performed, the working 
          directory is not set, and the the process does not change its identity to that of the submitter. 
          This property is intended for the use of DUCC developers. 
          \begin{description}
            \item[Default Value] true 
            \item[Type] Private 
          \end{description}
          
        \item[ducc.agent.launcher.ducc\_spawn\_path] \hfill \\
          This property specifies the full path to the ducc\_ling utility. During installation ducc\_ling 
          is normally moved to local disk and given setuid-root privileges. Use this property to tell 
          the DUCC agents the location of the installed ducc\_ling. 
          \begin{description}
            \item[Default Value] \ducchome/admin/ducc\_ling 
            \item[Type] Tuning             
          \end{description}
          
        \item[ducc.agent.launcher.process.stop.timeout] \hfill \\
          This property specifies the time, in milliseconds, the agent should wait before forcibly 
          terminating a job process (JP) after an attempted graceful shutdown. If the child process 
          does not terminate in the specified time, it is forcibly terminated with kill -9. 

          This type of stop can occur because of preemption or system shutdown. 
          \begin{description}
            \item[Default Value] 60000 
            \item[Type] Tuning 
          \end{description}
          
        \item[ducc.agent.launcher.process.init.timeout] \hfill \\
          This property specifies the time, in milliseconds, that the agent should wait for a job 
          process (JP) to complete initialization. If initialization is not completed in this time the 
          process is terminated and and InitializationTimout status is send to the job driver (JD) 
          which decides whether to retry the process or terminate the job. 

          \begin{description}
          \item[Default Value] 7200000 
          \item[Type] Tuning 
          \end{description}
          

        \item[ducc.agent.share.size.fudge.factor] \hfill \\

          The DUCC agent monitors the size of the resident memory of its spawned processes. If a 
          process exceeds its declared memory size by any significant amount it is terminated and 
          a ShareSizeExceeded message is sent. The Job Driver counts this towards the maximum 
          errors for the job and will eventually terminate the job if excessive such errors occur. 

          This property defines the percentage over the declared memory size that a process is 
          allowed to grow to before being terminated. 

          To disable this feature, set the value to -1. 
          \begin{description}
            \item[Default Value] 5 
            \item[Type] Tuning 
          \end{description}
          
          \item[ducc.agent.rogue.process.user.exclusion.filter] \hfill \\
          \phantomsection\label{itm:props-rogue.user}

            The DUCC Agents scan nodes for processes that should not be running; for example, 
            a job may have left a 'rogue' process alive when it exits, or a user may log in to a node 
            unexpectedly. These processes are reported to the administrators via the webserver for 
            possible action. 

            This configuration parameter enumerates userids which are ignored by the rogue-process 
            scan. 
            \begin{description}
            \item[Default Value] root,posstfix,ntp,nobody,daemon,100 
            \item[Type] Tuning 
            \end{description}
            
          \item[ducc.agent.rogue.process.exclusion.filter] \hfill \\
          \phantomsection\label{itm:props-rogue.process}
            The DUCC Agents scan nodes for processes that should not be running; for example, 
            a job may have left a 'rogue' process alive when it exits, or a user may log in to a node 
            unexpectedly. These processes are reported to the administrators via the webserver for 
            possible action. 

            This configuration parameter enumerates processes by name which are ignored by the 
            rogue process detector. 

            \begin{description}
              \item[Default Value] sshd:,-bash,-sh,/bin/sh,/bin/bash,grep,ps 
              \item[Type] Tuning 
            \end{description}
            

          \item[ducc.agent.launcher.cgroups.enable] \hfill \\
            \phantomsection\label{itm:props-agent.cgroups.enable} Enable or disable CGroups support.
            If CGroups are not installed on a specific machine, this is ignored.

            With CGroups the RSS for a managed process (plus any children processes it may spawn) is
            limited to the allocated share size. Additional memory use goes to swap space. DUCC
            monitors and limits swap use to the same proportion of total swap space as allocated
            share size is to total RAM. If a process exceeds its allowed swap space it is terminated
            and a ShareSizeExceeded message is sent to the Job Driver.

            Nodes not using CGroups fall back to the ducc.agent.share.size.fudge.factor.

            \begin{description}
              \item[Default Value] true
              \item[Type] Tuning 
            \end{description}
            
          \item[ducc.agent.launcher.cgroups.utils.dir] \hfill \\
            \phantomsection\label{itm:ducc.agent.launcher.cgroups.utils.dir}
            Location of CGroups programs, like cgexec. If CGroups are not
            installed on a specific machine, this is ignored.

            Depending on the OS, CGroups programs may be installed in different places. 
            Provide a comma separated list of directories the agent should search to find the programs.  

            \begin{description}
              \item[Default Value] /usr/bin
              \item[Type] Tuning 
            \end{description}



          \item[ducc.agent.exclusion.file] \hfill \\
            \phantomsection\label{itm:props-agent.cgroups.exclusion}
            This specifies the exclusion file to enable node based exclusion for various
            features.  Currently only CGroup exclusion is supported.

            The exclusion file has one line per agent of the form:
\begin{verbatim}
<node>=cgroups
\end{verbatim}
            If the keyword ``cgrouops'' is found, the node is excluded from CGroup
            support. 
          
            \begin{description}
              \item[Default Value] Not configured.
              \item[Type] Tuning 
            \end{description}
            

          \end{description}
      

\subsection{Process Manager Properties}

    \begin{description}

      \item[ducc.pm.configuration.class] \hfill \\
        This is the name of the pluggable java class used to implement the DUCC Process 
        Manager. 
        \begin{description}
          \item[Default Value] org.apache.uima.ducc.pm.config.ProcessManagerConfiguration 
          \item[Type] Private
        \end{description} 
        
      \item[ducc.pm.request.endpoint] \hfill \\
        This is the endpoint through which process manager receive state from the Orchestrator. 
        \begin{description}
          \item[Default Value] ducc.pm 
          \item[Type] Private 
        \end{description}
        
      \item[ducc.pm.request.endpoint.type] \hfill \\
        This is the JMS endpoint type used for state messages sent by the Orchestrator. 
        \begin{description}
          \item[Default Value] queue 
          \item[Type] Private 
        \end{description}
        
      \item[ducc.pm.state.update.endpoint] \hfill \\
        This is the endpoint through which process manager sends its heartbeat. The main receiver 
        is the Web Server for it's daemon status page. 
        \begin{description}
          \item[Default Value] ducc.pm.state 
          \item[Type] Private 
        \end{description}
        
      \item[ducc.pm.state.update.endpoint.type] \hfill \\
        This is the JMS endpoint type used for process manager heartbeats. The primary receiver 
        is the Web Server for its daemon status page. 
        \begin{description}
          \item[Default Value] topic 
          \item[Type] Private 
        \end{description}
        
      \item[ducc.pm.state.publish.rate] \hfill \\
        This is the rate at which the process manager publishes its heartbeat, in milliseconds. 
        \begin{description}
        \item[Default Value] 25000 
        \item[Type] Private 
        \end{description}
        

      \end{description}
      

\subsection{Job Process Properties}

    \begin{description}

      \item[ducc.uima-as.configuration.class] \hfill \\
        This is the name of the pluggable java class that implements the the UIMA-AS service 
        shell for job processes (JPs). 
        \begin{description}
          \item[Default Value] org.apache.uima.ducc.agent.deploy.uima.UimaAsServiceConfiguration 
          \item[Type] Private 
        \end{description}
        
      \item[ducc.uima-as.endpoint] \hfill \\
        This is the endpoint through which job processes (JPs) receive messages from the Agents. 
        \begin{description}
          \item[Default Value] ducc.job.managed.service 
          \item[Type] Private 
        \end{description}
        
      \item[ducc.uima-as.endpoint.type] \hfill \\
        This is the JMS endpoint type used for messages sent to the JPs from the Agents. 
        \begin{description}
          \item[Default Value] socket 
          \item[Type] Private 
        \end{description}
        
      \item[ducc.uima-as.endpoint.params] \hfill \\
        This configures the JP-to-Agent communication socket. It should be changed only by 
        DUCC developers. 
        \begin{description}
          \item[Default Value] transferExchange=true\&sync=false 
          \item[Type] Private
        \end{description} 
        
      \item[ducc.uima-as.saxon.jar.path] \hfill \\
        This configures the path the required Saxon jar. 
        \begin{description}
          \item[Default Value] file:\$\{DUCC\_HOME\}/lib/saxon/saxon8.jar
          \item[Type] Private 
        \end{description}
        

      \item[ducc.uima-as.dd2spring.xsl.path] \hfill \\
        This configures the path the required dd2spring xsl definitions. 
        \begin{description}
          \item[Default Value] \$\{DUCC\_HOME\}/resources/dd2sprint.xsl
          \item[Type] Private 
        \end{description}
        
      \item[ducc.uima-as.flow\_controller.specifier] \hfill \\
        This configures the pluggable class that implements the default flow controller used in the 
        DUCC job processes (JPs). 
        \begin{description}
          \item[Default Value] DuccJobProcessFC 
          \item[Type] Private 
        \end{description}
      \end{description}
      
