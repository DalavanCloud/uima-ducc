\section{DUCC Class Definitions}
\label{sec:ducc.classes}

    The class configuration file is used by the Resource Manager configure the rules used for job 
    scheduling. See the Resource Manager chapter for a detailed description of the DUCC schedueler. 

    The name of class configuration file is specified in ducc.properties. The default name is 
    ducc.classes [105] and is specified by the property ducc.rm.class.definitions property. 

    This file configures the classes and the associate scheduling rules of each class. It contains 
    properties to declare the following: 
    \begin{enumerate}
      \item The names of each class. 
      \item The default class to use if none is specified with the job. 
      \item The names of all the nodepools. 
      \item For each nodepool, the name of the file containing member nodes. 
      \item A set of properties for each class, declaring the rules enforced by that class. 
    \end{enumerate}

    The general properties are as follows. The default values are the defaults in the system as initially 
    installed. 

    \begin{description}

      \item[scheduling.class.set] \hfill \\
        This defines the set of class names for the installation.  The names themselves are arbitrary
        and correspond to the rules defined in subsequent properties.

        \begin{description}
          \item[Default Value] background low normal high urgent weekly fixed reserve JobDriver 
        \end{description}
          
      \item[scheduling.default.name] \hfill \\
        This is the default class that jobs are assigned to, when not otherwise designated in their 
        submission properties. 
        \begin{description}
          \item[Default Value] normal 
        \end{description}
    \end{description}        
    
    Nodepools are declared with a set of properties to name each nodepool and to name a file for 
    each pool that declares membership in the nodepool. For each nodepool a property of the form 
    scheduling.nodepool.NODEPOOLNAME is declared, where NODEPOOLNAME is one of the 
    declared nodepools. 

    The property to declare nodepool names is as follows: 

    \begin{description}
      \item[scheduling.nodepool] \hfill \\
      This is the list of nodepool names. For example: 
\begin{verbatim}
      scheduling.nodepool = res res1 res2 
\end{verbatim}
      \begin{description}
        \item[Default Value] reserve 
      \end{description}
    \end{description}
        
    This is an example of a declaration of three nodepools. 

\begin{verbatim}
scheduling.nodepool = res res1 res1 
scheduling.nodepool.res = res.nodes 
scheduling.nodepool.res1 = res1.nodes 
scheduling.nodepool.res2 = res2.nodes 
\end{verbatim}
    
    There is no way to enforce priority assignment to any given nodepool. It is possible to declare a 
    "preference", such that the resources in a given nodepool are considered first when searching for 
    nodes. To configure a preference, use the order decorattion on a nodepool specificaion. 

    To declare nodepool order, specify the property {\tt scheduling.nodepool.[poolname].order}. The
    nodepools are sorted numerically according to their order, and pools with lower order are
    searched before pools with higher order. The global nodepool always order "0" so it is usally
    searched first. For example, the pool configuration below establishes a search order of

    \begin{enumerate}
      \item global 
      \item res2 
      \item res 
      \item res1 
    \end{enumerate}
    
    This is an example of a declaration of three nodepools. 

\begin{verbatim}
scheduling.nodepool = res res1 res1 
scheduling.nodepool.res = res.nodes 
scheduling.nodepool.res.order = 4 
scheduling.nodepool.res1 = res1.nodes 
scheduling.nodepool.res1.order = 7 
scheduling.nodepool.res2 = res2.nodes 
scheduling.nodepool.res2.order = 2 
\end{verbatim}
    
    For each class named in scheduling.class.set a set of properties is specified, defining the rules 
    implemented by that class. Each such property is of the form 

\begin{verbatim}
scheduling.class.CLASSNAME.RULE = VALUE 
\end{verbatim}
    
    where 
    \begin{description}
      \item[CLASSNAME] specifies is the name of the class. 
      \item[RULE] specifies rule. Rules are described below. 
      \item[VALUE] specifies the value of the rule, as described below. 
      \end{description}
      
      The rules are: 
      \begin{description}

        \item[policy] \hfill \\
          This is the scheduling policy, required, and must be one of: 
          \begin{itemize}
            \item[] FAIR\_SHARE 
            \item[] FIXED\_SHARE 
            \item[] RESERVE 
          \end{itemize}
            
        \item[share\_weight] \hfill \\
          This is any integer. This is the weighted-fair-share weight for the class as discussed above. It is 
          only used when policy = FAIR\_SHARE. 

        \item[priority] \hfill \\
          This is the evaluation priority for the class as discussed above. This is used for all scheduling 
          policies. 

        \item[cap] \hfill \\
          This is an integer, or an integer with "\%" appended to denote a percentage. It is used for all 
          scheduling classes. 

          This is the class cap as discussed above. It may be an absolute value, in processes (which may 
          comprise more than one share quanta), or it may be specified as a percentage by appending 
          "\%" to the end. When specified as a percentage, it caps the shares allocated to this class as 
          that percentage of the total shares remaining when the class is evaluated.. It does not consider 
          shares that may have been available and assigned to higher-priority classes. 

        \item[nodepool] \hfill \\
          This is the name of the nodepool associated with this class. It must be one of the names 
          declared in the property scheduling.nodepool. 

        \item[prediction] \hfill \\
          Acceptable values are true and false. When set to true the scheduler uses prediction when 
          allocating shares. It is only used when policy = FAIR\_SHARE. 

        \item[prediction.fudge] \hfill \\
          Acceptable values are any integer, denoting milliseconds. This is the prediction fudge as 
          discussed above. It is only used when policy = FAIR\_SHARE. 

        \item[expand.by.doubling] \hfill \\
          Acceptable values are true and false. When set to true the scheduler doubles a job's shares 
          up to it's fair-share when possible, as discussed above. It is only used when policy = 
          FAIR\_SHARE. 

        \item[expand.by.doubling] \hfill \\
          Acceptable values are true and false. When set to true the scheduler doubles a job's shares up 
          to it's fair-share when possible, as discussed above. When set in ducc.classes it overrides the 
          defaults from ducc.properties. It is only used when policy = FAIR\_SHARE. 

        \item[initialization.cap] \hfill \\
          Acceptable values are any integer. This is the maximum number of processes assigned to a job 
          until the first process has successfully completed initialization. To disable the cap, set it to zero 
          0. It is only used when policy = FAIR\_SHARE. 

        \item[max\_processes] \hfill \\
          Acceptable values are any integer. This is the maximum number of processes assigned to a 
          FIXED\_SHARE request. If more are requested, the request is canceled. It is only used when 
          policy = FIXED\_SHARE. If set to 0 or not specified, there is no enforced maximum. 

        \item[max\_machines] \hfill \\
          Acceptable values are any integer. This is the maximum number of machines assigned to a 
          RESERVE request. If more are requested, the request is canceled. It is only used when policy = 
          RESERVE. If set to 0 or not specified, there is no enforced maximum. 

        \item[enforce.memory] \hfill \\
          Acceptable values are true and false. When set to true the scheduler requires that any machine 
          selected for a reservation matches the reservation's declared memory. The declared memory 
          is converted to a number of quantum shares. Only machines whose memory, when converted 
          to share quanta are selected. When set to false, any machine in the configured nodepool is 
          selected. It is only used when policy = RESERVE. 
      \end{description}
          

        
