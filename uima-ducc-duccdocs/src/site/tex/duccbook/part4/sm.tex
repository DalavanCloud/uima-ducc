% 
% Licensed to the Apache Software Foundation (ASF) under one
% or more contributor license agreements.  See the NOTICE file
% distributed with this work for additional information
% regarding copyright ownership.  The ASF licenses this file
% to you under the Apache License, Version 2.0 (the
% "License"); you may not use this file except in compliance
% with the License.  You may obtain a copy of the License at
% 
%   http://www.apache.org/licenses/LICENSE-2.0
% 
% Unless required by applicable law or agreed to in writing,
% software distributed under the License is distributed on an
% "AS IS" BASIS, WITHOUT WARRANTIES OR CONDITIONS OF ANY
% KIND, either express or implied.  See the License for the
% specific language governing permissions and limitations
% under the License.
% 
% Create well-known link to this spot for HTML version
\ifpdf
\else
\HCode{<a name='DUCC_RM'></a>}
\fi
\chapter{Service Management}
\label{chap:sm}
    The only administrative task relating to Service Management is registering
    global pingers, as described in
    \hyperref[subsec:services.pingers]{\em Service Pingers section} of this document.

    A globally-registered service pinger is a properties file that contains only
    service registraton options pertaining to pingers.  This file must be placed
    in DUCC's {\em runtime}/resources/service\_monitors directory.  It may be
    given any name but ``best practices'' would suggest it be named the
    same as the {\em service\_ping\_class}.  Services then use this pinger
    by specifying its filename in their {\em service\_ping\_class} option.
    
    Globally-registered pingers may be run internally as threads within the
    SM, or externally as processes.  To specify that a pinger be run internally,
    add the property 
\begin{verbatim}
    internal = true
\end{verbatim}
    to the registration file.  To specify that it run externally, add the property
\begin{verbatim}
    internal = false
\end{verbatim}
    to the registration file.

    The ``internal'' option is flagged as in illegal option when
    specified in service registrations and all pingers not specified as
    ``internal'' are run as {\em external} processes managed by the SM.

    Best practices dictate that the filename of an {\em external} pinger contain the
    postfix {\em .external} to clearly identify it as external.  

    As an example, the default UIMA-AS pinger is supplied in the global registry
    under the two names:
\begin{verbatim}
    org.apache.uima.ducc.sm.UimaAsPing
    org.apache.uima.ducc.sm.UimaAsPing.external
\end{verbatim}

    Note that users may override any of the properties in globally-registered
    {\em external} pingers, but only the {\em service\_ping\_arguments} of an {\em internal}
    pinger to protect its integrity by specify that argument in their own
    service registrations.

    By default external pingers are launched with the same JRE as used by the DUCC daemons but if the
    registration contains a entry for the JAVA\_HOME environment variable, that JRE will be used instead.

    For backwards compatibility the default UIMA-AS pinger is also registered under its old name:
\begin{verbatim}
    org.apache.uima.ducc.cli.UimaAsPing
    org.apache.uima.ducc.cli.UimaAsPing.external
\end{verbatim}
