% 
% Licensed to the Apache Software Foundation (ASF) under one
% or more contributor license agreements.  See the NOTICE file
% distributed with this work for additional information
% regarding copyright ownership.  The ASF licenses this file
% to you under the Apache License, Version 2.0 (the
% "License"); you may not use this file except in compliance
% with the License.  You may obtain a copy of the License at
% 
%   http://www.apache.org/licenses/LICENSE-2.0
% 
% Unless required by applicable law or agreed to in writing,
% software distributed under the License is distributed on an
% "AS IS" BASIS, WITHOUT WARRANTIES OR CONDITIONS OF ANY
% KIND, either express or implied.  See the License for the
% specific language governing permissions and limitations
% under the License.
% 
\documentclass[letterpaper]{article}

\def\DUCCSTANDALONE{}

% space between paragraphs
\usepackage{parskip}

\usepackage{hyperref}
% Margins
\usepackage[top=1in, bottom=.75in, left=.75in, right=.75in ]{geometry}

\usepackage{xcolor}
\usepackage{datetime}

% turn off section numbering
\setcounter{secnumdepth}{0}

\title{Reliable DUCC - Design}
\author{Written and maintained by the Apache \\
UIMA\texttrademark Development Community}
\date{}
\begin{document}
\maketitle

%Licensed to the Apache Software Foundation (ASF) under one
%or more contributor license agreements.  See the NOTICE file
%distributed with this work for additional information
%regarding copyright ownership.  The ASF licenses this file
%to you under the Apache License, Version 2.0 (the
%"License"); you may not use this file except in compliance
%with the License.  You may obtain a copy of the License at
%
%   http://www.apache.org/licenses/LICENSE-2.0
%
%Unless required by applicable law or agreed to in writing,
%software distributed under the License is distributed on an
%"AS IS" BASIS, WITHOUT WARRANTIES OR CONDITIONS OF ANY
%KIND, either express or implied.  See the License for the
%specific language governing permissions and limitations
%under the License.
Copyright \copyright~ 2012 The Apache Software Foundation

Copyright \copyright~ 2012 International Business Machines Corporation

{\addtolength{\leftskip}{10 mm}
     \paragraph{License and Disclaimer}
     The ASF licenses this documentation to you under the Apache License, Version 2.0 (the "License");
     you may not use this documentation except in compliance with the License.  You may obtain a copy of
     the License at
              
     \url{http://www.apache.org/licenses/LICENSE-2.0}
     
     Unless required by applicable law or agreed to in writing, this documentation and its contents are
     distributed under the License on an "AS IS" BASIS, WITHOUT WARRANTIES OR CONDITIONS OF ANY KIND,
     either express or implied.  See the License for the specific language governing permissions and
     limitations under the License.

     \paragraph{Trademarks}     
     All terms mentioned in the text that are known to be trademarks or service marks have been
     appropriately capitalized.  Use of such terms in this book should not be regarded as affecting the
     validity of the the trademark or service mark.

}

\vspace{.5in}

\newdateformat{mydate}{%
\twodigit{\THEDAY}-\twodigit{\THEMONTH}-\THEYEAR}

Publication date: \mydate\today

\newpage

% common macros in a single place
% These are used in the main file, and in the stand-alone wrappers
\newcommand{\distro}{apache-uima-ducc-0.8.0-SNAPSHOT.tgz}
\newcommand{\duccruntime}{\emph{ducc\_runtime}}
\newcommand{\ducchome}{\$DUCC\_HOME}
\newcommand{\todo}{{\sc \Large TODO:}  }
\newcommand{\note}{{\em Note:}  }


\section{Multiple DUCC head nodes}

This first major section describes support for multiple active DUCC head nodes.

\subsection{Introduction}
    DUCC can be configured to run reliably by having multiple head nodes,
    comprising one {\em master} and one or more {\em backup} head nodes.
    DUCC exploits Linux {\em keepalived} virtual IP addressing to enable
    this capability.
    
    The advantages are that if the {\em master} DUCC host becomes
    unusable, the {\em backup} DUCC can take over seamlessly
    such that active distributed Jobs, Reservations, Managed Reservations 
    and Services continue uninterrupted.  Take over also facilitates
    continued acceptance of new submissions and monitoring of new and
    existing submissions without interruption.

\subsection{Daemons}
   Each head node, whether {\em master} or {\em backup}, runs a Broker,
   Orchestrator, PM, RM, and SM.
   
   The Cassandra database is expected to be located on a node(s) separate from the head nodes.
   
   Likewise, the JD node(s) is separate from the head nodes.
   
   The Agents are distributed, as before.
        
\subsection{Configuring Host Machines}    
    See {\em Configuring Simple Virtual IP Address Failover Using Keepalived} 
    which can be found at this web address: 
    \url{https://docs.oracle.com/cd/E37670_01/E41138/html/section_uxg_lzh_nr.html}.

	Sample MASTER /etc/keepalived/keepalived.conf
	
	\begin{verbatim}
    ! Configuration File for keepalived

	vrrp_instance VI_1 {
	    state MASTER
	    interface eth0
	    virtual_router_id 51
	    priority 100
	    advert_int 1
	    authentication {
	        auth_type PASS
	        auth_pass 1111
	    }
	    virtual_ipaddress {
	        192.168.6.253
	    }
	}
   	\end{verbatim}

	Sample BACKUP /etc/keepalived/keepalived.conf
	
	\begin{verbatim}
    ! Configuration File for keepalived

	vrrp_instance VI_1 {
	    state BACKUP
	    interface eth0
	    virtual_router_id 51
	    priority 100
	    advert_int 1
	    authentication {
	        auth_type PASS
	        auth_pass 1111
	    }
	    virtual_ipaddress {
	        192.168.6.253
	    }
	}
   	\end{verbatim}

	Linux Commands
	
	Starting keepalived
	
    \begin{verbatim}
    > sudo service keepalived start
    Starting keepalived:                                       [  OK  ]
   	\end{verbatim}
   	
   	Querying keepalived
	
    \begin{verbatim}
    > /sbin/ip addr show dev eth0
	2: eth0: <BROADCAST,MULTICAST,UP,LOWER_UP> mtu 1500 qdisc mq state UP qlen 1000
	    link/ether 00:21:5e:20:02:84 brd ff:ff:ff:ff:ff:ff
	    inet 192.168.3.7/16 brd 192.168.255.255 scope global eth0
	    inet 192.168.6.253/32 scope global eth0
	    inet6 fe80::221:5eff:fe20:284/64 scope link 
	       valid_lft forever preferred_lft forever
   	\end{verbatim}

	Stopping keepalived
	
    \begin{verbatim}
    > sudo service keepalived stop
    Stopping keepalived: 
   	\end{verbatim}

\subsection{Configuring DUCC}  
    To configure DUCC to run reliable, one required property must
    be configured in the {\em site.ducc.properties} file.  Example:
    
    \begin{verbatim}
	ducc.head = 192.168.6.253
   	\end{verbatim}
    
    Use the virtual IP address configured for your host machines keepalived. 
    Use of the DNS name is also supported.  
    
\subsection{Webserver}

	Webserver for Master

	The {\em master} DUCC Webserver will display all pages normally with additional
	information in the heading upper left:
	
	reliable: \textbf{master}
	
	Webserver for Backup
	
	The {\em backup} DUCC Webserver will display some pages normally with additional
	information in the heading upper left:
	
	\underline{reliable}: \textbf{backup}
   	
   	Hovering over \underline{reliable} will yield the following information:
   	{\em Click to visit master}
   	
   	Several pages will display the following information (or similar):
   	
   	\begin{verbatim}
	no data - not master
   	\end{verbatim}

\section{Database}

	Configure the database to be on a separate machine from the reliable DUCC head nodes.
	In {\em site.ducc.properties} specify:
	
	\begin{verbatim}
	# Database location
    ducc.database.host = dbhost123
    ducc.database.jmx.host = dbhost123
    ducc.database.automanage = false
   	\end{verbatim}
	
	Two new convenience commands are available for DUCC administrators.  
	When issued by the DUCC administrator, these commands will start and stop the database 
	on the database host machine regardless of the value specified for 
	{\em ducc.database.automanage} in {\em site.ducc.properties}.
	
	\begin{verbatim}
	admin/db\_start
	afmin/db\_stop
   	\end{verbatim}
   	
   	The existing administrator commands {\em start\_ducc} and {\em stop\_ducc} will
   	honor the value specified for {\em ducc.database.automanage} in {\em site.ducc.properties}.

\subsection{Code changes}

The key changes include a new script (see ducc\_head\_mode.py) to 
interact with Linux to determine virtual IP address status and 
corresponding Java code (see common.head.ADuccHead.java)
that interprets the status to make transitions between 
{\em master} and {\em backup} states.

\subsubsection{new scripts}

\textbf{ducc\_head\_mode.py}

This is a new script employed at runtime by the various daemons to
determine the current mode of operation.  Status is determined 
though invocation of this script upon receipt of each Orchestrator
publication.

   \begin{verbatim}
    # purpose:    determine reliable ducc status
    # input:      none
    # output:     one of { unspecified, master, backup }
    # operation:  look in ducc.properties for relevant keywords
    #             and employ linux commands to determine if system
    #             has matching configured virtual IP address
   \end{verbatim}

\subsubsection{existing and new scripts}

{\renewcommand\labelitemi{}
\begin{itemize}
  \item \textbf{ducc\_post\_install} - no need to create webserver request log directory
  \item \textbf{ducc\_util.py} - incorporate host name into cassandra.pid, cassandra.console path; broker host must be local host; head node must be eligible with respect to keepalived.conf; head node local components are all daemons except Database and Agents; fix remote db stop; honor ducc.database.automanage flag in site.ducc.properties
  \item \textbf{ducc.py} - incorporate host name into cassandra.pid, cassandra.console path
  \item \textbf{start\_ducc.py} - head node local components must start on eligible local host
  \item \textbf{db\_start} - (new) start DUCC database
  \item \textbf{db\_stop} - (new) stop DUCC database
  \item \textbf{start\_sim} - (example) honor database autostart flag in ducc.properties
  item \textbf{stop\_sim} - (example) honor database autostart flag in ducc.properties
\end{itemize}
}

\subsubsection{configuration files}

\textbf{ducc.properties}  

   \begin{verbatim}
    # The name of the node where DUCC runs.
    # This property declares the node where the DUCC administrative processes run (Orchestrator,
    # Resource Manager, Process Manager, Service Manager).  This property is required and MUST be
    # configured in new installation.  The installation script ducc_post_install initializes this 
    # property to the node the script is executed on.
    # Reliable DUCC: if running reliably, then this value must be the same as that specified
    # for the virtual_ipaddress in /etc/keepalived/keepalived.conf.  DUCC CLI and Agents employ
    # this value to connect to the current reliable DUCC head node.
    ducc.head = <head-node>
   \end{verbatim}

Although not strictly true, the Orchestrator, RM, SM, PM, Webserver and Broker "must"
all be configured on the head node.  Reliable DUCC may work with other configurations,
but it has not been tested as such.

	\begin{verbatim}
	# If set to true, DUCC will start and stop the Cassandra database as part of its normal
	# start/stop scripting.  
	ducc.database.automanage = true
	\end{verbatim}

\textbf{log4j.xml}

   \begin{verbatim}
    Add DUCC\_NODENAME to log file name for OR, RM, PM, SM, and system-events.
    This allows reliable DUCC head nodes to share the same ducc\_runtime directory
    in the filesystem without collisions.
   \end{verbatim}

\subsubsection{agent}

{\renewcommand\labelitemi{}
\begin{itemize}
  \item \textbf{DuccWorkHelper} - use virtual IP address configured as ducc.head node
  \item \textbf{AgentEventListener} - ignore any incoming publications from backup producer
  \item \textbf{CGroupsTest} - employ changed DuccIdFactory signature
  \item \textbf{ServiceTester} - broker must be on ducc.head node
\end{itemize}
}

\subsubsection{cli}

{\renewcommand\labelitemi{}
\begin{itemize}
  \item \textbf{DuccMonitor} - use WS node or virtual IP address configured as ducc.head node
  \item \textbf{DuccUiUtilities} - use virtual IP address configured as ducc.head node (to submit, cancel..)
\end{itemize}
}

\subsubsection{common}

{\renewcommand\labelitemi{}
\begin{itemize}
  \item \textbf{AbstractDuccComponent} - remove commented-out code, remove print to console, head node local components are all daemons except Database and Agents
  \item \textbf{ADuccHead} - abstract class with reliable DUCC share functionality
  \item \textbf{IDuccHead} - reliable DUCC interface
  \item \textbf{IDuccEnv} - remove DUCC\_LOGS\_WEBSERVER\_DIR, not used
  \item \textbf{IStateServices} - database access control RW or RO
  \item \textbf{NullStateServices} - database access control RW or RO
  \item \textbf{StateServices} - database access control RW or RO
  \item \textbf{DuccDaemonRuntimeProperties} - incorporate hostname into logs directory location
  \item \textbf{InetHelper} - incorporate hostname into logs directory location
  \item \textbf{DuccPropertiesHelper} - fetch virtual IP address configured as ducc.head node
  \item \textbf{DuccPropertiesResolver} - Remove key ducc.broker.hostname, broker must be on ducc.head node
  \item \textbf{IDuccLoggerComponents} - Missing PM abbreviation
  \item \textbf{DuccIdFactory} - improved (generalized) to handle DB persisted sequence numbering
\end{itemize}
}

\subsubsection{database}

{\renewcommand\labelitemi{}
\begin{itemize}
  \item \textbf{IDuccHead} - reliable DUCC interface
  \item \textbf{DbOrchestratorProperties} - support for OR properties table
  \item \textbf{IDbOrchestratorProperties} - interface of OR properties table
  \item \textbf{IOrchestratorProperties} - interface for OR properties
  \item \textbf{IOrchestratorProperties} - database access control RW or RO
\end{itemize}
}

\subsubsection{orchestrator}

{\renewcommand\labelitemi{}
\begin{itemize}
  \item \textbf{DuccHead} - loggable wrapper around common.ADuccHead
  \item \textbf{OrchestratorCommonArea} - add restart capability for transition to {\em master}
  \item \textbf{OrchestratorComponent} - reject requests from CLI and JD and publications for Agents when not {\em master}, employ DB for Job and publication number persistence, use active workMap from common area, tag publication with node identity and producer state {\em master} or {\em backup}, make transitions between {\em master} and {\em backup} states
  \item \textbf{OrchestratorRecovery} - employ changed DuccIdFactory initialization requirements
  \item \textbf{ReservationFactory} - employ changed DuccIdFactory signature
  \item \textbf{StateJobAccounting} - log job state changes
  \item \textbf{StateManager} - use active workMap from common area
  \item \textbf{WorkMapHelper} - adding logging 
  \item \textbf{AOrchestratorCheckpoint} - refactor checkpointing, suspend when {\em backup} resume when {\em master}
  \item \textbf{IOrchestratorCheckpoint} - refactor checkpointing
  \item \textbf{OrchestratorCheckpoint} - refactor checkpointing
  \item \textbf{OrchestratorCheckpointDb} - refactor checkpointing
  \item \textbf{OrchestratorCheckpointFile} - refactor checkpointing
  \item \textbf{OrchestratorConfiguration} - employ changed DuccIdFactory for publication sequence numbering
  \item \textbf{OrDbDuccWorks} - specification to DB only when {\em master}
  \item \textbf{OrDbDuccWorks} - orchestrator properties to DB only when {\em master}
  \item \textbf{OrchestratorEventListener} - record to system events log daemon switches between {\em backup} and {\em master} 
  \item \textbf{ReservationFactory} - employ changed DuccIdFactory for Job numbering\item \textbf{ReservationFactory} - employ changed DuccIdFactory signature
  \item \textbf{JdScheduler} - suspend JD host management when {\em backup} resume when {\em master}
  \item \textbf{HealthMonitor} - use active workMap from common area
  \item \textbf{MaintenanceThread} - use active workMap from common area
  \item \textbf{AOrchestratorState} - refactor orchestrator state managements from files to DB
  \item \textbf{DuccWorkIdFactory} - refactor orchestrator state managements from files to DB
  \item \textbf{IOrchestratorState} - refactor orchestrator state managements from files to DB
  \item \textbf{OrchestratorState} - refactor orchestrator state managements from files to DB
  \item \textbf{OrchestratorStateDb} - refactor orchestrator state managements from files to DB
  \item \textbf{OrchestratorStateDbConversion} - refactor orchestrator state managements from files to DB
  \item \textbf{OrchestratorStateFile} - refactor orchestrator state managements from files to DB
  \item \textbf{AOrchestratorStateJson} - refactor orchestrator state managements from files to DB
  \item \textbf{SystemEventsLogger} - record all CLI interactions in system events log
  \item \textbf{TestSuite} - print whether {\em backup} or {\em master}
\end{itemize}
}

\subsubsection{pm}

{\renewcommand\labelitemi{}
\begin{itemize}
  \item \textbf{DuccHead} - loggable wrapper around common.ADuccHead
  \item \textbf{ProcessManagerComponent} - make transitions between {\em master} and {\em backup} states
\end{itemize}
}

\subsubsection{sm}

{\renewcommand\labelitemi{}
\begin{itemize}
  \item \textbf{DuccHead} - loggable wrapper around common.ADuccHead
  \item \textbf{ServiceHandler} - resume operations when state is {\em master}, quiesce operations when state is {\em backup}
  \item \textbf{ServiceManagerComponent} - make transitions between {\em master} and {\em backup} states, reject requests when in {\em backup} state, employ changed DuccIdFactory signature 
 \textbf{ServiceSet} - handle new state {\em Dispossessed}
\end{itemize}
}

\subsubsection{transport}

{\renewcommand\labelitemi{}
\begin{itemize}
  \item \textbf{JobDriverStateExchanger} - use virtual IP address configured as ducc.head node
  \item \textbf{AbstractDuccEvent} - tag publications with producer host identity and state {\em master} or {\em backup}
  \item \textbf{DaemonDuccEvent} - switch to {\em master} or {\em backup} state for recording to system event log
  \item \textbf{DuccEvent} - add events SWITCH\_TO\_MASTER and SWITCH\_TO\_BACKUP
  \item \textbf{JdEvent} - interrogate publications producer state {\em master} or {\em backup}
  \item \textbf{IService} - add service state Dispossessed, Service is not controlled by this Service Manager
\end{itemize}
}

\subsubsection{webserver}

{\renewcommand\labelitemi{}
\begin{itemize}
  \item \textbf{BrokerHelper} - use local host name to find co-located broker
  \item \textbf{DuccBoot} - make boot reusable for switch to {\em master}
  \item \textbf{DuccData} - create reset function for switch to {\em master}
  \item \textbf{DuccHead} - loggable wrapper around common.ADuccHead
  \item \textbf{WebServerComponent} - make transitions between {\em master} and {\em backup} states; incorporate hostname info logs directory location
  \item \textbf{WebServerConfiguration} - make boot reusable for switch to {\em master}
  \item \textbf{DuccHandler} - servlet to produce reliable DUCC state {\em master} or {\em backup}
  \item \textbf{DuccHandlerClassic} - servlets to produce \"no data - not master\" when appropriate
  \item \textbf{DuccHandlerJsonFormat} - servlets to produce \"no data - not master\" when appropriate
  \item \textbf{DuccWebServer} - add method getPort; use host as part of request log directory path; incorporate hostname info logs directory location
   \item \textbf{DuccWebServerHelper} - incorporate hostname info logs directory location
  \item \textbf{c4-ducc-mon.jsp} - web page header location for reliable DUCC state
  \item \textbf{ducc.js} - web page header updating for reliable DUCC state
\end{itemize}
}

\subsubsection{examples}

{\renewcommand\labelitemi{}
\begin{itemize}
  \item \textbf{start\_sim} - broker must be on head node
\end{itemize}
}


\section{Installing and Cloning}

This second major section describes support for installation of head node master and backup(s).

TBD

\section{Autostart}

This third major section describes support for autostart of head node and agent daemons.

TBD

\section{Monitoring and Switching}

This fourth major section describes support monitoring of multiple head nodes and switching to an alternate when the primary is dysfunctional.

TBD

\end{document}
