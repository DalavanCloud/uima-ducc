% Create well-known link to this spot for HTML version
\ifpdf
\else
\HCode{<a name='DUCC_TERMINOLOGY'></a>}
\fi
\chapter{Glossary}

\begin{description}
\item[Autostart Service] An autostart service is a registered service that is started automatically
  by DUCC when the DUCC system is booted.

\item[Dependent service or job] A dependent service or job is a service or job that specifies one
  or more service dependencies in their job specification. The service or job is dependent upon the
  referenced service being operational before being started by DUCC.

\item[DUCC] Distributed UIMA Cluster Computing.

\item[Implicit service] An emplicit service is a service that is started externally to DUCC but
  referenced by some dependent service or job.  DUCC will attempt to contact the service using
  the dependency string.  If contact is successful the job is started, otherwise it is 
  terminated before resources are allocated to it.

\item[Registered service] A registered service is a service that is registered with DUCC. DUCC
  saves the service specification and fully manages the service, insuring it is running when needed,
  and shutdown when not.

\item[Start-by-Reference Service] An on-demand service is a registered service that is not started when DUCC
  is started. Instead, the service is started when referenced in some job or services service
  dependency, and stopped when the referencing entity exits.

\item[Service Instance] A service instance is one physical process which runs a CUSTOM or UIMA-AS
  service.  Note that UIMA-AS services may be scaled-out to comprise more than one service instance.

\item[Orchestrator (OR)] The Orchestrator manages the lifecycle of all entities within DUCC.

\item[Process Manager (PM) ] The Process Manager coordinates distribution of work among the Agents.

\item[Resource Manager (RM) ] The Resource Manager schedules physical resources for DUCC work.

\item[Service Endpoint] In DUCC, the service endpoint provides a unique identifier for a service
  and in the case of UIMA-AS services, a well-known address for contacting the service. For CUSTOM
  services, the endpoint is of the form CUSTOM:string where string is any alphanumeric string
  provided by the service owner. For UIMA-AS services, the endpoint is of the form UIMA-AS:queue
  name:ActiveMQ-broker-URL.

\item[Service Manager (SM)] The Service Manager manages the life-cycles of UIMA-AS and CUSTOM
  services. It coordinates registration of services, starting and stopping of services, and ensures
  that services are available and remain available for the lifetime of the jobs.  Note that the
  Orchestrator manages the individual service instances; the Service Manager manages the collection
  of instances which comprise a service.

\item[Agent] DUCC Agent processes run on every node in the system. The Agent receives orders to
  start and stop processes on each node. Agents monitors nodes, sending heartbeat packets with node
  statistics to interested components (such as the RM and web-server). If CGroups are intstalled in
  the cluster, the Agent is responsible for managing the CGroups for each job process. All processes
  other than the DUCC management processes are are managed as children of the agents.

\item[DUCC-MON]  DUCC-MON is the DUCC web-server.

\item[Job Driver (JD)]The Job Driver is a thin wrapper that encapsulates a Job's Collection
  Reader. The JD executes as a process that is scheduled and deployed by DUCC.

\item[Job Process (JP)] The Job Process is a thin wrapper that encapsulates a job's pipeline
  components. The JP executes in a process that is scheduled and deployed by DUCC.

\item[Job specification] The Job Specification is a collection of properties that describe work to be
  scheduled and deployed by DUCC. It
  identifies the UIMA components (CR, AE, etc) that comprise the job and the ystem-wide
  properties of the job (classpaths, RAM requirements, etc). 

\item[Job] A DUCC job consists of the components required to deploy and execute a UIMA pipeline over
  a computing cluster. It consist of a JD to run the Collection Reader, a set of JPs to run the UIMA
  AEs, and a Job Specification to describe how the parts fit together.

\item[Share Quantum] The DUCC scheduler abstracts the nodes in the cluster as a single large
  congomerate of resources: memory, processor cores, etc.  The scheduler logically decomposes 
  the collection of resources into some number of equal-sized atomic units.  Each unit of work requiring
  resources is apportioned one or more of these atomic units.  The smallest possible atomic 
  unit is called the {\em share quantum}, or simply, {\em share}.

\item[Process]A process is one physical process executing on a machine in the DUCC cluster. DUCC
  jobs are comprised of one or more processes (JDs and JPs).  Each process is assigned one or
  more {\em shares} by the DUCC scheduler.

\item[Weighted Fair Share] A weighted fair share calculation is used to apportion resources
  equitably to the outstanding work in the system.  In a non-weighted fair-share system, all
  work requests are given equal consideration to all resources.  To provide some (``more important'')
  work more than equal resources, weights are used to give larger proportions of the resources to
  some classes of work.

\item[Work Items] A DUCC work item is one unit of work to be completed in a single DUCC process. It is
  usually initiated by the submission of a single CAS from the CR to a UIMA service. It could be
  thought of as a single "question" to be answered by a UIMA analytic. Usually each DUCC JP executes
  many work items per job.
\end{description}


