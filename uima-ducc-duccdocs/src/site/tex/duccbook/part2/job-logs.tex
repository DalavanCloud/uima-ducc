% 
% Licensed to the Apache Software Foundation (ASF) under one
% or more contributor license agreements.  See the NOTICE file
% distributed with this work for additional information
% regarding copyright ownership.  The ASF licenses this file
% to you under the Apache License, Version 2.0 (the
% "License"); you may not use this file except in compliance
% with the License.  You may obtain a copy of the License at
% 
%   http://www.apache.org/licenses/LICENSE-2.0
% 
% Unless required by applicable law or agreed to in writing,
% software distributed under the License is distributed on an
% "AS IS" BASIS, WITHOUT WARRANTIES OR CONDITIONS OF ANY
% KIND, either express or implied.  See the License for the
% specific language governing permissions and limitations
% under the License.
% 
% Create well-known link to this spot for HTML version
\ifpdf
\else
\HCode{<a name='DUCC_LOGS'></a>}
\fi
\chapter{Job Logs}
\label{chap:job-logs}

\paragraph{Overview}The DUCC logs are managed by Apache log4j.  The DUCC log4j configuration file is found in
\duccruntime/log4j.xml. It is not in the scope of this document to describe log4j or its
configuration mechanism. Details on log4j can be found at \url{http://logging.apache.org/log4j}.

The "user logs" are the Job Driver (JD) and Job Process (JP) logs. There is one log for each process 
of a job. The JD log is divided between two physical files: 

\begin{enumerate}
   \item Stdout and default UIMA logging output written by the UIMA collection reader.      
   \item The diagnostic logs written by the DUCC JD wrapper around the job's collection reader. 
\end{enumerate}

\paragraph{Contents of the Log Directory} A number of other useful files are written to the log directory: 

\begin{enumerate}
    \item A properties file containing the full job specification for the job. This includes all the 
      parameters specified by the user as well as the default parameters. This file is called
      {\tt job-specification.properties.}
    \item The UIMA pipeline descriptor constructed by DUCC that describes the process that is 
      dispatched to each Job Process (JP). The name of this file is of the form 
\begin{verbatim}
         JOBID-uima-ae-descriptor-PROCESS.xml 
\end{verbatim}
      where 
      \begin{description}
          \item[JOBID] This is the numerical id of the job as assigned by DUCC.
          \item[PROCESS] This is the process id of the Job Driver (JD) process.
      \end{description}      

    \item The UIMA-AS service descriptor that defines the process that defines the job as as UIMAAS 
      service. The name of this file is of the form 
\begin{verbatim}
         JOBID-uima-as-dd-PROCESS.xml 
\end{verbatim}
      
      where 
      \begin{description}
         \item[JOBID] This is the numerical id of the job as assigned by DUCC.
         \item[PROCESS] This is the process id of the Job Driver (JD) process.
      \end{description}
      
    \item A colllection of gzipped ``json'' files containing the performance breakdown of the job.
      These can be read and formatted using \hyperref[sec:cli.ducc-perf-stats]{ducc\_perf\_stats.}
\end{enumerate}

\paragraph{Job Process Logs}
The Job Process logs are written to the configured log directory.  There is one job process log
for every job processes started for the job.  The log names are of the following form:
\begin{verbatim}
         JOBID-TYPE-NODE-PROCESS.log 
\end{verbatim}
where 
\begin{description}
\item[JOBID] This is the numerical id of the job as assigned by DUCC.
\item[TYPE] This is either the string "UIMA" for JP logs, or "JD" for JD logs.
\item[NODE] This is the name of the machine where the process ran.
\item[PROCESS] This is the Unix process id of the process on the indicated node.
\end{description}

\paragraph{Job Driver Logs}
There are several Job Driver logs.
   988-JD-agent86-1-58087.log
   jd.out.log
   jd.err.log

\paragraph{Sample Log Directory}
This shows the contents a sample log directory for a small job that consisted of two processes.

\begin{verbatim}
    100-JD-node290-1-29383.log 
    100-uima-ae-descriptor-29383.xml 
    100-uima-as-dd-29383.xml 
    100-UIMA-node290-2-32766.log 
    100-UIMA-node291-63-13594.log 
    jd.out.log 
    job-specification.properties 
    job-performance-summary.json.gz
    job-processes-data.json.gz
    work-item-status.json.gz

\end{verbatim}

In this example, 

\begin{description}
     \item[100-JD-node290-1-29383.log]  \hfill \\
       is the diagnostic JD log, where the JD executed on node
       node290-1 in process 29383.

     \item[100-uima-ae-descriptor-29383.xml]  \hfill \\
       is the UIMA pipeline descriptor describing the
       service process that is launched in each JP, where the JD process is 29383.

     \item[100-uima-as-dd-29383.xml]  \hfill \\
       is the UIMA-AS service descriptor where the client is
       the JD process running in process 29383.

     \item[100-UIMA-node290-2-32766.log]  \hfill \\
       is a JP log for job 100, that ran on node
       node290-2, in process 32766.

     \item[100-UIMA-node291-63-13594.log]  \hfill \\
       is a JP log for job 100, that ran on node
       node291-63, in process 13594

     \item[jd.out.log]  \hfill \\
       is the user's JD log, containing the user's collection reader output.

     \item[job-performance-summary.json.gz]  \hfill \\
       This contains the raw statistics describing
       the operation of each analytic.  It corresponds to \hyperref[sec:performance]{Performance}
       tab of the Job Details page in the Web Server.

     \item[job-process.json.gz]  \hfill \\
       This contains the raw statistics describing
       the performance of each individual job process.  It corresponds \hyperref[sec:ws-processes]{Processes}
       tab of the Job Details page in the Web Server.

     \item[work-item-status.json.gz]  \hfill \\
       This contains the raw statistics describing
       the operation of each individual work-item.  It corresponds to \hyperref[sec:ws-work-items]{Work Items}
       tab of the Job Details page in the Web Server.
 \end{description}
     
