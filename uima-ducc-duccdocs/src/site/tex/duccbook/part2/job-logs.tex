% Create well-known link to this spot for HTML version
\ifpdf
\else
\HCode{<a name='DUCC_LOGS'></a>}
\fi
\chapter{Job Logs}


The DUCC logs are managed by log4j and are configured using ducc\_runtime/log4j.xml. It 
is not in the scope of this document to describe log4j or its configuration mechanism. Details on 
log4j can be found at http://logging.apache.org/log4j/1.2/. 

The "user logs" are the Job Driver (JD) and Job Process (JP) logs. There is one log for each process 
of a job. The JD log is divided between two physical files: 

\begin{enumerate}
  \item The logs and stdout written by the UIMA collection reader. The collection reader uses the 
    UIMA logger which is by default directed to stdout. 

    \item The diagnostic logs written the the DUCC JD wrapper around the job's collection reader. 
      This log is written using log4j. 
\end{enumerate}

A number of other usefiles are written to the log directory: 
\begin{enumerate}

  \item A properties file containing the full job specification for the job. This includes all the 
    parameters specified by the user as well as the default parameters. This file is written to 
    job-specification.properties. 


  \item The UIMA pipeline descriptor constructed by DUCC that describes the process that is 
    dispatched to each Job Process (JP). The name of this file is of the form 

\begin{verbatim}
         JOBID-uima-ae-descriptor-PROCESS.xml 
\end{verbatim}

    where 

    \begin{description}
        \item[JOBID] This is the numerical id of the job as assigned by DUCC.
        \item[PROCESS] This is the process id of the Job Driver (JD) process.
        \end{description}
      
      \item The UIMA-AS service descriptor that defines the process that defines the job as as UIMAAS 
        service. The name of this file is of the form 
\begin{verbatim}
         JOBID-uima-as-dd-PROCESS.xml 
\end{verbatim}
    
        where 
        \begin{description}
            \item[JOBID] This is the numerical id of the job as assigned by DUCC.
            \item[PROCESS] This is the process id of the Job Driver (JD) process.
        \end{description}

      \item A Java serialized object containing the performance breakdown for the job. This is used 
        by the Web Server to display the breakdown. This file is written to job-performancesummary.ser. 
 \end{enumerate}

 The JP logs are written by default to HOME/ducc/logs, where HOME is the submitting user's 
 home directory. In this directory, a subdirectory whose name is the numerical id of the job is 
 created by DUCC, where all logs for the job are written. 
 
 The collection reader's log is written to the file HOME/ducc/logs/JOBID/jd.out.logvia log4j. 
 It is written in multiple generations, and its size is governed by the same log4j configuration file 
 used for the DUCC Daemon processes. The size of each generation and the number of generations 
 is configured in the jdout appender stanza. 

 Each JP log and the diagnostic JD log is of the following form:

\begin{verbatim}
         JOBID-TYPE-NODE-PROCESS.log 
\end{verbatim}
 
 where 

\begin{description}
    \item[JOBID] This is the numerical id of the job as assigned by DUCC.
    \item[TYPE] This is either the string "UIMA" for JP logs, or "JD" for JD logs.
    \item[NODE] This is the name of the machine where the process runs.
    \item[PROCESS] This is the process id of the process on the indicated node.
\end{description}

This shows the contents a sample log directory for a small job that consisted of two processes.

\begin{verbatim}
         100-JD-bluej290-1-29383.log 
         100-uima-ae-descriptor-29383.xml 
         100-uima-as-dd-29383.xml 
         100-UIMA-bluej290-2-32766.log 
         100-UIMA-bluej291-63-13594.log 
         jd.out.log 
         job-performance-summary.ser 
         job-specification.properties 
\end{verbatim}

In this example, 

\begin{itemize}
     \item[] The file 100-JD-bluej290-1-29383.log is the diagnostic JD log, where the JD executed on node
       bluej290-1 in process 29383.

     \item[] The file 100-uima-ae-descriptor-29383.xml is the UIMA pipeline descriptor describing the
       service process that is launched in each JP, where the JD process is 29383.

     \item[] The file 100-uima-as-dd-29383.xml is the UIMA-AS service descriptor where the client is
       the JD process running in process 29383.

     \item[] The file 100-UIMA-bluej290-2-32766.log is a JP log for job 100, that ran on node
       bluej290-2, in process 32766.

     \item[] The file 100-UIMA-bluej291-63-13594.log is a JP log for job 100, that ran on node
       bluej291-63, in process 13594

     \item[] The file jd.out.log is the user's JD log, containing the user's collection reader output.

     \item[] The file job-performance-summary.ser is the serialized performance breakdown that is
       displayed in the Web Server The file job-specification.propeties is the properties file
       describing the job.
\end{itemize}
     
