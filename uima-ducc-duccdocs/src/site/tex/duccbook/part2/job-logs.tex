% Create well-known link to this spot for HTML version
\ifpdf
\else
\HCode{<a name='DUCC_LOGS'></a>}
\fi
\chapter{Job Logs}
\label{chap:job-logs}

The DUCC logs are managed by Apache log4j.  The DUCC log4j configuration file is
\duccruntime/log4j.xml. It is not in the scope of this document to describe log4j or its
configuration mechanism. Details on log4j can be found at \url{http://logging.apache.org/log4j}.

The "user logs" are the Job Driver (JD) and Job Process (JP) logs. There is one log for each process 
of a job. The JD log is divided between two physical files: 

\begin{enumerate}
   \item Stdout and default UIMA logging output written by the UIMA collection reader. 
     
   \item The diagnostic logs written the the DUCC JD wrapper around the job's collection reader. 
\end{enumerate}

A number of other useful files are written to the log directory: 
\begin{enumerate}

  \item A properties file containing the full job specification for the job. This includes all the 
    parameters specified by the user as well as the default parameters. This file is called
    {\tt job-specification.properties.}


  \item The UIMA pipeline descriptor constructed by DUCC that describes the process that is 
    dispatched to each Job Process (JP). The name of this file is of the form 

\begin{verbatim}
         JOBID-uima-ae-descriptor-PROCESS.xml 
\end{verbatim}

    where 

    \begin{description}
        \item[JOBID] This is the numerical id of the job as assigned by DUCC.
        \item[PROCESS] This is the process id of the Job Driver (JD) process.
        \end{description}
      
      \item The UIMA-AS service descriptor that defines the process that defines the job as as UIMAAS 
        service. The name of this file is of the form 
\begin{verbatim}
         JOBID-uima-as-dd-PROCESS.xml 
\end{verbatim}
    
        where 
        \begin{description}
            \item[JOBID] This is the numerical id of the job as assigned by DUCC.
            \item[PROCESS] This is the process id of the Job Driver (JD) process.
        \end{description}

      \item A colllection of gzipped ``json'' files containing the performance breakdown of the job.
        These can be read and formatted using \hyperref[sec:cli.ducc-perf-stats]{ducc\_perf\_stats.}
 \end{enumerate}

The Job Process logs are written to the configured log directory and are of the following form:
\begin{verbatim}
         JOBID-TYPE-NODE-PROCESS.log 
\end{verbatim}
where 
\begin{description}
    \item[JOBID] This is the numerical id of the job as assigned by DUCC.
    \item[TYPE] This is either the string "UIMA" for JP logs, or "JD" for JD logs.
    \item[NODE] This is the name of the machine where the process ran.
    \item[PROCESS] This is the Unix process id of the process on the indicated node.
\end{description}

This shows the contents a sample log directory for a small job that consisted of two processes.

\begin{verbatim}
    100-JD-bluej290-1-29383.log 
    100-uima-ae-descriptor-29383.xml 
    100-uima-as-dd-29383.xml 
    100-UIMA-bluej290-2-32766.log 
    100-UIMA-bluej291-63-13594.log 
    jd.out.log 
    job-performance-summary.ser 
    job-specification.properties 
\end{verbatim}

In this example, 

\begin{itemize}
     \item[] The file {\tt 100-JD-bluej290-1-29383.log} is the diagnostic JD log, where the JD executed on node
       bluej290-1 in process 29383.

     \item[] The file {\tt 100-uima-ae-descriptor-29383.xml} is the UIMA pipeline descriptor describing the
       service process that is launched in each JP, where the JD process is 29383.

     \item[] The file {\tt 100-uima-as-dd-29383.xml} is the UIMA-AS service descriptor where the client is
       the JD process running in process 29383.

     \item[] The file {\tt 100-UIMA-bluej290-2-32766.log} is a JP log for job 100, that ran on node
       bluej290-2, in process 32766.

     \item[] The file {\tt 100-UIMA-bluej291-63-13594.log} is a JP log for job 100, that ran on node
       bluej291-63, in process 13594

     \item[] The file {\tt jd.out.log} is the user's JD log, containing the user's collection reader output.

       \item[] These files are the performance summary files:
\begin{verbatim}
job-performance-summary.json.gz
job-processes-data.json.gz
job-specification.properties
work-item-status.json.gz
\end{verbatim}

\end{itemize}
     
