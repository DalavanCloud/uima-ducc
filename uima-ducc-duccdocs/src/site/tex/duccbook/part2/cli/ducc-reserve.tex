% Create well-known link to this spot for HTML version
\ifpdf
\else
\HCode{<a name='DUCC_CLI_RESERVE'></a>}
\fi
    \section{ducc\_reserve}

    \paragraph{Description:}
    The reserve CLI is used request a reservation of resources. Reservations can be for entire 
    machines or partial machines, based on memory requirements. All reservations are persistent: 
    the resources remain dedicated to the requestor until explicitly returned. All reservations are 
    performeed on an "all-or-nothing" basis: either the entire set of requested resources is reserved, 
    or the reservation request fails. 

    \paragraph{Usage:}
        \begin{description}
        \item[Executable Jar] java -jar \ducchome/lib/uima-ducc-reserve.jar {\em options}
        \item[Script wrapper] java -jar \ducchome/bin/ducc-reserve {\em options}
        \item[Java Main]      java -cp \ducchome/lib/uima-ducc-reserve.jar org.apache.uima.ducc.cli.DuccReservationSubmit {\em options}
        \end{description}

    \paragraph{Options:}
    
        \begin{description}

            \item[--debug ]          
              Prints internal debugging information, intended for DUCC developers or extended problem determination.
              
            \item[--description {[text]}]               
              The text is any string used to describe the reservation. It is displayed in the Web Server. 
              
            \item[--help ]             
              Prints the usage text to the console. 
              
            \item[--debug ]              
              Prints internal debugging information, intended for DUCC developers or extended problem determination.
              
            \item[--number-of-instances {[integer]}]               
              This specifies the number of full or partial machine reservations to schedule. 
              
            \item[--instance-memory-size {[KB|MB|GB|TB]}]]               
              This specifies the amount of memory the reserved machine must supoprt. For full machine 
              reservations, this is the total memory on the machine. For partial reservations, the machine 
              may have more memory, but not less than is specified. 
              
            \item[--scheduling\_class {[classname]}]               
              This specifies the name of the scheuling class the RM will use to determine the resource 
              allocation for each process. The default DUCC distribution provides class "reserve" for full 
              machine reservations, and "fixed" for partial machine reservations. 
              
            \item[--specifiecaiton {[file]}]               
              All the parameters used to request a reservation may be placed in a standard Java 
              properties file. This file may then be used to submit the request (rather than providing all 
              the parameters directory to submit). 

        \end{description}
            
    \paragraph{Notes:}
    Reservations may be for full machines, or partial machines based on memory. The mechanism 
    for distinguishing which type of reservation the job class. A job class implementing the 
    RESERVE scheduling policy results in a full machine being reserved. A job clas simplementing 
    the FIXED scheduling policy results in a partial machine being reserved. The default DUCC 
    distribution configures class reserve for full machine reservations, and class fixed for partial 
    reservations. 


