% Create well-known link to this spot for HTML version
\ifpdf
\else
\HCode{<a name='DUCC_CLI_PRLCESS_CANCEL'></a>}
\fi
    \section{ducc\_process\_cancel}

    \paragraph{Description:}

    \paragraph{Usage:}
    \begin{description}
    \item[Executable Jar] java -jar \ducchome/lib/uima-ducc-process-cancel.jar {\em options}
    \item[Script wrapper] java -jar \ducchome/bin/ducc-process-cancel {\em options}
    \item[Java Main]      java -cp \ducchome/lib/uima-ducc-process-cancel.jar org.apache.uima.ducc.cli.DuccManagedReservationCancel {\em options}
    \end{description}

    \paragraph{Options:}
    \begin{description}
        \item[--debug ]          
          Prints internal debugging information, intended for DUCC developers or extended problem determination.          
        \item[--id {[jobid]}]
          The DUCC ID is the id of the process to cancel.
        \item[--help]
          Prints the usage text to the console.
        \item[--reason]
          Optional. This specifies the reason the process is canceled, for display in the web server. 
        \item[--role\_administrator] The command is being issued in the role of a DUCC administrator.
          If the user is not also a registered administrator this flag is ignored.  (This helps to
          protect administrators from inadvertantly canceling work they do not own.)
     \end{description}
        
    \paragraph{Notes:}
    None.

