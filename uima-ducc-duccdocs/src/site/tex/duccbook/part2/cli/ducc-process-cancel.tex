% 
% Licensed to the Apache Software Foundation (ASF) under one
% or more contributor license agreements.  See the NOTICE file
% distributed with this work for additional information
% regarding copyright ownership.  The ASF licenses this file
% to you under the Apache License, Version 2.0 (the
% "License"); you may not use this file except in compliance
% with the License.  You may obtain a copy of the License at
% 
%   http://www.apache.org/licenses/LICENSE-2.0
% 
% Unless required by applicable law or agreed to in writing,
% software distributed under the License is distributed on an
% "AS IS" BASIS, WITHOUT WARRANTIES OR CONDITIONS OF ANY
% KIND, either express or implied.  See the License for the
% specific language governing permissions and limitations
% under the License.
% 
% Create well-known link to this spot for HTML version
\ifpdf
\else
\HCode{<a name='DUCC_CLI_PROCESS_CANCEL'></a>}
\fi
    \section{ducc\_process\_cancel}

    \paragraph{Description:}
    The cancel CLI is used to cancel a process that has previously been submitted but which has not yet 
    completed. 

    \paragraph{Usage:}
    \begin{description}
    \item[Script wrapper] \ducchome/bin/ducc\_process\_cancel {\em options}
    \item[Java Main]      java -cp \ducchome/lib/uima-ducc-cli.jar\\org.apache.uima.ducc.cli.DuccManagedReservationCancel {\em options}
    \end{description}

    \paragraph{Options:}
    \begin{description}
        \item[$--$debug ]          
          Prints internal debugging information, intended for DUCC developers or extended problem determination.          
        \item[$--$id {[jobid]}]
          The DUCC ID is the id of the process to cancel.
        \item[$--$help]
          Prints the usage text to the console.
        \item[$--$reason]
          Optional. This specifies the reason the process is canceled, for display in the web server. 
        \item[$--$role\_administrator] The command is being issued in the role of a DUCC administrator.
          If the user is not also a registered administrator this flag is ignored.  (This helps to
          protect administrators from inadvertently canceling work they do not own.)
     \end{description}
        
    \paragraph{Notes:}
    None.

