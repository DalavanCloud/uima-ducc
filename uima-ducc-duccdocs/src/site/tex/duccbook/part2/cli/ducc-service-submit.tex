    \section{ducc\_service\_submit}

    \paragraph{Description:}
    The ducc\_service\_submit CLI is used to submit a job as a service to DUCC. The CLI is similar to
    ducc\_submit with the following key differences:
    
    \begin{itemize}
        \item There is no Collection Reader. 

        \item There is no Job monitor for services because services don't generally end of their own
          accord.
        \end{itemize}
        
        Service jobs must supply a fully-formed DD XML descriptor.  On submission of a service, the
        DUCC CLI examines the service DD descriptor for the queue name, and the supplied jvm\_args for
        a broker URL. It forms a service ID of the following form which may be referenced in the
        --service\_dependency clauses of jobs and services which are dependent on this service:
\begin{verbatim}
UIMA-AS:[queue-name]:[broker-url] 
\end{verbatim}

    The {\em queue-name} is the ActiveMQ queue name used by the service.

    The {\em ducc\_service\_submit} and {\em ducc\_service\_cancel} commands are primarily used by
    the DUCC Service Manager for starting and stopping instances of registered services.  It is legal
    and supported for these to be used outside of the context of registered services, however.  DUCC
    recognizes the processes as service processes and manages dependencies accordinly. 

    \paragraph{Usage:}
    \begin{description}
    \item[Executable Jar] java -jar \ducchome/lib/uima-ducc-sesrvice-submit.jar {\em options}
    \item[Script wrapper] java -jar \ducchome/bin/ducc-service-submit {\em options}
    \item[Java Main]      java -cp \ducchome/lib/uima-ducc-service-submit.jar org.apache.uima.ducc.cli.DuccServiceSubmit {\em options}
    \end{description}

    \paragraph{Options:}
    \begin{description}

        \item[--classpath] The classpath used for the service.
          
        \item[--classpath\_order {[UserBeforeDucc | DuccBeforeUser]} ]
          When DUCC deploys a process, set the user-supplied classpath before DUCC-supplied
          classpath, or the reverse.
          
        \item[--debug ]
          Enable debugging messages. This is primarily for debugging DUCC itself. 
          
        \item[--description {[text]}] The text is any string used to describe the job. It is displayed
          in the Web Server.
          
        \item[--environment {[env vars]}] Blank-delimeted list of environment variables.  Example:
\begin{verbatim}
"TERM=xterm DISPLAY=me.org.net:1.0". 
\end{verbatim}

        \item[--help ]
        This prints the usage text to the console. 

      \item[--jvm {[path-to-java]}] This specifies the JVM to use. If not specified, the same JVM
        used by the Agents is used.

      \item[--jvm\_args {[list]} ]        
        This specifes extra JVM arguments to be provided to the server process. It is a blank delimeted 
        list of strings. Example: 
\begin{verbatim}
--jvm_args -Xmx100M -Xms50M 
\end{verbatim}

      \item[--log\_directory {[path-to-log directory]}] 
        This specifies the path to the directory for the user logs. If not specified, the default is the 
        user's home directory. Example: 

        Within this directory DUCC creates a subdirectory for each job, using the numerical 
        ID of the job. The format of the generated log file names is described ELSEWHERE.NEED.REF.

        Note: Note that --log\_directory specifies only the path to a directory where 
        logs are to be stored. In order to manage multiple processes running in multiple 
        machines DUCC, sub-directory and file names are generated by DUCC and may 
        not be directly controlled by the user.

      \item[--process\_classpath {[ClASSPATH]}] 
        This specifies the Java CLASSPATH to use in each Job Process (JP) and must be 
        specified.

      \item[--process\_DD {[DD descriptor]}] 
        This specifies the UIMA Deployment Descriptor for the service.

      \item[--process\_environment {[environment]}] This specifies environment parameters for the Job
        Processes. If present, they are added to the Job Process environment as the process is
        spawned. It must be a quoted, blankdelimeted lsit of name-value pairs.

        Note: On Secure Linux systems, the environemnt variable 
        LD\_LIBRARY\_PATH may not be passed to the user's program. If it is 
        necessary to pass LD\_LIBRARY\_PATH to the JP or JD processes, it must be 
        specified as DUCC\_LD\_LIBRARY\_PATH. Ducc (securely) passes this as 
        LD\_LIBRARY\_PATH, after the JP or JD has assumed the user's identity. For 
        example: 
\begin{verbatim}
"--process_environment TERM=xterm DISPLAY=:1.0 DUCC_LD_LIBRARY_PATH=/my/own/lib.so" 
\end{verbatim}
        
      \item[--process\_failures\_limit {[integer]}] 
        This specifies the maximum number of individual Job Process (JP) failures that are to be 
        tolerated before killing the job. The default is 15. If this limit is exceeded over the lifetime 
        of a job DUCC terminates the entire job. 

      \item[--process\_initialization\_failures\_cap {[integer]}] 
        This specifies the maximum number of independent Job Process initialization failures (i.e. 
        System.exit(), kill-15...) before the number of Job Processes is capped at the number in 
        state Running currently. The default is 99.

        Note that the job is NOT killed if there are processes that have passed initialization and are 
        running. If this limit is reached, the only action is to not start new processes for the job. 

      \item[--process\_jvm\_args {[list]}] 
        This specifies additinal arguments to be passed to the Job Process JVM. Example: 
\begin{verbatim}
--process_jvm_args -Xmx400M -Xms100M 
\end{verbatim}
        
      \item[--process\_memory\_size {[size]}] This specifies the maximum amount of RAM in GB to be
        allocated to each Job Process.  This value is used by the Resource Manager to allocate
        resources. if this amount is exceeded by a Job Process the Agent terminates the process with
        a ShareSizeExceeded message.

      \item[--scheduling\_class {[classname]}] This specifies the name of the scheuling class the RM
        will use to determine the resource allocation for each process. The names of the classes are
        installation dependent. If not specified, the default is taken from the global DUCC
        configuration ducc.properties. 

      \item[--service\_dependency{[list]}] This specifies a comma-delimeted list of services the job
        processes are dependent upon.  Each endpoint must be of the form UIMA-AS:endpoint:broker\_url
        where endpoint is the UIMA-AS service endpoint and broker\_url is the ActiveMQ broker URL.

        In the example are two dependencies, one with endpoint RandomSleepAE and broker
        tcp:bluej682:61616, and the other with endpoint OtherEp and broker URL
        tcp:bluej123:123. Example:
\begin{verbatim}
--service_dependency UIMA-AS:RandomSleepAE:tcp:bluej682:61616 UIMA-AS:OtherEp:tcp:bluej123:123 
\end{verbatim}

      \item[--service\_linger {[seconds]}] This is the time in milliseconds to wait after last
        referring job or service exits before stopping a non-autostarted service.

      \item[--service\_ping\_class {[classname]}] This is the class used to ping a service, if the
        default DUCC-supplied class is not used.  It is always required for CUSTOM services, and
        may be specified to override the default for UIMA-AS services.  It must extend
        org.apache.uima.ducc.common.AServicePing.

      \item[--service\_ping\_classpath {[classpath]}] If {\em service\_ping\_class} is specified,
        this is the classpath containing service\_custom\_ping class and dependencies.  If not
        specified, the Agent's classpath is used (which usually is not correct!).

      \item[--service\_ping\_dolog {[boolean]}] If specified, write pinger stdout and stderr
        messages to a log, else suppress the log.

      \item[--service\_ping\_jvm\_args {[java-system-property-assignments]}] -D jvm system property
        assignments to pass to jvm when running the pinger.

      \item[--service\_ping\_timeout {[time-in-ms]}] This is the time in milliseconds to wait for a ping to the
        service. 

      \item[--service\_request\_endpoint {[string]}] This specifies the expected service id.  The string
        must be in form {\tt queue:broker-url}.  This must match the service ID that is derived from
        the {\em --process\_DD}; if it does not match, the submit fails.  This is intended for use
        by applications using the service API and CLI as a fail-fast if something is wrong with the
        application.

      \item[--specifiecaiton {[file]}] All the parameters used to submit a job may be placed in a
        standard Java properties file.  This file may then be used to submit the job (rather than
        providing all the parameters directory to submit).
        For example, 

\begin{verbatim}
ducc_submit --specification job.props 
\end{verbatim}
        
        where the job.props contains: 

\begin{verbatim}
working_directory=/Users/challngr/projects/ducc/ducc_test/test/bin 
process_failures_limit=20 
driver_environment= DUCC_LD_LIBRARY_PATH=/a/other/bogus/path 
process_environment=AE_INIT_TIME=10000 DUCC_LD_LIBRARY_PATH=/a/bogus/path 
log_directory=/Users/challngr/ducc/logs/ 
process_initialization_failures_cap=99 
process_descriptor_AE=org.apache.uima.ducc.test.randomsleep.FixedSleepAE 
process_classpath=/home/bob/projects/ducky-service.jar 
description=../simple/jobs/1.job[AE] 
process_jvm_args=-Xmx100M -DdefaultBrokerURL=tcp://localhost:61616 
scheduling_class=fixed 
process_memory_size=15 
\end{verbatim}
        
        \item[--working\_directory {[directory-name]}]
          This specifies the working directory to be set by the Job Driver and Job Process processes. 
          If not specified, the current directory is used.
    \end{description}
        
    \paragraph{Notes:}
    When searching for UIMA XML resource files such as descriptors, DUCC searches both the 
    classpath and the data path according to the following rules: 

    \begin{enumerate}
        \item If the resource ends in .xml it is assumed the resource is a file and the path is either
          an absolute path or a path relative to the specified working directory. If the file is not
          found the search exits and the job is terminated.

        \item If the resource does not end in .xml, DUCC creates a path by replacing the "." 
          separators with "/" and appending ".xml". It then searches the CLASSPATH for the 
          resource as a file. 
    \end{enumerate}

    If the resource is found in either place the search is successful. Otherwise the search 
    fails and the job is terminated. 

    Note: Note that in the current implementation, resources are NOT searched    
    for inside jars in the classpath. Files must be supplied. 

