% Create well-known link to this spot for HTML version
\ifpdf
\else
\HCode{<a name='DUCC_CLI_SERVICE_SUBMIT'></a>}
\fi
    \section{ducc\_service\_submit}
    \label{sec:cli.service-submit}
    \paragraph{Description:}
    The ducc\_service\_submit CLI is used to submit work as a service to DUCC. The CLI is similar to
    ducc\_submit with the following key differences:
    
    \begin{itemize}
        \item There is no Collection Reader. 
        \item There is no Job Monitor  for services.
    \end{itemize}
        
    The {\em ducc\_service\_submit} and \hyperref[sec:cli.service-cancel]{{\em
        ducc\_service\_cancel}} commands are primarily used by the DUCC Service Manager for starting
    and stopping instances of registered services but they are supported for general use as well.

    DUCC will attempt to restart failed services instances up to a configurable limit.  DUCC will
    NOT restart submitted services aftar a system failure however.  If full restart after system
    failure is required, one must \hyperref[subsec:cli.ducc-services.register]{register} the
    service.
 

    \paragraph{Usage:}
    \begin{description}
    \item[Executable Jar] java -jar \ducchome/lib/uima-ducc-sesrvice-submit.jar {\em options}
    \item[Script wrapper] \ducchome/bin/ducc\_service\_submit {\em options}
    \item[Java Main]      java -cp \ducchome/lib/uima-ducc-service-submit.jar org.apache.uima.ducc.cli.DuccServiceSubmit {\em options}
    \end{description}

    \paragraph{Options:}
    \begin{description}

        \item[$--$classpath] The CLASSPATH used for the service, if the service is a
          \hyperref[sec:services.types]{UIMA-AS services}.  If not specified, the CLASSPATH used
          by the {\em ducc\_service\_submit} command is used.
          
        \Item[$--$classpath\_order {[UserBeforeDucc | DuccBeforeUser]} ] When DUCC deploys a process,
          set the user-supplied classpath before DUCC-supplied classpath, or the reverse.  This is
          only valid for  \hyperref[sec:services.types]{UIMA-AS services}.
          
        \item[$--$debug ]
          Enable debugging messages. This is primarily for debugging DUCC itself. 
          
        \item[$--$description {[text]}] The text is any quoted string used to describe the job. It is
          displayed in the Web Server.

          Note: When used as a CLI option, the description string must usually be quoted to protect
          it from the shell.
    
        \item[$--$environment {[env vars]}]  
          This specifies environment parameters for the Service. If present, they are added 
          to the Job Process environment as the process is spawned. It is a blank-delimited 
          list of name-value pairs. For example: 
\begin{verbatim}

--environment 'TERM=xterm DISPLAY=:1.0'
\end{verbatim}
          
          Note: On Secure Linux systems, the environemnt variable 
          LD\_LIBRARY\_PATH may not be passed to the user's program. If it is 
          necessary to pass LD\_LIBRARY\_PATH to the JP or JD processes, it must be 
          specified as DUCC\_LD\_LIBRARY\_PATH. Ducc (securely) passes this as 
          LD\_LIBRARY\_PATH, after the JP or JD has assumed the user's identity. For 
          example: 
\begin{verbatim}
--environment TERM=xterm DISPLAY=:1.0 DUCC\_LD\_LIBRARY\_PATH=/my/own/path
\end{verbatim}
          
             Note: When used as a CLI option, the environment string must usually be
             quoted to protect it from the shell.
          
        \item[$--$help ] This prints the usage text to the console.

        \item[$--$jvm {[path-to-java]}] This specifies the JVM to use for 
          \hyperref[sec:services.types]{UIMA-AS services}. If not
          specified, the same JVM used by the Agents is used.  

          Note: The path must be the full path the the Java executable (not 
          simply the JAVA_HOME environment variable.).  Example:
\begin{verbatim}
--jvm /share/jdk1.6/bin/java 
\end{verbatim}


        \item[$--$jvm\_args {[list]} ]        
          This specifes extra JVM arguments to be provided to the server process for
          \hyperref[sec:services.types]{UIMA-AS services}. It is a blank delimeted 
            list of strings. Example: 
\begin{verbatim}
--jvm_args '-Xmx100M -Xms50M' 
\end{verbatim}

          Note: When used as a CLI option, the argument string must usually be quoted to protect
          it from the shell.
    
          \item[$--$log\_directory {[path-to-log directory]}] This specifies the path to the directory for
            the individual service instance logs. If not specified, the default is the user's home
            directory. Example:
\begin{verbatim}
--log\_directory /home/bob 
\end{verbatim}
        
        Within this directory DUCC creates a subdirectory for each job, using the numerical 
        ID of the job. The format of the generated log file names as described
        \hyperref[chap:job-logs]{here}.
        
        Note: Note that $--$log\_directory specifies only the path to a directory where 
        logs are to be stored. In order to manage multiple processes running in multiple 
        machines DUCC, sub-directory and file names are generated by DUCC and may 
        not be directly specified. 

      \item[$--$process\_DD {[DD descriptor]}] 
        This specifies the UIMA Deployment Descriptor for \hyperref[sec:services.types]{UIMA-AS services}.

      \item[$--$process\_executable {[program-name]}] For \hyperref[sec:services.types]{CUSTOM
          services}, this specifies the full path of the program to execute.

      \item[$--$process\_executable\_args {[list-of-arguments]}] For \hyperref[sec:services.types]{CUSTOM
          services}, this specifies the program arguments, if any.

      \item[$--$process\_failures\_limit {[integer]}] 
        This specifies the maximum number of consecutive individual service instance failures that are to be 
        tolerated before stopping the service. The default is five (5). If the instance is successfully
        restarted, the count is reset to zero (0), so that the occasional process failure does not cause
        the entire service to be terminated.
        
      \item[$--$process\_memory\_size {[size]}] This specifies the maximum amount of RAM in GB to be
        allocated to each Job Process.  This value is used by the Resource Manager to allocate
        resources. 

      \item[$--$scheduling\_class {[classname]}] This specifies the name of the scheuling class the RM
        will use to determine the resource allocation for each process. The names of the classes are
        installation dependent. If not specified, the default is taken from the global DUCC
        configuration \hyperref[sec:ducc.properties]{ducc.properties.}

      \item[$--$service\_dependency{[list]}] This specifies a comma-delimeted list of services the job
        processes are dependent upon. Service dependencies are discussed in detail
        \hyperref[sec:service.endpoints]{here}. Example:
\begin{verbatim}
--service_dependency UIMA-AS:RandomSleepAE:tcp:bluej682:61616 UIMA-AS:OtherEp:tcp:bluej123:123 
\end{verbatim}

        Note: When used as a CLI option, the list must usually be
        quoted to protect it from the shell.
          

      \item[$--$service\_linger {[seconds]}] This is the time in milliseconds to wait after the last
        referring job or service exits before stopping a non-autostarted service.

      \item[$--$service\_ping\_class {[classname]}] This is the Java class used to ping a service. 

        This parameter is required for CUSTOM services.

        This parameter may be specified for UIMA-AS services; however, DUCC supplies a default
        pinger for UIMA-AS services.

      \item[$--$service\_ping\_classpath {[classpath]}] If {\em service\_ping\_class} is specified,
        this is the classpath containing service\_custom\_ping class and dependencies.  If not
        specified, the Agent's classpath is used (which will generally be incorrect.)

      \item[$--$service\_ping\_dolog {[boolean]}] If specified, write pinger stdout and stderr
        messages to a log, else suppress the log. See \hyperref[sec:service.pingers]{Service Pingers}
        for details.

      \item[$--$service\_ping\_jvm\_args {[java-system-property-assignments]}] If 
        {\em service\_ping\_class} is specified, these are the arguments 
        to pass to jvm when running the pinger. The arguments are specified as a blank-delimited
        list of string.  Example:
\begin{verbatim}
--process_jvm_args '-Xmx400M -Xms100M' 
\end{verbatim}
        
        Note: When used as a CLI option, the arguments must usually be
        quoted to protect them from the shell.

      \item[$--$service\_ping\_timeout {[time-in-ms]}] This is the time in milliseconds to wait for a
        ping to the service.  If the timer expires without a response the ping is ``failed''. After
        a certain number of consecutive failed pings, the service is considered ``down.''  See
        \hyperref[sec:service.pingers]{Service Pingers} for more details.

      \item[$--$service\_request\_endpoint {[string]}] This specifies the expected service id.  

        This string is optional for UIMA-AS services; if specified, however, it must be of the
        form {\tt UIMA-AS:queue:broker-url}, and both the queue and broker must match those specified in the
        service DD specifier.

        If the service is CUSTOM, the endpoint is required, and must be of the form
        {\tt CUSTOM:string} where the contents of the string are determined by the service.

      \item[$--$specifiecaiton, $-$f {[file]}] All the parameters used to submit a service may be placed in a
        standard Java properties file.  This file may then be used to submit the service (rather than
        providing all the parameters directory to submit).
        For example, 

\begin{verbatim}
ducc_service_submit --specification svc.props 
ducc_service_submit -f svc.props 
\end{verbatim}
        
        where the svc.props contains: 

\begin{verbatim}
process_environment = AE_INIT_TIME=5000 AE_INIT_RANGE=1000 INIT_ERROR=0
description         = Test Service 3
process_jvm_args    = -DdefaultBrokerURL=tcp://agent86:61616
process_classpath   = /home/bob/lib/app.jar
process_memory_size = 15
working_directory   = /home/bob/services
process_DD          = /home/bob/services/service.xml
scheduling_class    = fixed
service_dependency  =  UIMA-AS:FixedSleepAE_4:tcp://agent86:61616
\end{verbatim}
        
        \item[$--$working\_directory {[directory-name]}]
          This specifies the working directory to be set by the Job Driver and Job Process processes. 
          If not specified, the current directory is used.
    \end{description}
        
    \paragraph{Notes:}
    When searching for UIMA XML resource files such as descriptors, DUCC searches both the 
    classpath and the data path according to the following rules: 

    \begin{enumerate}
        \item If the resource ends in .xml it is assumed the resource is a file and the path is either
          an absolute path or a path relative to the specified working directory. If the file is not
          found the search exits and the job is terminated.

        \item If the resource does not end in .xml, DUCC creates a path by replacing the "." 
          separators with "/" and appending ".xml". It then searches the CLASSPATH for the 
          resource as a file. 
    \end{enumerate}

    If the resource is found in either place the search is successful. Otherwise the search 
    fails and the job is terminated. 

    Note: Note that in the current implementation, resources are NOT searched    
    for inside jars in the classpath. Files must be supplied. 

