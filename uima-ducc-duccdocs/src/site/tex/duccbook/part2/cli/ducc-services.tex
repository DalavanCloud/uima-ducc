% Create well-known link to this spot for HTML version
\ifpdf
\else
\HCode{<a name='DUCC_CLI_SERVICES'></a>}
\fi
    \section{ducc\_services}
    \label{sec:cli.ducc-services}

    \paragraph{Description:}

        The ducc\_services CLI is used to manage service registration. It has a number of functions 
        as listed below.
        
        The functions include: 
        \begin{description}
            \item[Register] This registers a service with the Service Manager by saving a service
              specification in the Service Manager's registration area. The specification is
              retained by DUCC until it is unregistered.

              The registration consists primarily of a service specification identical to that used
              with \hyperref[sec:cli.service-submit]{ducc\_submit\_service}.  This specification is
              used when the Service Manager needs to start a service instance.  A second properties
              file, the {\em meta properties} for the service, contains additional state and
              management properties.  The registered properties for a service are made available for
              viewing from the DUCC Web Server's \hyperref[sec:ws-service-details]{service details}
              page.
              
            \item[Unregister] This unregisters a service with the Service Manager. When a service is
              unregistered DUCC optionally stops the service instance, if any, and discard the
              saved specification.
              
            \item[Start] The start function instructs DUCC to alllocate resources for a service and to
              start it in those resources. The service remains running until explictly stopped. DUCC
              will attempt to keep the service instances running if they should fail. The start function
              is also used to increase the number of running service instances if desired.
              
            \item[Stop] The stop function stops some or all service instances.
              
            \item[Query] The query function returns detailed information about all known services, both
              registerd and otherwise.
              
            \item[Modify] The modify function allows some aspectes of a registered service to be updated
              without reregistereing the service. It optionally alters the running service instances to
              conform with the updates.    
        \end{description}
            

    \paragraph{Usage:}
       \begin{description}
          \item[Executable Jar] java -jar \ducchome/lib/uima-ducc-services.jar {\em options}
          \item[Script wrapper] \ducchome/bin/ducc\_services {\em options}
          \item[Java Main]      java -cp \ducchome/lib/uima-ducc-services.jar org.apache.uima.ducc.cli.DuccServiceApi {\em options}
          \end{description}
          
          The ducc\_services CLI requires one of the verbs listed above as the first argument. Other arguments are determined
          by the verb.

    \paragraph{Options:}

    \subsection{Common Options}
        These options are common to all of the service verbs:
        \begin{description}
           \item[$--$debug ]          
             Prints internal debugging information, intended for DUCC developers or extended problem determination.                    
           \item[$--$help]
             Prints the usage text to the console. 
        \end{description}
        
    \subsection{ducc\_services $--$register Options}
    \label{subsec:cli.ducc-services.register}
       The {\em register} function submits a service specification to DUCC.  DUCC stores this 
       information until it is {\em unregistered}.  Once registered, a service may be
       started, stopped, etc.

       \begin{description}
           \item[$--$register {[specification file] [options]}] The specification file is optional.  If
             specified, it has the same contents as described for the \hyperref[sec:cli.service-submit]{{\em
                 ducc\_service\_submit}} command.  As with {\em ducc\_service\_submit}, any of the
             keywords in the specification may be overriden on the command line.

           \item[$--$autostart {[true or false]}] This indicates whether to register the service as
             an autostarted service.  If not specified, the default is {\em false}.
             
           \item[$--$instances {[n]}] This specifies the number of instances to start when the service
             is started.  If not specified, the default is 1.
             
           \item[Other keywords] These are the same as described for \hyperref[sec:cli.service-submit]{{\em
                 ducc\_service\_submit}}
       \end{description}


    \subsection{ducc\_services $--$start Options}

    The start function instructs DUCC to alllocate resources for a service and to start it in those
    resources. The service remains running until explictly stopped. DUCC will attempt to keep the
    service instances running if they should fail. The start function is also used to increase the
    number of running service instances if desired.
    
       \begin{description}
       \item[$--$start {[service-id or endpoint]}] This indicates that a service is to be started. The service id
         is either the numeric ID assigned by DUCC when the service is registered, or the service
         endpoint string.  Example:
\begin{verbatim}
ducc_services --start 23 
ducc_services --start UIMA-AS:Service23:tcp://bob.com:12345 
\end{verbatim}
         
       \item[$--$instances {[integer]}] This is the number of instances to start. If omitted, sufficient
         instances to match the registered number are started. If more than the registered number of
         instances is running this command has no effect.

         If the number of instances is specified, the number is added
         to the currently number of running instances. Thus if five instances are running and
\begin{verbatim}
         ducc_services --start 33 --instances 5
\end{verbatim}
         is issued, five more service instances are started for service 33 for a totoal of ten,
         regardless of the number specified in the registration. The registry is updated if the
         {\em --update} option is also specified. Examples:
\begin{verbatim}
ducc_services --start 23 --intances 5 
ducc_services --start UIMA-AS:Service23:tcp://bob.com:12345 --instances 3 --update 
\end{verbatim}

       \item[$--$update]If specified, the registry is updated to the total number of started
         instances.  Example:
\begin{verbatim}
ducc_services --start UIMA-AS:Service23:tcp://bob.com:12345 --instances 3 --update 
\end{verbatim}
       \end{description}

    \subsection{ducc\_services $--$stop Options}
    The stop function instructs DUCC to stop some number of service instances. If no specific number
    is specified, all instances are stopped. This is used only for registered services. Use
    \hyperref[sec:cli.service-cancel]{{\em ducc\_service\_cancel}} command to stop submitted services.

    \begin{description}

  \item[$--$stop {[service-id or endpoint]}] This specifies the service to be stopped. The service id
         is either the numeric ID assigned by DUCC when the service is registered, or the service
         endpoint string. Example:
\begin{verbatim}
ducc_services --stop 23 
ducc_services --stop UIMA-AS:Service23:tcp://bob.com:12345 
\end{verbatim}
         
       \item[$--$instances {[integer]}] This is the number of instances to stop. If omitted, all
         instances for the service are stopped.  If the number of instances is specified, then only
         the specified number of instances are stopped. Thus if ten instances are running for a
         service with numeric id 33 and
\begin{verbatim}
ducc_services --stop 33 --instances 5
\end{verbatim}
         is issued, five (randomly selected) service instances are stopped for
         service 33, leaving five running.  The registry is updated if the {\em --update} option is
         specified. The registered number of instances is never reduced to zero even if the number of
         running instances is reduced to zero.

         Example: 
\begin{verbatim}
ducc_services --stop 23 --intances 5 
ducc_services --stop UIMA-AS:Service23:tcp://bob.com:12345 --instances 3  
\end{verbatim}

       \item[$--$update] If specified, the registry is updated to the total number of instances
         remaining, but is never reduced below one (1). Example: 
\begin{verbatim}
ducc_services --stop UIMA-AS:Service23:tcp://bob.com:12345 --instances 3 --update
\end{verbatim}

    \end{description}

    \subsection{ducc\_services $--$modify Options}
    The modify function dynamically updates some of the attributes of a registered service. 
    
    \begin{description}
        \item[$--$modify {[service-id or endpoint]}]  This identifies the service to modify. The service id is either
          the numeric ID assigned by DUCC when the service is registered, or the service endpoint
          string.  Example:
\begin{verbatim}
ducc_services --modify 23 --instances 3 
ducc_services --modify UIMA-AS:Service23:tcp://bob.com:12345 --intances 2 
\end{verbatim}

        \item[ --instances {[integer]}] This updates the number of services instances that are
          started when the service is started.  Only the registration is updated. If the {\em $--$activate}
          option is also specified, running instances are stopped or started as needed to match the
          new number.

          Example: 
\begin{verbatim}
ducc_services --modify 23 --intances 5 
ducc_services --modify UIMA-AS:Service23:tcp://bob.com:12345 --instances 3 --activate 
\end{verbatim}

        \item[ --activate {[integer]}] When specified, the number of running service instances is
          increased or decreased to match the newly specified number.

          Example: 
\begin{verbatim}
ducc_services --modify 23 --intances 5 
ducc_services --modify UIMA-AS:Service23:tcp://bob.com:12345 --instances 3 --activate 
\end{verbatim}

        \item[ --autostart {["true" or "false"]}] This changes the autostart property for the
          registered services. When set to "true", the service is started automatically when the
          DUCC system is started.  If the service is not currently started, it will now start
          with the registered number of instances.  If the service is running, the instances
          remain running.

          One way to think of this is: if {\em autostart} is true, DUCC will attempt to keep
          the registered number of instances running at all times.  If {\em autostart} is
          false, all instance start and stop is manual.
          Example: 
\begin{verbatim}
ducc_services --stop UIMA-AS:Service23:tcp://bob.com:12345 --autostart false 
\end{verbatim}
        \end{description}

    \subsection{ducc\_services $--$query Options}
    The query function returns details about all known services of all types and classes, including 
    the DUCC ids of the service instances (for submitted and registered services), the DUCC ids of 
    the jobs using each service, and a summary of each service's queue and performance statistics, 
    when available. 
    
    All information returned by {\em ducc\_services $--$query} is also available via the
    \hyperref[ws:services-page]{Services Page} of the Web Server as well as the DUCC Service API (see 
    the \href{apidocs/index.html}{JavaDoc}).

    \begin{description}
    \item[$--$query {[service-id or endpoint]}] This indicates that a service is to be stopped. The
      service id is either the numeric ID assigned by DUCC when the service is registered, or the
      service endpoint string.

      If no id is given, information about all services is returned. 

      Below is a query against a system with three services. 

      The service with endpoint {\tt UIMA-AS:FixedSleepAE\_6:tcp://bobmach291:61617} is a service
      submitted outside of DUCC so it is marked as Implicit and has no implementing processes tha
      are known to DUCC. It is used by job 0 (``References'') and is active, available, and being
      actively pinged. The ActiveMq queue statistics are shown.

      The service with endpoint {\tt UIMA-AS:FixedSleepAE\_5:tcp://bobmach:61617} is a 
      registered service, whose registered numeric id is 2. It is registered for two instnaces and 
      no autostart. Since it is not autostarted, it will be terminated when it is no longer used. It 
      will linger for 5 seconds after the last referencing job completes, in case a subsequent job 
      that uses it enters the system (not a realistic linger time!). It has two active
      instances whose DUCC Ids are 9 and 5. It is currently used (referenced) 
      by DUCC jobs 1 and 5. 

      The service with endpoint {\tt UIMA-AS:FixedSleepAE\_1:tcp://bobmach:61617} is a 
      submitted service. It was submitted twice, and so has two implementors, DUCC service 
      jobs 0 and 1. It is referenced by job 7. It will continue to run until somebody cancels it, 
      even if it is not used. 

\begin{verbatim}
Service: UIMA-AS:FixedSleepAE_6:tcp://bobmach291:61617 
Service Class : Implicit 
Implementors : (N/A) 
References : 0 
Dependencies : none 
Service State : Available 
Ping Active : true 
Autostart : false 
Manual Stop : false 
Queue Statistics: 
Consum Prod Qsize minNQ maxNQ expCnt inFlgt  DQ  NQ Disp 
    78  240   170     2 36414      0      0 636 806  636 

Service: UIMA-AS:FixedSleepAE_5:tcp://bobmach291:61617 
Service Class : Registered as ID 2 instances[2] linger[5] 
Implementors : 9 8 
References : 1 5 
Dependencies : none 
Service State : Available 
Ping Active : true 
Autostart : false 
Manual Stop : false 
Queue Statistics: 
Consum Prod Qsize minNQ maxNQ expCnt inFlgt  DQ  NQ Disp 
    52   44     0     0     3      0      0 402 402  402 

Service: UIMA-AS:FixedSleepAE_1:tcp://bobmach291:61617 
Service Class : Submitted 
Implementors : 1 0 
References : 7 
Dependencies : none 
Service State : Available 
Ping Active : true 
Autostart : false 
Manual Stop : false 
Queue Statistics: 
Consum Prod Qsize minNQ   maxNQ expCnt inFlgt  DQ  NQ Disp 
    52    0     0     1 1504371      0      0  35  35   35 
\end{verbatim}
    \end{description}
    \paragraph{Notes:}

