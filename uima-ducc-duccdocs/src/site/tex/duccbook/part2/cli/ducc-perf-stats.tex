    \section{ducc\_perf\_stats}

    \paragraph{Description:}
    This CLI is used to format job history and performance data into CSV or (mostly) human readable
    form for post-analysis.  This may be run while a job is executing to monitor the current job, or
    after it exits.  This command produces the equivalent of the web servers job details page (see
    \prettyref{sec:ws-job-details}).

    \paragraph{Usage:}
    \begin{description}
    \item[Executable Jar] java -jar \ducchome/lib/uima-ducc-perf-stats.jar {\em options}
    \item[Script wrapper] java -jar \ducchome/bin/ducc-perf-stats {\em options}
    \item[Java Main]      java -cp \ducchome/lib/uima-ducc-perf-stats.jar org.apache.uima.ducc.cli.DuccPerfStats {\em options}
    \end{description}

    \paragraph{Options:}
    \begin{description}
        \item[--job id] This specifies the job to report on.
        \item[--directory dir] This specifies the job's log directory. (DUCC writes usage information into this
          directory.)
        \item[--report {[summary | workitems | provesses]}]
          This specifies the type of report:
          \begin{description}
              \item[summary] This produces a per-AE summary of the performance of that AE, including
                total time spent in the analytic, maximum time spent, minimum time, and total CASs
                processed.
              \item[workitms] This produces a performance break down of each each input CAS (work
                item), including the work item id, ending state, time spent in queue after dispatch,
                processing time, the node it executed on, and the process id it ran in.
              \item[processes] This produces a summary of all the processes which have executed on
                behalf of the job, including the node, processid, initialization time, current memory usage,
                maximum memory usage, page faults, swap space in use, maximum swap used, \%CPU,
                garbage collection statistics, and work item statistics (processed, errors, retried, etc.).
          \end{description}
        \item[--help] Prints the usage text to the console. 
     \end{description}
        
    \paragraph{Notes:}
    None.

