2y% Create well-known link to this spot for HTML version
\ifpdf
\else
\HCode{<a name='DUCC_CLI_PROCESS_SUBMIT'></a>}
\fi
    \section{ducc\_process\_submit}
    \label{sec:cli.ducc-process-submit}
    \paragraph{Description:}
       Use {\em ducc\_process\_submit} to submit a Managed Reservation, also known as an
       {\em arbitrary process} to DUCC.  The intention
       of this function is an alternative to utilities such as {\em ssh}, in order to allow the
       spawned processes to be fully managed by DUCC.  This allows the DUCC scheduler to allocate
       the necessary resources (and prevent over-allocation), and the DUCC run-time environment
       to manage process lifetime.

       If {\em process\_attach\_console} is specified, Stdin, Stderr, and Stdout of the remote
       process are redirected to the submitting console.  It is thus possible to run interactive
       sessions with remote processes where the resources are managed by DUCC.

    \paragraph{Usage:}
    \begin{description}
    \item[Executable Jar] java -jar \ducchome/lib/uima-ducc-process-submit.jar {\em options}
    \item[Script wrapper] \ducchome/bin/ducc\_process\_submit {\em options}
    \item[Java Main]      java -cp \ducchome/lib/uima-ducc-process-submit.jar org.apache.uima.ducc.cli.DuccManagedReservationCancel {\em options}
    \end{description}

    \paragraph{Options:}
    \begin{description}
    
        \item[--cancel\_on\_interrupt ] Cancel managed reservation on interrupt
          (Ctrl-C).  If running with {\em--wait\_for\_completion} and this flag is specified,
          terminating the submit process will result in the remote process being terminated.

        \item[--description {[text]}] The text is any string used to describe the process. It is
          displayed in the Web Server. When specified on a command-line the text usually must be
          surrounded by quotes to protect it from the shell.

        \item[--debug ] Prints internal debugging information, intended for DUCC developers or
          extended problem determination.

        \item[--environment {[env vars]}] Blank-delimited list of environment variables. If
          specified, this is used for all DUCC processes in the job. Example:
\begin{verbatim}
             --environment "TERM=xterm DISPLAY=me.org.net:1.0". 
\end{verbatim}
          
          Note: On Secure Linux systems, the environment variable 
          LD\_LIBRARY\_PATH may not be passed to the user's program. If it is 
          necessary to pass LD\_LIBRARY\_PATH to the JP or JD processes, it must be 
          specified as DUCC\_LD\_LIBRARY\_PATH. Ducc (securely) passes this as 
          LD\_LIBRARY\_PATH, after the JP or JD has assumed the user's identity. For 
          example: 
             \begin{verbatim}
-environment TERM=xterm DISPLAY=:1.0 DUCC_LD_LIBRARY_PATH=/my/own/path
            \end{verbatim}

        \item[--help] Prints the usage text to the console.

        \item[--log\_directory {[path-to-log directory]} ]

          This specifies the path to the directory for the user logs. If not specified, the default is the 
          user's home directory. Example: 
\begin{verbatim}
--log_directory /home/bob 
\end{verbatim}
          
          Within this directory DUCC creates a sub-directory for each process, using the numerical 
          ID of the job. The format of the generated log file names as described
          \hyperref[chap:job-logs]{here}.
          
          Note: Note that --log\_directory specifies only the path to a directory where 
          logs are to be stored. In order to manage multiple processes running in multiple 
          machines DUCC, sub-directory and file names are generated by DUCC and may 
          not be directly specified. 

        \item[--process\_attach\_console] If specified, redirect remote process stdio is 
          redirected the local submitting console.
          
        \item[--process\_executable {[program name]}] This is the full path to a program to be
          executed.

        \item[--process\_executable\_args {[argument list]}] This is a list of arguments for
          {\em process\_executable}, if any.   When specified on a command-line the text usually must be
          surrounded by quotes to protect it from the shell.

        \item[--process\_memory\_size {[size]} ] This specifies the maximum amount of RAM in GB to
          be allocated to each process.  This value is used by the Resource Manager to allocate
          resources. if this amount is exceeded by a process the Agent terminates the process with a
          ShareSizeExceeded message.

        \item[--scheduling\_class {[classname]} ] This specifies the name of the scheduling class the
          RM will use to determine the resource allocation for each process. The names of the
          classes are installation dependent. If not specified, the default is taken from the global
          DUCC configuration ducc.properties.

        \item[--specification, $-$f {[file]} ] All the parameters used to submit a process may be placed
          in a standard Java properties file.  This file may then be used to submit the process
          (rather than providing all the parameters directory to submit).
          
          For example, 
\begin{verbatim}
ducc_process_submit --specification job.props 
ducc_process_submit -f job.props 
\end{verbatim}

          where the job.props contains: 
\begin{verbatim}
working_directory   = /home/bob/projects
process_environment = AE_INIT_TIME=10000 DUCC_LD_LIBRARY_PATH=/a/bogus/path 
log_directory       = /home/bob/ducc/logs/ 
description         = Simple Process
scheduling_class    = fixed 
process_memory_size = 15 
\end{verbatim}

        \item[--wait\_for\_completion ] If specified, the submit command does not return control to
          the console immediately, and instead monitors the DUCC state traffic and prints
          information about the process as it progresses.
          
        \item[--working\_directory ] This specifies the working directory to be set by the Job
          Driver and Job Process processes.  If not specified, the current directory is used.

     \end{description}
        
    \paragraph{Notes:}

