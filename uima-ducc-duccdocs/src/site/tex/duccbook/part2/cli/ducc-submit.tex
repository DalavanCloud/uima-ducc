% Create well-known link to this spot for HTML version
\ifpdf
\else
\HCode{<a name='DUCC_CLI_SUBMIT'></a>}
\fi

    \section{ducc\_submit}
    \label{sec:cli.ducc-submit}
       The source for this section is ducc\_duccbook/documents/part-user/cli/submit.xml.
       \paragraph{Description:}
           The submit CLI is used to submit work for execution by DUCC. DUCC assigns a unique id to the
           job and schedules it for execution. The submitter may optionally request that the progress of
           the job is monitored, in which case the state of the job as it progresses through its
           lifetime is printed on the console.
       \paragraph{Usage:}
           \begin{description}
             \item[Script wrapper] \ducchome/bin/ducc\_submit {\em options}
             \item[Java Main]      java -cp \ducchome/lib/uima-ducc-cli.jar org.apache.uima.ducc.cli.DuccJobSubmit {\em options}
           \end{description}

        \paragraph{Options:}
           \begin{description}

           \item[$--$all\_in\_one $<$local $|$ remote $>$]
               Run driver and pipeline in single process.  If {\em local} is specified, the
               process is executed on the local machine, for example, in the current Eclipse session.
               If {\em remote} is specified, the jobs is submitted to DUCC as a {\em manged reservation}
               and run on some (presumably larger) machine allocated by DUCC.

           \item[$--$attach\_console] If specified, redirect remote stdout and stderr
             to the local submitting console.

           \item[$--$cancel\_on\_interrupt].  If the job is started with $--$wait\_for\_completion, this
             option causes the job to be canceled if the submit command is terminated,
             e.g., with CTL-C. If $--$cancel\_job\_on\_interrupt is not
             specified, the job monitor will be terminated but the job will continue to run.

             If $--$wait\_for\_completion is not specified this option is ignored. 

           \item[$--$classpath] The CLASSPATH used for the job.  If specified, this is used
             for both the Job Driver and each Job Process. If not specified the CLASSPATH found by the underlying
             {\tt DuccJobSubmit.main()} method is used.

           \item[$--$classpath\_order {[user-before-ducc $|$ ducc-before-user]} ]
             When DUCC deploys a process, set the user-supplied CLASSPATH before DUCC-supplied
             CLASSPATH, or the reverse. The default is user-before-ducc.
             
           \item[$--$debug] Enable debugging messages. This is primarily for debugging DUCC itself.

           \item[$--$description {[text]}] The text is any string used to describe the job. It is
             displayed in the Web Server. When specified on a command-line the text usually 
             must be surrounded by quotes to protect it from the shell.  The default is ``none''.

           \item[$--$driver\_debug {[debug-port]}] Append JVM debug flags to the JVM arguments
             to start the JobDriver in remote debug mode.  The remote process debugger will attempt
             to contact the specified port.

           \item[$--$driver\_descriptor\_CR {[descriptor.xml]} ] This is the XML descriptor for the
             Collection Reader.  This descriptor is a resource that is searched for in the CLASSPATH
             and data path as described in the ~\hyperref[par:cli.submit.notes]{notes below}.

           \item[$--$driver\_descriptor\_CR\_overrides {[list]} ]             
             This is the Job Driver collection reader configuration overrides. They are specified as 
             name/value pairs in a comma-delimited list. For example: 
             \begin{verbatim}
--driver_descriptor_CR_overrides name1=value1,name2=value2...
             \end{verbatim}
             
             
%           \item[$--$driver\_environment {[list]} ]
%
%             This specifies environment parameters for the Job Driver. If present, they are added to the 
%             Job Driver's environment as the process is spawned. It must be a quoted, blank-delimeted 
%             lsit of name-value pairs. For example: 
%             \begin{verbatim}
%"TERM=xterm DISPLAY=:1.0" 
%             \end{verbatim}
%             
%             Note: On Secure Linux systems, the environemnt variable 
%             LD\_LIBRARY\_PATH may not be passed to the user's program. If it is 
%             necessary to pass LD\_LIBRARY\_PATH to the JP or JD processes, it must be 
%             specified as DUCC\_LD\_LIBRARY\_PATH. Ducc (securely) passes this as 
%             LD\_LIBRARY\_PATH, after the JP or JD has assumed the user's identity. For 
%             example: 
%             \begin{verbatim}
%"--process\_environment TERM=xterm DISPLAY=:1.0 DUCC\_LD\_LIBRARY\_PATH=/my/own/
%            \end{verbatim}

           \item[$--$driver\_exception\_handler {[classname]}] This specifies a developer-supplied
             exception handler for the Job Driver.  It must
             implement org.apache.uima.ducc.common.jd.plugin.IJdProcessExceptionHandler or extend
             org.apache.uima.ducc.common.jd.plugin.AbstractJdProcessExceptionHandler.  A default
             handler is provided.


           \item[$--$driver\_jvm\_args {[list]} ]

             This specifies extra JVM arguments to be provided to the Job Driver process. It is a blank delimited 
             list of strings. Example: 
             \begin{verbatim}
--driver_jvm_args -Xmx100M -Xms50M 
             \end{verbatim}

             Note: When used as a CLI option, the list must usually be
             quoted to protect it from the shell.
             
           \item[$--$environment {[env vars]}] Blank-delimited list of environment variable
             assignments. If specified, this is used for all DUCC processes in the job. Example:
             \begin{verbatim}
--environment TERM=xterm DISPLAY=:1.0
             \end{verbatim}
             
             Additional entries may be copied from the user's environment based on the setting of
             ducc.submit.environment.propagated in the global DUCC configuration ducc.properties.

             Note: When used as a CLI option, the environment string must usually be
             quoted to protect it from the shell.

           \item[$--$help ]

             Prints the usage text to the console. 

           \item[$--$jvm {[path-to-java]}  ]

             States the JVM to use. If not specified, the same JVM used by the Agents is used.  This is
             the full path to the JVM, not the JAVA\_HOME.
             Example: 
\begin{verbatim}
--jvm /share/jdk1.6/bin/java 
\end{verbatim}
             
           \item[$--$log\_directory {[path-to-log-directory]} ]

             This specifies the path to the directory for the user logs. If not specified, the default is
             \$HOME/ducc/logs. Example: 
             \begin{verbatim}
--log_directory /home/bob 
             \end{verbatim}
             
             Within this directory DUCC creates a sub-directory for each job, using the unique numerical 
             ID of the job. The format of the generated log file names as described
             \hyperref[chap:job-logs]{here}.
             
             Note: Note that $--$log\_directory specifies only the path to a directory where 
             logs are to be stored. In order to manage multiple processes running in multiple 
             machines, sub-directory and file names are generated by DUCC and may 
             not be directly specified. 

           \item[$--$process\_DD {[DD descriptor]}  ]

             This specifies a UIMA Deployment Descriptor for the job processes for DD-style jobs. 
             This is mutually exclusive with $--$process\_descriptor\_AE, $--$process\_descriptor\_CM, 
             and $--$process\_descriptor\_CC. This descriptor is a resource that is searched for in the 
             CLASSPATH and data path as described in the ~\hyperref[par:cli.submit.notes]{notes below}.
             For example:
             \begin{verbatim}
--process_DD /home/billy/resource/DD_foo.xml 
             \end{verbatim}

           \item[$--$process\_debug {[debug-port]}] Append JVM debug flags to the JVM
             arguments to start the Job Process in remote debug mode.  The remote process will start
             its debugger and attempt to contact the debugger (usually Eclipse) on the specified
             port.
             
           \item[$--$process\_deployments\_max {[integer]} ]

             This specifies the maximum number of Job Processes to deploy at any given time. If not 
             specified, DUCC will attempt to provide the largest number of processes within the 
             constraints of fair\_share scheduling and the amount of work remaining.
             in the job. Example:
             \begin{verbatim}
--process_deployments_max 66 
             \end{verbatim}


           \item[$--$process\_descriptor\_AE {[descriptor]}  ]

             This specifies the Analysis Engine descriptor to be deployed in the Job Processes. This 
             descriptor is a resource that is searched for in the CLASSPATH and data path as described 
             in the ~\hyperref[par:cli.submit.notes]{notes below}.
             It is mutually exclusive with $--$process\_DD For example: 
             \begin{verbatim}
--process_descriptor_AE /home/billy/resource/AE_foo.xml 
             \end{verbatim}


           \item[$--$process\_descriptor\_AE\_overrides {[list]}  ]

             This specifies AE overrides. It is a comma-delimited list of name/value pairs. Example: 
             \begin{verbatim}
--process_descriptor_AE_Overrides name1=value1,name2=value2 
             \end{verbatim}
             
           \item[$--$process\_descriptor\_CC {[descriptor]}  ]

             This specifies the CAS Consumer descriptor to be deployed in the Job Processes. This 
             descriptor is a resource that is searched for in the CLASSPATH and data path as described 
             in the ~\hyperref[par:cli.submit.notes]{notes below}.
             It is mutually exclusive with $--$process\_DD For example: 
             \begin{verbatim}
--process_descriptor_CC /home/billy/resourceCCE_foo.xml 
             \end{verbatim}
             
           \item[$--$process\_descriptor\_CC\_overrides {[list]}  ]

             This specifies CC overrides. It is a comma-delimited list of name/value pairs. Example: 
             \begin{verbatim}
--process_descriptor_CC_overrides name1=value1,name2=value2 
             \end{verbatim}
             
           \item[$--$process\_descriptor\_CM {[descriptor]} ]

             This specifies the CAS Multiplier descriptor to be deployed in the Job Processes. This 
             descriptor is a resource that is searched for in the CLASSPATH and data path as described 
             in the ~\hyperref[par:cli.submit.notes]{notes below}.
             It is mutually exclusive with $--$process\_DD For example: 
             \begin{verbatim}             
--process_descriptor_CM /home/billy/resource/CM_foo.xml 
             \end{verbatim}

           \item[$--$process\_descriptor\_CM\_overrides {[list]}  ]

             This specifies CM overrides. It is a comma-delimited list of name/value pairs. Example: 
             \begin{verbatim}
--process_descriptor_CM_overrides name1=value1,name2=value2 
\end{verbatim}
             
           \item[$--$process\_failures\_limit {[integer]} ]

             This specifies the maximum number of individual Job Process (JP) failures allowed
             before killing the job. The default is fifteen(15). If this limit is exceeded over the lifetime 
             of a job DUCC terminates the entire job. 
             \begin{verbatim}
--process_failures_limit 23
\end{verbatim}
                          
           \item[$--$process\_initialization\_failures\_cap {[integer]} ] This specifies the maximum
             number of failures during a UIMA process's initialization phase.  If the number is
             exceeded the system will allow processes which are already running to continue, but
             will assign no new processes to the job.  The default is ninety-nine(99). Example:
             \begin{verbatim}
--process_initialization_failures_cap 62 
             \end{verbatim}
             
             Note that the job is NOT killed if there are processes that have passed initialization and are 
             running. If this limit is reached, the only action is to not start new processes for the job. 

           \item[$--$process\_initialization\_time\_max {[integer]}] This is the maximum time a process
             is allowed to remain in the ``initializing'' state, before DUCC terminates it.  The error
             counts as an initialization error towards the initialization failure cap.

           \item[$--$process\_jvm\_args {[list]} ] This specifies additional arguments to be passed to
             the Job Process JVM as a blank-delimited list of strings. Example:
             \begin{verbatim}
--process_jvm_args -Xmx400M -Xms100M
             \end{verbatim}

             Note: When used as a CLI option, the arguments must usually be
             quoted to protect them from the shell.
                          
           \item[$--$process\_memory\_size {[size]} ] This specifies the maximum amount of RAM in GB
             to be allocated to each Job Process.  This value is used by the Resource Manager to
             allocate resources.

           \item[$--$process\_per\_item\_time\_max {[integer]} ] This specifies the maximum time in
             minutes that the Job Driver will wait for a Job Processes to process a CAS. If a
             timeout occurs the process is terminated and the CAS marked in error (not retried). If
             not specified, the default is 1 minute. Example:
             \begin{verbatim}
--process_per_item_time_max 60 
             \end{verbatim}
             
           \item[$--$process\_thread\_count {[integer]} ] This specifies the number of threads per
             process to be deployed. It is used by the Resource Manager to determine how many
             processes are needed, by the Job Process wrapper to determine how many threads to
             spawn, and by the Job Driver to determine how many CASs to dispatch. If not specified,
             the default is 4. Example:
             \begin{verbatim}
--process_thread_count 7 
             \end{verbatim}
             
           \item[$--$scheduling\_class {[classname]} ] This specifies the name of the scheduling class
             the RM will use to determine the resource allocation for each process. The names of the
             classes are installation dependent. If not specified, the default is taken from the
             global DUCC configuration ducc.properties.  Example:
             \begin{verbatim}
--scheduling_class normal 
             \end{verbatim}
          

           \item[$--$service\_dependency{[list]}] This specifies a comma-delimited list of services the job
             processes are dependent upon. Service dependencies are discussed in detail
             \hyperref[sec:service.endpoints]{here}. Example:
\begin{verbatim}
--service_dependency UIMA-AS:RandomSleepAE:tcp:bluej682:61616 UIMA-AS:OtherEp:tcp:bluej123:123 
\end{verbatim}

           \item[$--$specification, $-$f {[file]}  ]

             All the parameters used to submit a job may be placed in a standard Java properties file. 
             This file may then be used to submit the job (rather than providing all the parameters 
             directory to submit). The leading $--$ is omitted from the keywords.

             For example, 
\begin{verbatim}
ducc_submit --specification job.props 
ducc_submit -f job.props 
\end{verbatim}

             where job.props contains: 
\begin{verbatim}
working_directory                   = /home/bob/projects/ducc/ducc_test/test/bin 
process_failures_limit              = 20 
driver_descriptor_CR                = org.apache.uima.ducc.test.randomsleep.FixedSleepCR 
environment                         = AE_INIT_TIME=10000 LD_LIBRARY_PATH=/a/bogus/path 
log_directory                       = /home/bob/ducc/logs/ 
process_thread_count                = 1 
driver_descriptor_CR_overrides      = jobfile:../simple/jobs/1.job,compression:10 
process_initialization_failures_cap = 99 
process_per_item_time_max           = 60 
driver_jvm_args                     = -Xmx500M 
process_descriptor_AE               = org.apache.uima.ducc.test.randomsleep.FixedSleepAE 
classpath                           = /home/bob/duccapps/ducky_process.jar 
description                         = ../simple/jobs/1.job[AE] 
process_jvm_args                    = -Xmx100M -DdefaultBrokerURL=tcp://localhost:61616 
scheduling_class                    = normal 
process_memory_size                 = 15 
\end{verbatim}

             Note that properties in a specifications file may be overridden by other command-line
             parameters, as discussed \hyperref[chap:cli]{here}.

           \item[$--$suppress\_console\_log] If specified, suppress creation of the log files that 
             normally hold the redirected stdout and stderr.

           \item[$--$time-stamp ]

             If specified, messages from the submit process are timestamped. This is intended primarily 
             for use with a monitor with --wait\_for\_completion. 

           \item[$--$wait\_for\_completion ]             
             If specified, the submit command monitors the job and prints periodic
             state and progress information to the console.  When the job completes, the monitor
             is terminated and the submit command returns.
             
           \item[$--$working\_directory ]             
             This specifies the working directory to be set by the Job Driver and Job Process processes. 
             If not specified, the current directory is used.
  \end{description}
             
  \paragraph{Notes:}
  \phantomsection\label{par:cli.submit.notes}
  When searching for UIMA XML resource files such as descriptors, DUCC searches both the 
  CLASSPATH and the data path according to the following rules: 
  
  \begin{enumerate}
  \item If the resource ends in .xml it is assumed the resource is a file and the path is either an 
    absolute path or a path relative to the specified working directory. If the file is not found 
    the search exits and the job is terminated. 
    
  \item If the resource does not end in .xml, DUCC creates a path by replacing the "." 
    separators with "/" and appending ".xml". It then searches two places: 
    \begin{enumerate}
    \item The user's CLASSPATH as a file (that is, not in a jar), and 
    \item In the jar files provided in the user's CLASSPATH. 
    \end{enumerate}
    If the resource is found in either place the search is successful. Otherwise the search 
    fails and the job is terminated. 
    
    The resource search-order rules apply to all of the following submit parameters: 
    \begin{itemize}
    \item[]$--$driver\_descriptor\_CR 
    \item[]$--$process\_descriptor\_AE 
    \item[]$--$process\_descriptor\_CC 
    \item[]$--$process\_descriptor\_CM 
   \end{itemize}
 \end{enumerate}
