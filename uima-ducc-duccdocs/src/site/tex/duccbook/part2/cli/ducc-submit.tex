    \section{ducc\_submit}
       The source for this section is ducc\_duccbook/documents/part-user/cli/submit.xml.
       \paragraph{Description:}
           The submit CLI is used to submit work for execution by DUCC. DUCC assigns a unique id to the
           job and schedules it for execution. The submitter may optionally request that the progress of
           the job is monitored, in which case the state of the job as it progresses through its
           lifetime is printed on the console.
       \paragraph{Usage:}
           \begin{description}
             \item[Executable Jar] java -jar \ducchome/lib/uima-ducc-submit.jar {\em options}
             \item[Script wrapper] java -jar \ducchome/bin/ducc-submit {\em options}
             \item[Java Main]      java -cp \ducchome/lib/uima-ducc-submit.jar org.apache.uima.ducc.cli.DuccJobSubmit {\em options}
           \end{description}

        \paragraph{Options:}
           \begin{description}

           \item[--all\_in\_one $<$local | remote $>$]
               Run driver and pipeline in single process.  If {\em local} is specified, the
               process is executed on the local machine, for example, in the current Eclipse session.
               If {\em remote} is specified, the jobs is submitted to DUCC as a {\em manged reservation}
               and run on some (presumably larger) machine allocated by DUCC.

           \item[--cancel\_job\_on\_interrupt].  If the job is started with --wait\_for\_completion, this
             option causes the job to be canceled with Ctrl-C. If --cancel\_job\_on\_interrupt is not
             specified, the job monitor will be terminated but the job will continue to run.

             If --wait\_for\_completin is not specified this option is ignored. 

           \item[--classpath] The classpath used for the job.  If specified, this is used
             for both the driver and each process, overriding {\em process\_classpath} and
             {\em driver\_classpath}.

           \item[--classpath\_order {[UserBeforeDucc | DuccBeforeUser]} ]
             When DUCC deploys a process, set the user-supplied classpath before DUCC-supplied
             classpath, or the reverse.
             
           \item[--debug] Enable debugging messages. This is primarily for debugging DUCC itself.

           \item[--description {[text]}] The text is any string used to describe the job. It is
             displayed in the Web Server.             

           \item[--driver\_attach\_console] If specified, redirect remote job driver stdout and stderr
             to the local submitting console.

           \item[--driver\_classpath {[classpath]}]
             This is the classpath for the Job Driver, necessary for DUCC to find the Collection Reader. 

           \item[--driver\_debug {[debugger-address]}] Append JVM debug flags to the jvm arguments
             to start the JobDriver in remote debug mode.  The remote process debugger will attempt
             to contact the specified port. The address is of the form {\tt host:port}.

           \item[--driver\_descriptor\_CR {[descriptor.xml]} ]
             This is the XML descriptor for the Collection Reader. It is searched for as a resource as 
             described above. 

           \item[--driver\_descriptor\_CR\_overrides {[list]} ]
             
             This is the Job Driver collection reader configuration overrides. They are specified as 
             name/value pairs in a comma-delimeted list. For example: 
             \begin{verbatim}
--driver\_descriptor\_CR\_overrides name1=value1,name2=value2...
             \end{verbatim}
             
             
           \item[--driver\_environment {[list]} ]

             This specifies environment parameters for the Job Driver. If present, they are added to the 
             Job Driver's environment as the process is spawned. It must be a quoted, blank-delimeted 
             lsit of name-value pairs. For example: 
             \begin{verbatim}
"TERM=xterm DISPLAY=:1.0" 
             \end{verbatim}
             
             Note: On Secure Linux systems, the environemnt variable 
             LD\_LIBRARY\_PATH may not be passed to the user's program. If it is 
             necessary to pass LD\_LIBRARY\_PATH to the JP or JD processes, it must be 
             specified as DUCC\_LD\_LIBRARY\_PATH. Ducc (securely) passes this as 
             LD\_LIBRARY\_PATH, after the JP or JD has assumed the user's identity. For 
             example: 
             \begin{verbatim}
"--process\_environment TERM=xterm DISPLAY=:1.0 DUCC\_LD\_LIBRARY\_PATH=/my/own/
            \end{verbatim}

           \item[--driver\_exception\_handler {[classname]}] Driver Exception handler class.  Must
             implement org.apache.uima.ducc.common.jd.plugin.IJdProcessExceptionHandler.

           \item[--driver\_jvm\_args {[list]} ]

             This specifes extra JVM arguments to be provided to the Job Driver process. It is a blankdelimeted 
             list of strings. Example: 
             \begin{verbatim}
--driver\_jvm\_args -Xmx100M -Xms50M 
             \end{verbatim}
             
           \item[--driver\_memory\_size {[size-in-GB]} ]

             This specifies the size of memory for the Job Driver, in GB. Example: 
             \begin{verbatim}
--driver\_memory\_size 16 
             \end{verbatim}

           \item[--environment {[env vars]}] Blank-delimeted list of environment variables.  example:
             "TERM=xterm DISPLAY=me.org.net:1.0".  If specified, this is used for all DUCC processes
             in the job, overriding {\em --driver\_environment} and {\em --process\_environment} if they
             are specified.
             
           \item[--help ]

             Prints the usage text to the console. 

           \item[--jvm {[path-to-java]}  ]

             States the JVM to use. If not specified, the same JVM used by the Agents is used. 
             Example: 
             \begin{verbatim}
--jvm /share/jdk1.6/bin/java 
             \end{verbatim}
             
           \item[--log\_directory {[path-to-log directory]} ]

             This specifies the path to the directory for the user logs. If not specified, the default is the 
             user's home directory. Example: 
             \begin{verbatim}
--log\_directory /home/bob 
             \end{verbatim}
             
             Within this directory DUCC creates a subdirectory for each job, using the numerical 
             ID of the job. The format of the generated log file names is descripbed in Chapter 5, Job 
             Logs [47]. 
             
             Note: Note that --log\_directory specifies only the path to a directory where 
             logs are to be stored. In order to manage multiple processes running in multiple 
             machines DUCC, sub-directory and file names are generated by DUCC and may 
             not be directly specified. 

           \item[--process\_attach\_console] If specified, redirect remote process (as
             opposed to driver) stdout and stderr to the local submitting console.

           \item[--process\_classpath {[ClASSPATH]} ]

             This specifies the Java CLASSPATH to use in each Job Process (JP) and must be 
             specified. Example: 
             \begin{verbatim}
--process\_classpath a.jar:b.jar 
             \end{verbatim}
             
           \item[--process\_DD {[DD descriptor]}  ]

             This specifies a UIMA Deployment Descriptor for the job processes for DD-style jobs. 
             This is mutually exclusive with --process\_descriptor\_AE, --process\_descriptor\_CM, 
             and --process\_descriptor\_CC. This descriptor is a resource that is searched for in the 
             CLASSPATH and data path as described in the notes [26]. For example: 
             \begin{verbatim}
--process\_DD /home/billy/resource/DD\_foo.xml 
             \end{verbatim}

           \item[--process\_debug {[debugger-address]}] Append JVM debug flags to the jvm
             arguments to start the Job Process in remobe debug mode.  The remote process will
             start its debugger and attempt to contact the (Eclipse) debugger on the specified port.
             The address is of the form {\tt host:port}.
             
           \item[--process\_deployments\_max {[integer]} ]

             This specifies the maximum nunber of Job Processes to deploy at any given time. If not 
             specified, DUCC will attempt to provide the largest number of processes, within the 
             constraints of fair\_share scheduling and the number of pending work items still to be done 
             in the job. 
             \begin{verbatim}
--process\_deployments\_max 66 
             \end{verbatim}


           \item[--process\_descriptor\_AE {[descriptor]}  ]

             This specifies Analysis Engine descriptor to be deployed in the Job Processes. This 
             descriptor is a resource that is searched for in the CLASSPATH and data path as described 
             in the notes. It is mutually exclusive with --process\_DD For example: 
             \begin{verbatim}
--process\_descriptor\_AE /home/billy/resource/AE\_foo.xml 
             \end{verbatim}


           \item[--process\_descriptor\_AE\_overrides {[list]}  ]

             This specifies AE overrides. It is a comma-delimeted list of name/value pairs. Example: 
             \begin{verbatim}
--process\_descriptor\_AE\_Overrides name1=value1,name2=value2 
             \end{verbatim}
             
           \item[--process\_descriptor\_CC {[descriptor]}  ]

             This specifies the CAS Consumer descriptor to be deployed in the Job Processes. This 
             descriptor is a resource that is searched for in the CLASSPATH and data path as described 
             in the notes. It is mutually exclusive with --process\_DD For example: 
             \begin{verbatim}
--process\_descriptor\_CC /home/billy/resourceCCE\_foo.xml 
             \end{verbatim}
             
           \item[--process\_descriptor\_CC\_overrides {[list]}  ]

             This specifies CC overrides. It is a comma-delimeted list of name/value pairs. Example: 
             \begin{verbatim}
--process\_descriptor\_CC\_overrides name1=value1,name2=value2 
             \end{verbatim}
             
           \item[--process\_descriptor\_CM {[descriptor]} ]

             This specifies the CAS Multiplier descriptor to be deployed in the Job Processes. This 
             descriptor is a resource that is searched for in the CLASSPATH and data path as described 
             in the notes. It is mutually exclusive with --process\_DD For example: 
             \begin{verbatim}             
--process\_descriptor\_CM /home/billy/resource/CM\_foo.xml 
             \end{verbatim}

           \item[--process\_descriptor\_CM\_overrides {[list]}  ]

             This specifies CM overrides. It is a comma-delimeted list of name/value pairs. Example: 
             \begin{verbatim}
--process\_descriptor\_CM\_overrides name1=value1,name2=value2 
\end{verbatim}
             
           \item[--process\_environment {[environment]} ]

             This specifies environment parameters for the Job Processes. If present, they are added 
             to the Job Process environment as the process is spawned. It must be a quoted, blankdelimeted 
             lsit of name-value pairs. For example: 
             \begin{verbatim}

"--process\_environment TERM=xterm DISPLAY=:1.0" 
             \end{verbatim}
  
             Note: On Secure Linux systems, the environemnt variable 
             LD\_LIBRARY\_PATH may not be passed to the user's program. If it is 
             necessary to pass LD\_LIBRARY\_PATH to the JP or JD processes, it must be 
             specified as DUCC\_LD\_LIBRARY\_PATH. Ducc (securely) passes this as 
             LD\_LIBRARY\_PATH, after the JP or JD has assumed the user's identity. For 
             example: 

             \begin{verbatim}
"--process\_environment TERM=xterm DISPLAY=:1.0 DUCC\_LD\_LIBRARY\_PATH=/my/own/
             \end{verbatim}

           \item[--process\_failures\_limit {[integer]} ]

             This specifies the maximum number of individual Job Process (JP) failures that are to be 
             tolerated before killing the job. The default is 15. If this limit is exceeded over the lifetime 
             of a job DUCC terminates the entire job. 
             \begin{verbatim}
"--process\_failures\_limit 23" 
\end{verbatim}
             
           \item[--process\_get\_meta\_time\_max {[integer]}  ]

             When a job is started the Job Driver issus a single "get-meta" requests to the
             (DUCCgenerated) endpoint of the JP processes for the job to insure that at least one
             UIMA-AS server processes for the job have started. This parameter specifies the time in
             seconds to wait for a response. If the request times out the Job Driver assumes that no
             UIMA-AS service for the job was able to start and it terminates the job. If not
             specified, the timeout is 2 minutes. Example:
             \begin{verbatim}
"--process\_get\_meta\_time\_max 10" 
             \end{verbatim}
             
           \item[--process\_initialization\_failures\_cap {[integer]}  ]

             This specifies the maximum number of independent Job Process initialization failures (i.e. 
             System.exit(), kill-15...) before the number of Job Processes is capped at the number in 
             state Running currently. The default is 99. Example: 
             \begin{verbatim}
--process\_initialization\_failures\_cap 62 
             \end{verbatim}
             
             Note that the job is NOT killed if there are processes that have passed initialization and are 
             running. If this limit is reached, the only action is to not start new processes for the job. 

           \item[--process\_jvm\_args {[list]}  ]

             This specifies additinal arguments to be passed to the Job Process JVM. Example: 
             \begin{verbatim}
--process\_jvm\_args -Xmx400M -Xms100M 
             \end{verbatim}
             
           \item[--process\_memory\_size {[size]}  ]

             This specifies the maximum amount of RAM in GB to be allocated to each Job Process. 
             This value is used by the Resource Manager to allocate resources. if this amount is 
             exceeded by a Job Process the Agent terminates the process with a ShareSizeExceeded 
             message.

           \item[--process\_per\_item\_time\_max {[integer]}  ]

             This specifies the maximum time in minutes that the Job Driver will wait for a Job 
             Processes to process a CAS. If a timeout occurs the process is terminated and the CAS 
             marked in error (not retried). If not specified, the default is 1 minute. Example: 
             \begin{verbatim}
--process\_per\_item\_time\_max 60 
             \end{verbatim}
             
           \item[--process\_thread\_count {[integer]} ]

             This specifies the number of threads per process to be deployed. It is used by the Resource 
             Manager to determine how many processes are needed, by the Agent to determine 
             howmany threads to spawn, and by the Job Driver to determine how many CASs to 
             dispatch. If not specified, the default is 4. Example: 
             \begin{verbatim}
--process\_thread\_count 7 
             \end{verbatim}
             
           \item[--scheduling\_class {[classname]}  ]

             This specifies the name of the scheuling class the RM will use to determine the resource 
             allocation for each process. The names of the classes are installation dependent. If not 
             specified, the default is taken from the global DUCC configuration ducc.properties. 
             Example: 
             \begin{verbatim}
--schedling\_class normal 
             \end{verbatim}
             
           \item[--service\_dependency{[list]}  ]

             This specifies a comma-delimeted list of services the job processes are dependent upon. 
             Each endpoint must be of the form UIMA-AS:endpoint:broker\_url where endpoint is the 
             UIMA-AS service endpoint and broker\_url is the ActiveMQ broker URL. 
             
             In the example are two dependencies, one with endpoint RandomSleepAE and 
             broker tcp:bluej682:61616, and the other with endpoint OtherEp and broker URL 
             tcp:bluej123:123. Example:
             \begin{verbatim}
 --service\_dependency UIMA-AS:RandomSleepAE:tcp:bluej682:61616, \ 
UIMA-AS:OtherEp:tcp:bluej123:123 
             \end{verbatim}

           \item[--specifiecaiton {[file]}  ]

             All the parameters used to submit a job may be placed in a standard Java properties file. 
             This file may then be used to submit the job (rather than providing all the parameters 
             directory to submit). 

             For example, 
             \begin{verbatim}
ducc\_submit --specification job.props 
\end{verbatim}

             where the job.props contains: 
\begin{verbatim}
working\_directory=/Users/challngr/projects/ducc/ducc\_test/test/bin 
process\_get\_meta\_time\_max=5 
process\_failures\_limit=20 
driver\_descriptor\_CR=org.apache.uima.ducc.test.randomsleep.FixedSleepCR 
driver\_environment=DUCC\_LD\_LIBRARY\_PATH=/a/other/bogus/path 
process\_environment=AE\_INIT\_TIME=10000 DUCC\_LD\_LIBRARY\_PATH=/a/bogus/path 
driver\_classpath=/home/bob/duccapps/ducky\_driver.jar 
log\_directory=/Users/challngr/ducc/logs/ 
process\_thread\_count=1 
driver\_descriptor\_CR\_overrides=jobfile:../simple/jobs/1.job,compression:10 
process\_initialization\_failures\_cap=99 
process\_per\_item\_time\_max=60 
driver\_jvm\_args=-Xmx500M 
process\_descriptor\_AE=org.apache.uima.ducc.test.randomsleep.FixedSleepAE 
process\_classpath=/home/bob/duccapps/ducky\_process.jar 
description=../simple/jobs/1.job[AE] 
process\_jvm\_args=-Xmx100M -DdefaultBrokerURL=tcp://localhost:61616 
scheduling\_class=normal 
process\_memory\_size=15 
\end{verbatim}
             
           \item[--timestamp ]

             If specified, messages from the submit process are timestamped. This is intended primarily 
             for use with a monitor with --wait\_for\_completion. 

           \item[--wait\_for\_completion ]             
             If specified, the submit command does not return control to the consoke immediately, 
             and instead monitors the DUCC state traffic and prints information about the job as it 
             progresses. 
             
           \item[--working\_directory ]             
             This specifies the working directory to be set by the Job Driver and Job Process processes. 
             If not specified, the current directory is used.
  \end{description}
             
  \paragraph{Notes:}
  
  When searching for UIMA XML resource files such as descriptors, DUCC searches both the 
  classpath and the data path according to the following rules: 
  
  \begin{enumerate}
  \item If the resource ends in .xml it is assumed the resource is a file and the path is either an 
    absolute path or a path relative to the specified working directory. If the file is not found 
    the search exits and the job is terminated. 
    
  \item If the resource does not end in .xml, DUCC creates a path by replacing the "." 
    separators with "/" and appending ".xml". It then searches two places: 
    \begin{enumerate}
    \item The user's CLASSPATH as a file (that is, not in a jar), and 
    \item In the jar files provided in the user's CLASSPATH. 
    \end{enumerate}
    If the resource is found in either place the search is successful. Otherwise the search 
    fails and the job is terminated. 
    
    The resource search-order rules apply to all of the following submit parameters: 
    � --driver\_descriptor\_CR 
    � --process\_descriptor\_AE 
    � --process\_descriptor\_CC 
    � --process\_descriptor\_CM 
  \end{enumerate}
