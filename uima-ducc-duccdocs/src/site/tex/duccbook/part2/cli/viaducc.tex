% Create well-known link to this spot for HTML version
\ifpdf
\else
\HCode{<a name='DUCC_VIADUCC'></a>}
\fi
    \section{viaducc and java\_viaducc}
    \label{sec:cli.viaducc}

    \paragraph{Description:}
        Viaducc is a small script wrapper around the \hyperref[sec:cli.ducc-process-submit]{\em
          ducc\_process\_submit } CLI to facilitate execution of Eclipse workspaces directly in DUCC,
        and to provide a simple interface to submit arbitrary processes.

        When run as the command ``viaducc'', the arguments are bundled into the form expected by
        \hyperref[sec:cli.ducc-process-submit]{\em ducc\_process\_submit } and submits the command
        to DUCC.

        When run as the command ``java\_viaducc'', the arguments are assumed to be a Java classname
        and its arguments.  This is passed for execution by DUCC via DUCC's default JRE, or
        optionally, a specific JRE supplied by the user.

        If ``java\_viaducc'' is installed as a symbolic link from the local JRE/bin directory to the
        ``viaducc'' command in the DUCC runtime/bin directory, it may be specified as an alternative
        to the ``java'' command in an eclipse launcher.  The remote stdin and stdout of the deployed
        DUCC process are redirected to your Eclipse console.  This provides essentially transparent
        execution of code in your Eclipse workspaces in DUCC-managed resources.

    \paragraph{Usage:}
\begin{verbatim}
     viaducc [defines] [command and parameters]"
\end{verbatim}
     or
\begin{verbatim}
     java_viaducc [defines] [java-class and parameters]"
\end{verbatim}

     The ``defines'' are described below.  The ``command and parameters'' are either any command
     (with full path) and it's arguments, or a Java class (with a ``main'') and its arguments (including
     the classpath if necessary.)

     \paragraph{Defines}

     The arguments are specified in the syntax of Java ``-D'' system properties, to be more consistent
     with execution under Eclipse.
     \begin{description}
         \item[-DDUCC\_MEMORY\_SIZE] This specifies the memory required, in GB.  If not specified, the
           smallest memory quanta configured for the scheduler is used.
         \item[-DDUCC\_CLASS] This is the scheduling class to submit the process to.  It should generally
           be a non-preemptable class.  If not specified, it defaults to class ``fixed''.
         \item[-DDUCC\_ENVIRONMENT] This species additional environment parameters to pass to the job.
           It should specify a quoted string of blank-delimeted K=V environment values.  For example:
\begin{verbatim}
      -DDUCC_ENVIRONMENT="DUCC_RLIMIT_NOFILE=1000 V1=V2 A=B"
\end{verbatim}
         \item[-DJAVA\_BIN] This species the exact ``java'' command to use, for ``java\_viaducc''.  it
           must be a full path to some JRE that is known to be installed on all the DUCC nodes.  If not
           specified, the JRE used to run ducc is uses.
    \end{description}
        
