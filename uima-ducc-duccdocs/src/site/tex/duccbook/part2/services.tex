
      \section{Overview.} 
      A DUCC service is defined by the following two criteria:
      \begin{itemize}
          \item A service is one or more long-running processes that await requests from
            UIMA pipeline components and return something in response.  These processes
            are usually managed by DUCC but need not be.
          \item A sservice is accompanied by a small program called a ``pinger'' that
            the DUCC Service Manager uses to guage the availability and health of the
            service.  This process must always be supplied to DUCC; however, DUCC will
            supply a default pinger for UIMA-AS services.
      \end{itemize}

      A service is usually a UIMA-AS service, but DUCC supports any arbitrary process
      as a service.

      The DUCC service manager implements several high-level functions:
      
      \begin{itemize}
          \item Insure services are available for jobs before allowing the jobs to start. This fail-fast
            prevents unncessary allocation of resources (with potential eviction of healthy processes)
            for jobs that can't run, as well as quick feedback to users that something is amis.
      
          \item Manage the startup and management of services: allocate resources, spawn the
            processes, insure the processes stay alive, handle errors, etc.
      
          \item Report on the state and availablity of services.
       \end{itemize}


    \section{Service Types.}
    \label{sec:services.types}
      DUCC supports two types of services: UIMA-AS and CUSTOM:
      
      \begin{description}
          \item[UIMA-AS] This is a "normal" UIMA-AS service. DUCC fully supports all aspects of UIMA-AS
            services with minimal effort from developers.  A default ``pinger'' is supplied by DUCC
            for UIMA-AS services.  (It is legal to define a CUSTOM pinger for a UIMA-AS service,
            however.)
            
          \item[CUSTOM] This is any arbitrary service.  Developers must provide a CUSTOM pinger
            and declare it in the service registration.            
      \end{description}

      \section{Service References and Endpoints} 
      \label{sec:service.endpoints}
      Services are referenced by an entity called a service
      endpoint. The service endpoint is a formatted string used to uniquely identify each
      service and to supply contact information to the pingers.  A service endpoint
      is of the form 
\begin{verbatim}
      <service-type>:<unique id and contact information>
\end{verbatim}
      
      The {\em service-type} must be either UIMA-AS or CUSTOM.
      
      The {\em unique id and contact information} is any string needed to insure the service is
      uniquely named.  This string is passsed to the service pinger and must contain sufficient
      information for the pinger to contact the service.  For UIMA-AS services, service endpoint is
      inferred by the CLI by inspection of the UIMA XML descriptor.  The UIMA-AS
      service endpoint is always of the form:
\begin{verbatim}
      UIMA-AS:queue-name:broker-url
\end{verbatim}
      where {\em queue-name} is the name of the ActiveMQ queue used by the service, and {\em broker-url}
      is the ActiveMQ broker URL.

      Jobs or other services may register dependencies on specific services by listing one or more
      service endpoints int their specifications. See the 
      \hyperref[sec:cli.ducc-submit]{\em job } and 
      \hyperref[sec:cli.ducc-services]{\em services } CLI descriptions for details.
      
      
      \section{Service Classes.} 
      \label{sec:service.classes}
      Services may be started externally to DUCC, explicitly through
      DUCC as a job, or as registered services.  It is also possible to register a ``ping-only''
      service that has no process managed by DUCC, consisting only of a pinger.  To distiinguish
      among the various behaviours and management mechanisms for service we define a number
      of {\em service classes}.

      \subsection{Implicit Services.} 
      \label{sec:services.implicit}
      An external service is started externally to DUCC and discovered by DUCC only when it is
      referenced by a job's {\em service\_dependency} parameter.  Such a service is called
      an {\em implicit} service.

      On submission of a job with a dependency on a, the SM sets up a "ping" thread based on the
      service endpoint of the dependency to discover if the service exists at the endpoint. If so,
      the SM adds the service to its list of known services and marks the job "ready to schedule".

      When jobs referencing implicit services exit, a timer is set and DUCC continues to monitor the
      service against the possibility that subsequent jobs will need it. Once the last job using the
      service has exited and the service timer expired, the SM stops the monitors and purges the
      service from its records.

      \subsubsection{Implicit UIMA-AS services}
      If a job \hyperref[sec:service.endpoints]{references} a UIMA-AS service that is not known to the
      DUCC Service Manager, the Service Manager will start
      its default internal pinger to monitoring the ActiveMq queue and service response.  The
      service is monitored throughout the lifetime of the job. If the service should stop
      responding, its state is updated to "not-responding" but the job is allowed to continue as
      DUCC cannot tell if the job is using the service.

      \subsubsection{Implicit CUSTOM services, or ``ping-only'' Service} 
      If a job \hyperref[sec:service.endpoints]{references} a CUSTOM service, the service must be
      registered and have a CUSTOM \hyperref[sec:service.pingers]{pinger} associated with it. Such a
      service is refered to as a ``ping-only'' service.  DUCC will start the pinger and monitor the
      service as expected.
      
      
      \subsection{Submitted Services.} A \hyperref[sec:cli.service-submit]{submitted service} is a
      service that is submitted to DUCC as a with the ducc\_service\_submit CLI.  Both UIMA-AS and
      CUSTOM services may be submitted for execution by DUCC. Because DUCC is managing this service
      it can provide more support than for implicit services.  However, DUCC does not persist the
      service definition. If a submitted service exits involuntarily (crashes), DUCC will make
      some number of attempts to restart it.  If the configured restart count is exceeded DUCC
      will stop the service.

      A submitted service is intended for short-term services, and for easy debugging of services.
      To persist a service over system restarts and other failures, services should be registered.
      
      \subsection{Registered Services.} \hyperref[sec:cli.ducc-services]{Registered services} are
      fully managed by DUCC. A service is registered with DUCC using the DUCC services CLI. Service
      registrations are persisted by DUCC and last over DUCC or system restarts.

      There are several variants on Registered Services:
      \begin{description}

        \item[Autostarted Services] An autostarted service is a registered service that is flagged to be
          automatically started when the DUCC system is started. When DUCC is started, the SM checks the
          service registry for all service that are marked for automatic startup.  If flagged for autostart,
          the DUCC Service Manager submits the registered numuber of instances
          on behalf of the registering user.  If an instance should die, DUCC automaticlly restarts
          the instance.  Short of massive failures, DUCC will insure the service is always running
          and immediately ready for use.
          
        \item[On-Demand Services] An on-demand service is a registered service that is started only
          when referenced by the service-dependency of another job or service. If the service is
          already started (e.g. by reference from some other job), the dependent job/service is
          marked ready to schedule as indicated above. If not, the service registry is checked and
          if a matching service is found, DUCC startes it. When the service has completed
          initialization a pinger is started and all jobs waiting on it are then started.
          
          When the last job or service that references the on-demand service exits, a timer is
          established to keep the service alive for a while (in anticipation that it will be needed
          again soon.)  When the keep-alive timer exipires, and there are no more dependent
          jobs or services, the on-demand service is automatically stopped to free up its resources for
          other work.

        \item[Ping-Only Services] 
          \phantomsection\label{subsub:services.ping-only}
          \hyperref[sec:services.implicit]{Ping-only services} consist of only
          a ping thread.  The service itself is not managed in any way by DUCC.  This is useful for
          managing dependencies on services that are not under DUCC control: DUCC can detect and
          report on the health of these services and take appropriate actions on dependent jobs if
          the services are not responsive.
      \end{description}
          
      \section{Service Pingers}
      \label{sec:service.pingers}
      A service pinger is a small program that queries a service on behalf of the
      DUCC Service Manager to:
      \begin{itemize}
        \item Report on the availiability of the service, and
        \item Report on the healh of the service.
      \end{itemize}
      
      Service pingers are always written in Java and must implement an abstract class,
\begin{verbatim}
      org.apache.uima.ducc.common.AServicePing
\end{verbatim}
      When a service is deployed by
      DUCC, the Service Manager spawns a DUCC process that instantiates the pinger for
      the service.  On a regular basis, the Service Manager sends a request to the pinger
      to query the service health.

      \subsection{Declaring a Pinger in A Service}

      If your service is a UIMA-AS service, there is no need to create or declare a pinger as  DUCC
      provides a default pinger.  If a CUSTOM pinger is required, it must be declared in the service
      descriptor, and the service must be registered.  See
      \hyperref[sec:cli.ducc-services]{ducc\_services} for details on service registration spcifying
      ping directives.

      \subsection{Implementing a Pinger}
      Pingers must implement the class {\tt org.apache.uima.ducc.common.AServicePing}.  The abstract class
      is shown below.
      \begin{figure}[H]
\begin{verbatim}
package org.apache.uima.ducc.common;

public abstract class AServicePing
{
    /**
     * Called by the ping driver, to pass in useful things the pinger may want.
     * @param endpoint This is the name of the service endpoint, as passed in
     *                 at service registration.
     */
    public abstract void init(String endpoint)  throws Exception;

    /**
     * Stop is called by the ping wrapper when it is being killed.  Implementors may optionally
     * override this method with conenction shutdown code.
     */
    public abstract void stop();

    /**
     * Returns the object with application-derived health and statistics.
     * @return {@link ServiceStatistcs} This object contains the informaton the service manager and web server require
     *     for correct management and display of the service.
     */
    public abstract ServiceStatistics getStatistics();
    
}
\end{verbatim}
        \caption{Service Ping Abstract Class}
        \label{fig:service.abstract.pinger}

      \end{figure}
      
      The ServiceStatistics class defines these methods:
      \begin{description}
        \item[ServiceStatistics(boolean alive, boolean healthy, String info)] This is the constructor.
          \begin{description}
            \item[boolean alive] Set this to ``true'' if the service is responsive.  If a pinger responds
              ``false'' (or does not respond), the Service Manager will assume the service is unavailable
              and will not allow jobs dependent on this service to start.  (Dependent jobs that are already
              started are allowed to continue, but are annoated in the web server, such that developers
              will know the job may not be functioning because of the service.)
            \item[boolean healthy] The pinger may perform analysis on the service to determine whether
              the service is ``healthy'' or not.  This is strictly subjective and is used by the
              Service Manager only for reporting to the web server.
            \item[String info] This is any string in any format.  The pinger sets health and availability
              data into it for display in the webserver.  For example, the default UIMA-AS pinger sets
              ActiveMQ service statistics into this string.)
          \end{description}
          
          \item[void setAlive(boolean alive)] Set the ``aliveness'' of the service.

          \item[void setUnhealthy(boolean val)] Set the ``healthiness'' of the service.
            
          \item[void setInfo(String info)] update the service information string.
      \end{description}

      A sample CUSTOM pinger is shown below. The pinger assumes a simple
      service port that, on connection, returns an integer.  If the connect and read of the integer succeds,
      the ping is marked successful.   This assumes the service endpoint is of the form
\begin{verbatim}
      CUSTOM:hostname:port
\end{verbatim}
      where ``hostname'' is the name of a host where the custom service is running, and ``port'' is
      a TCP/IP port the service uses to respond to pings.
      
      \begin{figure}[H]
\begin{verbatim}
import java.io.DataInputStream;
import java.io.InputStream;
import java.net.Socket;
import org.apache.uima.ducc.common.AServicePing;
import org.apache.uima.ducc.common.ServiceStatistics;

public class CustomPing
    extends AServicePing
{
    String host;
    String port;
    public void init(String endpoint) throws Exception {
        String[] parts = endpoint.split(":");
        host = parts[1];
        port = parts[2];
    }

    public void stop()  {  }

    public long readLong(DataInputStream dis) throws Exception {
        return Long.reverseBytes(dis.readLong());
    }

    public ServiceStatistics getStatistics() {
        ServiceStatistics stats = new ServiceStatistics(false, false,"<NA>");
        try {
            Socket sock = new Socket(host, Integer.parseInt(port));
            DataInputStream dis = new DataInputStream(sock.getInputStream());

            long stat1 = readLong(dis); long stat2 = readLong(dis); 
            long stat3 = readLong(dis); long stat4 = readLong(dis);

            stats.setAlive(true);  stats.setHealthy(true);
            stats.setInfo(  "S1[" + stat1 + "] S2[" + stat2 + "] S3[" + stat3 + "] S4[" + stat4 + "]" );
        } catch ( Throwable t) {
        	t.printStackTrace();
            stats.setInfo(t.getMessage());
        }
        return stats;        
    }
}
\end{verbatim}
        \caption{Sample UIMA-AS Service Pinger}
        \label{fig:service.custom.pinger}

      \end{figure}
      
      \subsection{Building And Testing Your Pinger}
      This section provides the information needed to use the pinger API and build a
      custom pinger. 

      \paragraph{Establish compile classpath} One DUCC jar is required in the classpath to build your pinger:
\begin{verbatim}
      uima-ducc-common.jar
\end{verbatim}      
      This provides the definition for the {\em AServicePing} and {\em ServiceStatistics} classes.

      \paragraph{Create a resistration}Next, create a service registration for the pinger.  While
      debugging, set the directive
\begin{verbatim}
      service_ping_dolog = true
\end{verbatim}
      This will log any output from  {\tt System.out.println()} to your home directory in
\begin{verbatim}
      $HOME/ducc/logs/services
\end{verbatim}

      Once the pinger is debugged you may want to turn logging off:
\begin{verbatim}
service_ping_dolog = false
\end{verbatim}
      
      A sample service registration may look something like the following:
\begin{verbatim}
     bash-3.2$ cat myping.svc

     description              = Ping-only service
     service_request_endpoint = CUSTOM:localhost:7175
     service_ping_class       = CustomPing
     service_ping_classpath   = /myhome/CustomPing.class:/home/ducc/ducc_runtime/lib/uima_ducc_common.jar
     service_ping_dolog       = true
     service_ping_timeout     = 500
     scheduling_class         = fixed
\end{verbatim}
       
      \paragraph{Register and start the pinger} Start up your custom service so the pinger has something to connect to, and then start
      the pinger.  It may be easier to debug the pinger if you start the service outside of DUCC. Once
      the pinger is working it is straighforward to integrate it into the pinger's service registration.
\begin{verbatim}
/home/ducc/ducc_runtime/bin/ducc_services --register myping.svc
/home/ducc/ducc_runtime/bin/ducc_services --start CUSTOM:localhost:7175
\end{verbatim}

      That's it!  Check the web server to make sure the service ``comes alive''.  Check your pinger's
      debugging log if it doesn't.  Once registered, you can stop and start the pinger at will using
\begin{verbatim}
  ducc_services --start <serviceid>
  ducc_services --stop <serviceid>
\end{verbatim}
     where $<$serviceid$>$ is the id returned when you registered the pinger.
