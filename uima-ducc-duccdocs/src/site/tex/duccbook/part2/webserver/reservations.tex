% 
% Licensed to the Apache Software Foundation (ASF) under one
% or more contributor license agreements.  See the NOTICE file
% distributed with this work for additional information
% regarding copyright ownership.  The ASF licenses this file
% to you under the Apache License, Version 2.0 (the
% "License"); you may not use this file except in compliance
% with the License.  You may obtain a copy of the License at
% 
%   http://www.apache.org/licenses/LICENSE-2.0
% 
% Unless required by applicable law or agreed to in writing,
% software distributed under the License is distributed on an
% "AS IS" BASIS, WITHOUT WARRANTIES OR CONDITIONS OF ANY
% KIND, either express or implied.  See the License for the
% specific language governing permissions and limitations
% under the License.
% 

\section{Reservation Page}
\label{sec:ws-reservations}

This page shows details of all reservations.  There are two types of reservations: {\em managed}
and {\em unmanaged.}.

A {\em managed reservation} is a reservation whose process is fully managed by DUCC.  This process
is any arbitrary process and is submitted with the
\hyperref[sec:cli.ducc-process-submit]{ducc\_process\_submit} CLI.  The lifetime of the reservation
starts at the time DUCC assigns a unique ID, and ends when the process terminates for any reason.

An {\em unmanaged reservation} is essentially a sandbox for the user.  DUCC starts no processes
in the reservation and manages none of the processes which run on that node.  The lifetime of the
reservation starts at the time DUCC assigns a unique ID, and ends when the submitter or system
administrator cancels it.  {\em Managed reservations} can potentially last an indefinite
period of time.

The Reservations page contains the following columns: 
\begin{description}

\item[Id] \hfill \\
  This is the unique DUCC numeric id of the reservation as assigned when the reservation is made.
  If this is a {\em managed} reservation, the ID is hyperlinked to a
  \hyperref[sec:ws-managed-reservation-details]{Managed Reservation Details} page with extended
  details on the process running in the reservation.

\item[Start] \hfill \\
  This is the time the reservation was mode.
  
\item[End] \hfill \\
  This is the time the reservation was canceled or otherwise ended.
  
\item[User] \hfill \\
  This is the userid if the person who made the reservation.
  
\item[Class] \hfill \\
  This is the scheduling class used to schedule the reservation.
  
\item[Type] \hfill \\
  This is the reservation type, {\em managed} or {\em unmanaged}, as described 
  \hyperref[sec:ws-reservations]{above}.

\item[State] \hfill \\
  % 1. org.apache.uima.ducc.transport.event.common.IDuccState
  This is the status of the reservation. Values include: Received - Reservation
  has been vetted, persisted, and assigned unique Id.
  \begin{description}
  \item[Assigned] - The reservation is active. 
  \item[Completed] - The reservation has been terminated.
  \item[Received] - The Reservation has been vetted, persisted, and assigned a unique ID.
  \item[WaitingForResources] - The reservation is waiting for the Resource Manager to find and 
    schedule resources. 
  \end{description}

\item[Reason] \hfill \\

  % 2. org.apache.uima.ducc.transport.event.common.IDuccCompletionType

  If a reservation is not active, this shows the reason.  Note that for
  {\em unmanaged reservations}, even if the user has processes running in the
  reservation, DUCC does NOT attempt to terminate those processes (hence, ``unmanaged''.)

  For {\em managed reservations}, DUCC does terminate the associated process.

  \begin{description}
  \item[CanceledBySystem] - In the case of the special JobDriver reservation, this is
    canceled by DUCC and reestablished on reboot; hence the state is a result of DUCC
    having been restarted.

    In all other cases, it is a result of DUCC being restarted {\em COLD}.  When
    DUCC is started {\em COLD}, all previous reservations are canceled.  (When DUCC
    is started {\em WARM}, the default, previous reservations are preserved.)
  \item[CanceledByAdmin] - The DUCC administrator released the reservation. 
  \item[CanceledByUser] - The reservation owner released the reservation. 
  \item[ResourcesUnavailable] - The Resource Manager was unable to find free or freeable resources 
    match the resource request. 
  \item[ProgramExit] - The reservation is a {\em managed} reservation and the associated
    process has exited.
  \end{description}

\item[Allocation] \hfill \\
  This is the number of resources (shares for FIXED policy reservations, processes for
  RESERVE policy reservations) that are allocated.

\item[UserProcesses] This is the number of processes owned by the user running in all
  shares of the reservation.  
  
  Note that even for {\em unmanaged} reservations, the DUCC agent tracks processes owned
  by the user and reports on them.  This allows better identification and management of
  abandoned reservations.

\item[Size] \hfill \\
  The memory size in GB of the each allocated unit.  This is the amount of memory that
  was {\em requested}.  In the case of RESERVE policy reservations, that actual memory
  of the reserved machine may be greater.
  
\item[Host Names] \hfill \\
  The node names of the machines where the resources are allocated.
  
\item[Description] \hfill \\
  This is the description string from the --description string from submit.
\end{description}

