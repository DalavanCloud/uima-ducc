% 
% Licensed to the Apache Software Foundation (ASF) under one
% or more contributor license agreements.  See the NOTICE file
% distributed with this work for additional information
% regarding copyright ownership.  The ASF licenses this file
% to you under the Apache License, Version 2.0 (the
% "License"); you may not use this file except in compliance
% with the License.  You may obtain a copy of the License at
% 
%   http://www.apache.org/licenses/LICENSE-2.0
% 
% Unless required by applicable law or agreed to in writing,
% software distributed under the License is distributed on an
% "AS IS" BASIS, WITHOUT WARRANTIES OR CONDITIONS OF ANY
% KIND, either express or implied.  See the License for the
% specific language governing permissions and limitations
% under the License.
% 

\section{System  Details Page}
\label{sec:system-details}

This page shows information relating to the DUCC System itself:
\begin{description}
  \item[Administration]This displays system administrators and implements
    the interface to various administrative controls.
  \item[Classes] This shows the current system's scheduling class definitions.
  \item[Daemons] This shows the status of all DUCC processes.
  \item[DuccBook] This is a link to the book you are reading.
  \item[Machines] This shows details of all the machines in the DUCC cluster.
\end{description}

\subsection{Administration}

   This page has two tabs:
   \begin{description}   
     \item[Administrators] This shows the user-ids that are authorized to administer
       DUCC.  In addition to executing the ``Control'' functions described below,
       administrators may cancel any job, reservation, or service, and may modify
       services they do not own.  

       In order to perform administrative functions, the following must be satisfied:
       \begin{enumerate}
         \item The user is logged-in to the web server.
         \item The user is a registered administrator.
         \item The user has set the role as ``administrator'' in the DUCC Preferences
           page.  This is a safeguard so that administrators who are also users
           are less likely to inadvertently affect other people's jobs.
       \end{enumerate}
     \item[Control] Currently DUCC supports a single administrative control function
       via the web server: Stop new job submissions and re-enable them.  If submissions
       are blocked, all existing work runs normally, but no new work is accepted.
     \end{description}


\subsection{Classes}
This page shows the definitions of the DUCC scheduling classes.  The scheduling classes are
discussed in more detail in the \hyperref[sec:rm.job-classes]{Resource Manager} section.

\subsection{Daemons}
\label{sec:system-details.daemons}

This page shows the current state of all DUCC processes.  By default, only the administrative
processes, Orchestrator, ProcessManager, ResourceManager, ServiceManager, and Webserver are
shown.  A button in the upper left of the page titled ``Show Agents'' enables display of
the status of all the DUCC agents as well. (Agents are suppressed by default because the
page is expensive to render for large systems.)

The columns shown on this page include

   \begin{description}
      \item[Status] \hfill \\
        This indicates whether the daemon is running and broadcasting state {\em up},
        or not {\em down}.  
        
        All DUCC daemons broadcast a heartbeat containing process state.  If the Status
        is {\em down}, either the daemon is not functioning, or something is preventing
        state from reaching the web server via DUCC's ActiveMQ instance.

      \item[Daemon Name] \hfill \\
        This is the name of the process.

      \item[Boot Time] \hfill \\ 
        This shows the date and time of the latest boot of the specific process.
          
      \item[Host IP] \hfill \\ 
        This is the IP address of the processor where the process is running.

      \item[Host Name] \hfill \\ 
        This shows the hostname of the processor where the process is running.

      \item[PID] \hfill \\ 
        This is the Unix processid of the DUCC process.


      \item[Publication Size (last)] \hfill \\ 
        This shows the size of the most recent state publication of the process, in bytes.

      \item[Publication Size (max)] \hfill \\ 
        This shows the size of the largest state publication of the process, in bytes.

      \item[Heartbeat (last)] \hfill \\ 
        This shows the number of seconds since the last state publication for the process. 
         Large numbers here indicate potential cluster or DUCC problems.

      \item[Heartbeat (max)] \hfill \\ 
        This shows the longest delay since a state publication for the process was received
        at the web server.  Large numbers here indicate potential cluster or DUCC problems.

      \item[Heartbeat (max) TOD] \hfill \\ 
        This shows the time the longest delay of a state publication occurred.

      \item[JConsole URL] \hfill \\ 
        This is the jconsole URL for the process.

   \end{description}
      
\subsection{Machines}

This page shows the states of all the machines in the DUCC cluster.

The columns shown on this page include

   \begin{description}
      \item[Status] \hfill \\
        This shows the current state of a machine.  Values include:
        \begin{description}
          \item[defined] The node is in the DUCC
            \hyperref[sec:admin-ducc.nodes]{nodes file}, but no DUCC process has been
            started there, or else there is a communication problem and
            the state messages are not being delivered.
            \item[up] The node has a DUCC Agent process running on it and the
              web server is receiving regular heartbeat packets from it.
            \item[down] The node had a healthy DUCC Agent on it at some point
              in the past (since the last DUCC boot), but the web server has stopped
              receiving heartbeats from it. 

              The agent may have been manually shut down, may have crashed, or there
              may be a communication problem.

              Additionally, very heavy loads from jobs running the the node can cause
              the DUCC Agents heartbeats to be delayed.
        \end{description}


      \item[IP] \hfill \\
        This is the IP address of the node.

      \item[Name] \hfill \\
        This is the hostname of the node.

      \item[Reserve(GB) size] \hfill \\
        This is the largest reservation that can be made on this node.

        This is usually somewhat less than the physical memory size because it is 
        rounded down to the nearest \hyperref[chap:rm]{share quantum}.  The purpose of this
        column is to assist users in requesting the right size for full machine 
        reservations.

      \item[Memory(GB) total] \hfill \\
        This is the amount of memory, in GB, as reported by each machine.
        
        Usually the amount will be slightly less than the installed memory.  This is because
        a small bit of memory is usually reserved by the hardware for its own purposes.  For 
        example, a machine with 48GB of installed memory may report only 47GB available.

      \item[Swap(GB) in use] \hfill \\
        This is the total size in-use swap data.  DUCC shows any value greater than 0 in
        red as swapping can very significantly slow applications.  However, swap use does
        not always mean there is a performance problem.  This is flagged by DUCC simply
        as an alert of a potential problem

      \item[Alien PIDs] \hfill \\
        This shows the number of processes not owned by DUCC, the operating system, or
        jobs scheduled on each node.  The Unix Process IDS of these processes is displayed
        in a hover.

        DUCC preconfigures many of the standard operating 
        \hyperref[itm:props-rogue.process]{system process} and 
        \hyperref[itm:props-rogue.user]{userids}.  This list may be updated by each
        installation.

        A common cause of alien PIDs is errant process run in unmanaged reservations.  A
        user may reserve a machine for use as a sandbox.  If the reservation is released
        without properly terminating all the processes, they may linger.  When ducc 
        schedules the node for other purposes, significant performance penalties may be
        paid due to competition between the legitimately scheduled work and the leftover
        ``alien'' processes.  The purpose of this column is to bring attention to this situation.

      \item[Shares (total)] \hfill \\
        This shows the total number of scheduling share supported on this node.

      \item[Shares(in use)] \hfill \\
        This shows the total number of scheduling share in use on the node.

      \item[Heartbeat(last)] \hfill \\
        This shows the number of seconds since the last agent heartbeat from this machine.

      \end{description}
      
