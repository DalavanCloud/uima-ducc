% 
% Licensed to the Apache Software Foundation (ASF) under one
% or more contributor license agreements.  See the NOTICE file
% distributed with this work for additional information
% regarding copyright ownership.  The ASF licenses this file
% to you under the Apache License, Version 2.0 (the
% "License"); you may not use this file except in compliance
% with the License.  You may obtain a copy of the License at
% 
%   http://www.apache.org/licenses/LICENSE-2.0
% 
% Unless required by applicable law or agreed to in writing,
% software distributed under the License is distributed on an
% "AS IS" BASIS, WITHOUT WARRANTIES OR CONDITIONS OF ANY
% KIND, either express or implied.  See the License for the
% specific language governing permissions and limitations
% under the License.
% 
  
    \section{Job Details Page}
    \label{sec:ws-job-details}

    This page shows details of all the processes that run in support of a job. 
    The information is divided among four tabs:
    \begin{description}
      \item[Processes] This tab contains details on all the processes for the job, both
        active, and defunct.
      \item[Work Items] This tab shows details for each individual work-item in the job.
      \item[Performance] This tab shows a performance break-down of all the UIMA analytics
        in the job.
      \item[Specification] This tab shows the job specification for the job.
      \end{description}
      
    \subsection{Processes}
    \label{sec:ws-processes}
    The processes page contains the following columns:
    
    \begin{description}

        \item[Id] \hfill \\
          This is the DUCC-assigned numeric id of the process (not the Operating System's
          processid). Process 0 is always the Job Driver.          

        \item[Log] \hfill \\
          This is the log name for the process. It is hyperlinked to the log itself.

        \item[Size] \hfill \\
          This is the size of the log in MB. If you find you have trouble viewing the log
          from the Web Server it could be because it is too big to view in the server and needs to
          be read by some other means than the Web Server.  (It is not currently paged in by 
          the Web Server, it is read in full.)

        \item[Hostname] \hfill \\
          This is the name of the node where the process ran.

        \item[PID] \hfill \\
          This is the Unix process ID (PID) of the process.

        \item[State Scheduler] \hfill \\
          % The information comes from here:
          % State Scheduler: org.apache.uima.ducc.transport.event.common.IResourceState.ResourceState

          This shows the Resource Manager state of the job. It is one of:
          \begin{description}
              \item[Allocated] - The node is currently allocated for this job by the RM.
              \item[Deallocated] - The resource manager has deallocated the shares for the job on
                this node.
          \end{description}

        \item[Reason Scheduler or extraordinary status] \hfill \\
          \phantomsection\label{itm:job-details-sched}


          % The information comes from here:
          % Reason Scheduler: org.apache.uima.ducc.transport.event.common.IResourceState.ProcessDeallocationType
          This column provides a reason for the scheduler state, when the scheduler state is other than ``Allocated''. 
          These all have ``hovers'' that provide more information
          if it is available.

            \begin{description}          
                \item[AutonomousStop] - The process terminated unexpectedly of its own accord ("crashed", or
                  simply exited.) 

                \item[Exception] - The process is terminated by the JD exception handler. 

                \item[Failed] - The process is terminated by the Agent because the JP wrapper was able to detect and 
                  communicate a fatal condition (Exception) in the pipeline.. 
                  
                \item[FailedInitialization] - The process is terminated because the UIMA initialization step failed. 
                  
                \item[Forced] - The node is preempted by RM for other work because of fair share. 
                  
                \item[JobCanceled] - The job was canceled by the user or a system administrator. 
                  
                \item[JobCompleted] - The process is canceled because of DUCC restart. 
                  
                \item[JobFailure] - The job failure limit is exceeded, causing the job to be canceled by the JD.                    
                  
                \item[InitializationTimeout] - The UIMA initialization phase exceeded the configured timeout. 
                  
                \item[Killed] - The agent terminated the process for some reason. The ``Reason Agent'' field
                  should have more details in this case.
          
                \item[Stopped]	- The process was terminated by the Agent for some reason.  The hover should
                  contain more information.
                          
                \item[Voluntary] - The job is winding down, there's no more work for this node, so it stops. 
                  
                \item[Unknown] - None of the above. This is an exceptional condition, sometimes an
                  internal DUCC error. Check the JP and JD logs for possible causes..
            \end{description}

          \item[State Agent] \hfill \\
          \phantomsection\label{itm:job-details-state}

          % This state comes from here:
          % State Agent: org.apache.uima.ducc.transport.event.common.IProcessState.ProcessState
            This shows the DUCC Agent's view of the state of the process.
            \begin{description}
               \item[Starting] The DUCC process manager as issued a request to the assigned to
                 start the process.
               \item[Initializing] The process is initializing.  Usually this means the UIMA analytic
                 pipeline (Job Process) is executing it's initialization method.
              \item[Running] The Job Process has completed the initialization phase and is ready for, 
                or actively executing work.
              \item[Stopped] The DUCC Agent reports the process is stopped and (and has exited).
              \item[Failed] The DUCC Agent reports the process failed with errors.  This usually
                means that UIMA-AS has detected exceptions in the pipeline and reported them
                to the Job Driver for logging.
              \item[FailedInitialization] The process died during the UIMA initialization phase.
              \item[InitializationTimeout] The process exceeded the site's limit for time spent
                in UIMA initialization.
              \item[Killed] The DUCC Agent killed the process for some reason.  There are
                three rosins for this:
                \begin{enumerate}
                  \item The Job Processes failed to initialize,
                  \item The Job Process timed out during initialization,
                  \item The process Exocet's its allowed swap.
                \end{enumerate}
              \item[Abandoned] It is possible to cancel a specific process of a job.  Usually
                this is because it became ``stuck'' because of hardware failure.  It a process
                is killed in \hyperref[sec:cli.ducc-cancel]{this way}, the state is recoreded as {\em Abandonded}.
            \end{description}
            
          \item[Reason Agent] \hfill \\
          \phantomsection\label{itm:job-details-agent}

          This shows extended reason information if a process exited other than having run out
          of work to do.

            \begin{description}
              \item[AgentTimedOutWatingForORState] The DUCC Agent is expecting a state update
                from the DUCC Orchestrator.  Timer on this wait has expired.  This usually 
                indicates an infrastructure or communication problem.
              \item[Croaked] The process exited for no good or clear reason, it simply vanished.
              \item[Deallocated] WHAT IS THIS?
              \item[ExceededShareSize] The process exceeded it's declared memory size.
              \item[ExceededSwapThreshold] The process exceeded the configured swap threshold.
              \item[FailedInitialization] The process was terminated because the UIMA 
                initialization step failed.
              \item[InitializationTimeout] The process was terminated because the UIMA initialization
                step took too long.
              \item[JPHasNoActiveJob] This is set when an agent looses connectivity while its
                JPs are running. The job finishes (stopped or killed). The agent regains
                connectivity. The OR publish no longer includes the job but the agent still has
                processes running for that job. The agent kills ghost processes with the reason:
                JPHasNoActiveJob.
              \item[LowSwapSpace] The process was terminated because the system is about to run
                out of swap space.  This is a preemptive measure taken by DUCC to avoid exhaustion
                of swap, to effect orderly eviction of the job before the operating system starts
                its own reaping procedures.
              \item[AdministratorInitiated] The process was canceled by an administrator.
              \item[UserInitated] The process was canceled by the owning user.
            \end{description}
            
          \item[Time Init] \hfill \\
            This is the clock time this process spent in initialization.
            
          \item[Time Run] \hfill \\
            This is the clock time this process spent in executing, not including
            initialization.
            
          \item[Time GC] \hfill \\
            This is amount of time spent in Java Garbage Collection for the process.
            
          \item[Count GC] \hfill \\
            This is the number of garbage collections performed by the process.
            
          \item[Pgin] \hfill \\
            This is the number of page-in events on behalf of the process.

          \item[Swap] \hfill \\
            This is the amount of swap space on the machine being consumed by the process.

          \item[\%GC] \hfill \\
            Percentage of time spent in garbage collections by this process, relative to total of
            initialization + run times.
            
          \item[\%CPU] \hfill \\
            Currant CPU percent consumed by the process.  This will be $>$ 100\% on 
            multi-core systems if more than one core is being used.  Each core contributes
            up to 100\% CPU, so, for example, on a 16-core machine, this can be as high
            as 1600\%.
            
          \item[RSS] \hfill \\
            The amount of real memory being consumed by the process (Resident Storage Size)
            
          \item[Time Avg] \hfill \\
            This is the average time in seconds spent per work item in the process.
            
          \item[Time max] \hfill \\
            This is the minimum time in seconds spent per work item in the process.
            
          \item[Time min] \hfill \\
            This is the minimum time in seconds spent per work item in the process.
            
          \item[Done] \hfill \\
            This is the number of work items processed in this process.
            
          \item[Error] \hfill \\
            This is the number of exceptions processing work items in this process.
            
          \item[Retry] \hfill \\
            This is the number of work items that were retried in this process for any reason, excluding
            preemption.
            
          \item[Preempt] \hfill \\
            This is the number of work items that were preempted from this process, if
            fair-share caused preemption.
            
          \item[JConsole URL] \hfill \\
            This is a URL that can be used to connect via JMX to the processes, e.g. via
            jconsole.

      \end{description}

   \subsection{Work Items}
   \label{sec:ws-work-items}
   This tab provides details for each individual work item.  Columns include:

   % The data comes from here: org.apache.uima.ducc.common.jd.files.IWorkItemState.State    
   \begin{description}
     \item[SeqNo]  \hfill \\
       This is the sequence work items are fetched from the Collection Reader's
       getNext() method by the DUCC Job Driver.
     \item[Id]  \hfill \\
       This is the name of the work item.
     \item[Status]  \hfill \\
       The is the current state of the work item.  
       States include:
       \begin{description}
         \item[ended] The work item is complete.
         \item[error] The work item ended with errors.
         \item[lost] The work item was queued to ActiveMQ but never dequeued by
         any Job Process.
         \item[operating] The work item is current being executed.
         \item[retry] The work item is being retried.
         \item[start] The work item has been picked up for execution and DUCC is waiting
           for confirmation that it is running.
         \item[queued] The work item has been queued to ActiveMQ but not picked up by any
           Job Process yet.
       \end{description}
       If a work item has not yet been retrieved from the Collect Reader it does not show
       on this page.
     \item[Queuing Time (sec)]  \hfill \\
       The time spent in ActiveMQ after being queued, and before
       being picked up by a Job Process.
     \item[Processing Time (sec)]  \hfill \\
       The time spent processing the work item.
     \item[Node (IP)]  \hfill \\
       The node IP where the work item was processed.
     \item[Node (Name]  \hfill \\
       The node name where the work item was processed.
     \item[PID]  \hfill \\
       The Unix Process Id that the work item was processed in.
   \end{description}
   

   \subsection{Performance}
   \label{sec:performance}
   This tab shows performance summaries of all the pipeline components.  The statistics
   are aggregated over all instances of each component in each process of the job.
   
   \begin{description}
     \item[Name]  \hfill \\
       The short name of the analytic.  The full name is shown in the command-line
       tool \hyperref[sec:cli.ducc-perf-stats]{ducc\_perf\_stats}
     \item[Total]  \hfill \\
       This is the total time in days, hours, minutes, and seconds taken by each
       component of the pipeline.
     \item[\% of Total]  \hfill \\
       This is the percent of the total usage consumed by this analytic.
     \item[Avg]  \hfill \\
       This is the average time spent by all the instances of the analytic.
     \item[Min]  \hfill \\
       This is the minimum time spent by any instance of the analytic.
     \item[Max]  \hfill \\
       This is the maximum time spent by any instance of the analytic.
   \end{description}
   
   \subsection{Specification}
   This tab shows the full job specification in the form of a Java Properties
   file.  This will include all the parameters specified by the user, plus those
   filled in by DUCC.
