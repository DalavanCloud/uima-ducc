% 
% Licensed to the Apache Software Foundation (ASF) under one
% or more contributor license agreements.  See the NOTICE file
% distributed with this work for additional information
% regarding copyright ownership.  The ASF licenses this file
% to you under the Apache License, Version 2.0 (the
% "License"); you may not use this file except in compliance
% with the License.  You may obtain a copy of the License at
% 
%   http://www.apache.org/licenses/LICENSE-2.0
% 
% Unless required by applicable law or agreed to in writing,
% software distributed under the License is distributed on an
% "AS IS" BASIS, WITHOUT WARRANTIES OR CONDITIONS OF ANY
% KIND, either express or implied.  See the License for the
% specific language governing permissions and limitations
% under the License.
% 
% Create well-known link to this spot for HTML version
\ifpdf
\else
\HCode{<a name='DUCC_ERROR_HANDLER'></a>}
\fi
\chapter{Job Error Handler}
\label{chap:job-error-handler}

\begin{sloppypar}
\paragraph {Overview} The {\em ErrorHandler} allows for the per job customized handling of runtime anomalies.

\paragraph {Operation} The Job Driver comes with a built-in {\em ErrorHandler}.  Its purpose is to 
instruct the Job Driver on what action(s) to take when a work item error is encountered.

The {\em ErrorHandler} implements {\em org.apache.uima.ducc.IErrorHandler}.

\begin{verbatim}
public interface IErrorHandler {
	public void initialize(String initializationData);
	public IErrorHandlerDirective handle(String serializedCAS, Object userException);
}

public interface IErrorHandlerDirective {
	public boolean isKillJob();
	public boolean isKillProcess();
	public boolean isKillWorkItem();
}
\end{verbatim}

By default, the {\em ErrorHandler} returned directive:
\begin{enumerate}
\item returns isKillJob == false, unless the number of work items errors exceeds 15 for the Job
\item returns isKillProcess == true
\item returns isKillWorkItem == true
\end{enumerate}

\paragraph {Programmability} The Job Driver built-in {\em ErrorHandler} behavior can be modified
according to the {\em driver\_jvm\_args} in the Job Specification:

\begin{description}
\item[-DJobDriverErrorHandlerMaximumNumberOfTimeoutRetrysPerWorkItem=N], where N is the maximum
number of timeout retrys for each work item before that work item is considered in error.
\end{description}

\paragraph {Replacement} The {\em ErrorHandler} can be replaced.  The steps necessary are:
\begin{enumerate}
\item Create a new ErrorHandler.class that implements {\em org.apache.uima.ducc.IErrorHandler}, which is located in the uima-ducc-user.jar.
\item Put your replacement class in a jar file and modify ducc.properties file to include your jar file as part of ducc.local.jars variable.
\end{enumerate}

\end{sloppypar}
