% 
% Licensed to the Apache Software Foundation (ASF) under one
% or more contributor license agreements.  See the NOTICE file
% distributed with this work for additional information
% regarding copyright ownership.  The ASF licenses this file
% to you under the Apache License, Version 2.0 (the
% "License"); you may not use this file except in compliance
% with the License.  You may obtain a copy of the License at
% 
%   http://www.apache.org/licenses/LICENSE-2.0
% 
% Unless required by applicable law or agreed to in writing,
% software distributed under the License is distributed on an
% "AS IS" BASIS, WITHOUT WARRANTIES OR CONDITIONS OF ANY
% KIND, either express or implied.  See the License for the
% specific language governing permissions and limitations
% under the License.
% 
\section{Service Details Page}
\label{sec:ws-service-details}

This page shows details of the processes which implement. 

The information is divided between four tabs:

   \begin{description}
       \item[Deployments] This tab contains details on all the processes implementing
         the service, if any.
       \item[Registry] This tab shows the registration information for the service.
       \item[Files] This tab shows the files in the log directory. 
       \item[History] This tab contains details on all the completed processes implementing the service, if any.  
   \end{description}  

   \subsection{Deployments}
   \label{sec:ws-services-processes}

   The deployments page contains the following columns:
   \begin{description}
      \item[Id] \hfill \\
        This is the {\DUCC}-assigned numeric id of the process.  This format of this
        id is two numbers:
\begin{verbatim}
    RESID.SHAREID
\end{verbatim}
        Here, the {\em RESID} is the Orchestrator assigned instance ID.  The {\em SHAREID} is the 
        instance ID assigned by the Resource Manager.  Together these form a unique
        ID for each process that runs in the service.
               
      \item[State] \hfill \\
        The state of this service instance.
               
      \item[Services] \hfill \\
        The current state of service dependencies.
                                
      \item[Log] \hfill \\
        This is the log name for the process. It is hyperlinked to the log itself.
        
      \item[Log Size] \hfill \\
        This is the size of the log in MB. If you find you have trouble viewing the log
        from the web server it could be because it is too big to view in the browser.
        
      \item[Host Name] \hfill \\
        This is the name of the node where the process is running (or ran).
        
      \item[PID] \hfill \\
        This is the Unix process ID (PID) of the process.
       
      \item[Memory] \hfill \\
        The service process actual memory size (GB).
                
      \item[State Scheduler] \hfill \\
        This shows the Resource Manager state of the service instance. It is one of:
        
        \begin{description}
            \item[Allocated] - The node is still allocated for this service instance by the RM.
            \item[Deallocated] - The resource manager has deallocated the resources for the service instance on
              this node.
        \end{description}
        
      \item[Reason Scheduler or Extraordinary Status] \hfill \\
        These are the same as for the \hyperref[itm:job-details-sched]{job details.}

      \item[State Agent] \hfill \\
        These are the same as for the \hyperref[itm:job-details-state]{job details.}

      \item[Reason Agent] \hfill \\
        These are the same as for the \hyperref[itm:job-details-agent]{job details.}

      \item[Exit] \hfill \\
        The process exit code or signal.

      \item[Time Init] \hfill \\
        Most services are UIMA-AS services and therefore have an {\em initialization} phase
        to their lifetimes.  This field shows the time spent in that phase.

      \item[Time Run] \hfill \\
        The current duration of the instance, or total duration if it has 
        terminated.
        
      \item[Time GC] \hfill \\
        This is amount of time spent in Java Garbage Collection for the process.

      \item[Pgin] \hfill \\
        This is the number of page-in events on behalf of the process.
        
      \item[Swap] \hfill \\
        This is the amount of swap space on the machine being consumed by the process.
        
      \item[\%CPU] \hfill \\
        Current CPU percent consumed by the process.  This will be $>$ 100\% on 
        multi-core systems if more than one core is being used.  Each core contributes
        up to 100\% CPU, so, for example, on a 16-core machine, this can be as high
        as 1600\%.

      \item[RSS] \hfill \\
        The amount of real memory being consumed by the process (Resident Storage Size)

      \item[JConsole URL] \hfill \\
        This is a URL that can be used to connect via JMX to the processes, e.g. via
        jconsole.

   \end{description}

   \subsection{Registry}
   \label{sec:ws-managed-reservation-specification}
   This tab shows the full service specification in the form of a Java Properties
   file.  This will include all the parameters specified by the user, plus those
   filled in by {\DUCC}.
        
   The registry for a Service contains two types of entries:
   \begin{enumerate}
     \item Service specification properties, prefixed with ``svc''. These comprise
       the service specification that the Service Manager submits on behalf of
       a user in order to start registered services.
     \item Meta properties, prefixed with ``meta''.  This is the Service Manager's state
       record for the service as it is running.  In addition to state it contains
       properties required for service registration that are not used for
       service submission.
   \end{enumerate}
           
   \subsection{Files}
   This tab shows the files in the log directory.
   
           
   \subsection{History}
   This tab shows the completed service instances.
   
