% 
% Licensed to the Apache Software Foundation (ASF) under one
% or more contributor license agreements.  See the NOTICE file
% distributed with this work for additional information
% regarding copyright ownership.  The ASF licenses this file
% to you under the Apache License, Version 2.0 (the
% "License"); you may not use this file except in compliance
% with the License.  You may obtain a copy of the License at
% 
%   http://www.apache.org/licenses/LICENSE-2.0
% 
% Unless required by applicable law or agreed to in writing,
% software distributed under the License is distributed on an
% "AS IS" BASIS, WITHOUT WARRANTIES OR CONDITIONS OF ANY
% KIND, either express or implied.  See the License for the
% specific language governing permissions and limitations
% under the License.
% 
% Create well-known link to this spot for HTML version
\ifpdf
\else
\HCode{<a name='DUCC_State_Directory'></a>}
\fi
\chapter{State Directory}
\label{chap:state-directory}
    \section{Overview}
    
 	The {\em state} directory located under {\em \duccruntime} comprises 
 	persistent data used by {\DUCC} for continuity of operations across lifetimes.

	\section{Backup}
	\label{sec:state-directory.backup}
	
    The \hyperref[subsec:admin.db-tool]{\em db\_tool}
    should be employed to regularly backup the database portion of the
    {\em state} directory for later restoration in the event of disk
    catastrophe.  
    Regular backup can be accomplished via {\em crontab}, an example of which can
    be found in {\em \duccruntime/admin/cron}.
    
	\section{Sub-directories}
	
	Sub-directories under the {\em state} directory are described below.
	
	\subsection{agents}
	
	Each agent writes a unique named properties file into this 
	directory when it starts-up, replacing the prior one if any.
	An example follows below.

	{\em uima-ducc-demo-1-boot.properties}
	\begin{verbatim}
	#
	#Wed Oct 12 02:53:15 UTC 2016
	daemonName=uima-ducc-demo-1
	nodeIpAddress=10.20.2.187
	pid=3712
	jmxUrl=service\:jmx\:rmi\:///jndi/rmi\://uima-ducc-vm2\:11000/jmxrmi
	nodeName=uima-ducc-vm2.apache.org
	bootTime=2016.10.12 02\:53\:15 Wed
	\end{verbatim}
	
	The {\DUCC} Web Server employs this information for display
	on the System.Daemons page.
	
	\subsection{daemons}
	
	Each head-node daemon writes a unique named properties file into this
	directory when it starts-up, replacing the prior one if any.
	An example follows below.
	
	{\em Orchestrator-boot.properties}
	\begin{verbatim}
	#
	#Wed Oct 12 02:53:51 UTC 2016
	daemonName=Orchestrator
	nodeIpAddress=10.20.2.187
	pid=4577
	jmxUrl=service\:jmx\:rmi\:///jndi/rmi\://uima-ducc-vm2\:11014/jmxrmi
	nodeName=uima-ducc-vm2.apache.org
	bootTime=2016.10.12 02\:53\:51 Wed
	\end{verbatim}
	
	The {\DUCC} Web Server employs this information for display 
	on the System.Daemons page.
	
	\subsection{database}
	
	This directory comprises the sub-directories and files for running
	the {\DUCC} Cassandra database.
	See also \hyperref[sec:state-directory.backup]{\em Backup}.
	
	Several {\DUCC} head node daemons write a plethora of information
	pertaining to Jobs, Reservations, Services and System.Machines.
	The {\DUCC} Web Server employs the database to gather information for display
	on the corresponding web pages.
	
	\section{Files}
	
	Files under the {\em state} directory are described below.
	
	\subsection{cassandra.pid}
	
	This file is written each time the {\DUCC} Cassandra database starts-up.
	The contents represent the Linux process identifier (PID) of the
	live {\DUCC} Cassandra database on the head node.
	An example follows below.
	
	{\em cassandra.pid}
	\begin{verbatim}
	2664
	\end{verbatim}
	
	\subsection{duccling.version}
	
	This file is written each time one or more {\DUCC} daemons start-up.
	The contents represent the version information for the 
	privileged {\em ducc\_ling} command.
	An example follows below.
	
	{\em duccling.version}
	\begin{verbatim}
	050 ducc_ling Version 2.1.0 compiled Aug 16 2016 at 17:10:55
	\end{verbatim}
	
	\subsection{orchestrator.properties}
	
	This file is written each time the {\DUCC} Orchestrator assigns 
	a sequence number to a unit of work (Job, Reservation, Managed Reservation,
	or Service Instance).
	An example follows below.
	
	{\em orchestrator.properties}
	\begin{verbatim}
	#Tue Oct 18 20:33:16 UTC 2016
	seqno=1187
	\end{verbatim}
	
	\subsection{orchestrator-state.json}
	
	This file is written each time the {\DUCC} Orchestrator publishes
	live state information to its peer daemons.  
	It comprises publication sequence numbers.
	An example follows below.
	
	{\em orchestrator-state.json}
	\begin{verbatim}
	{"sequenceNumberState":545390,"sequenceNumberStateAbbreviated":4}
	\end{verbatim}
	
	\subsection{sm.properties}

	This file is written each time the {\DUCC} Services Manager assigns 
	a sequence number to a service registration.
	An example follows below.
	
	{\em sm.properties}
	\begin{verbatim}
	#Service Manager Properties
	#Tue Aug 16 17:15:49 UTC 2016
	service.seqno=5
	\end{verbatim}
	