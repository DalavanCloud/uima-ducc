% 
% Licensed to the Apache Software Foundation (ASF) under one
% or more contributor license agreements.  See the NOTICE file
% distributed with this work for additional information
% regarding copyright ownership.  The ASF licenses this file
% to you under the Apache License, Version 2.0 (the
% "License"); you may not use this file except in compliance
% with the License.  You may obtain a copy of the License at
% 
%   http://www.apache.org/licenses/LICENSE-2.0
% 
% Unless required by applicable law or agreed to in writing,
% software distributed under the License is distributed on an
% "AS IS" BASIS, WITHOUT WARRANTIES OR CONDITIONS OF ANY
% KIND, either express or implied.  See the License for the
% specific language governing permissions and limitations
% under the License.
% 
\section{WebServer Authentication}
\label{sec:WebServer Authentication}

    By default, DUCC is configured such that there is effectively no
    authentication enforcement by the WebServer. No password entry is permitted
    on the Login panel and any userid specified is accepted whether it exists or
    not.
    
    To enable your own authentication measures, you should perform the following
    steps:
    
    \begin{enumerate}
      \item Author an authentication manager Java class implementing interface
      \begin{verbatim}
org.apache.uima.ducc.common.authentication.IAuthenticationManager
      \end{verbatim}
      \item Create an authentication jar file comprising the
      authentication manager Java class
      \item Install your authentication jar file and any dependency jar files
      into your DUCC's lib folder
      \item Update your ducc.properties file with authentication class name
      and jar file name(s) information
      \item Create a ducc.administrators file
      \end{enumerate}
      
    Note: When a user clicks on the WebServer Login link, the login dialog is
    shown. On that dialog panel is shown the \mbox{authenticator: {\em
    version}}, which is supplied by your authentication manager implementation's {\em
    \mbox{getVersion()}} method. Also shown are boxes for userid and password
    entry. If your authentication manager implemenation's {\em \mbox{isPasswordChecked()}}
    method returns true then the password box will accept input, otherwise it will be
    disabled.
    
\subsection{Example Implementation}
    
    Shown below is an example implementation which can be used as a template
    for coding protection by means of interfacing with your site's security
    measures.
    
    In this example, the SiteSecurity Java class is presumed to be existing
    and available code at your installation.
    
    \begin{verbatim}
package org.apache.uima.ducc.example.authentication.module;

import org.apache.uima.ducc.common.authentication.AuthenticationResult;
import org.apache.uima.ducc.common.authentication.IAuthenticationManager;
import org.apache.uima.ducc.common.authentication.IAuthenticationResult;
import org.apache.uima.ducc.example.authentication.site.SiteSecurity;

public class AuthenticationManager implements IAuthenticationManager {

    private final String version = "example 1.0";
    
    @Override
    public String getVersion() {
        return version;
    }

    @Override
    public boolean isPasswordChecked() {
        return true;
    }

    @Override
    public IAuthenticationResult isAuthenticate(String userid, String domain,
            String password) {
        IAuthenticationResult authenticationResult = new AuthenticationResult();
        authenticationResult.setFailure();
        try {
            if(SiteSecurity.isAuthenticUser(userid, domain, password)) {
                authenticationResult.setSuccess();
            }
        }
        catch(Exception e) {
            //TODO
        }
        return authenticationResult;
    }

    @Override
    public IAuthenticationResult isGroupMember(String userid, String domain,
            Role role) {
        IAuthenticationResult authenticationResult = new AuthenticationResult();
        authenticationResult.setFailure();
        try {
            if(SiteSecurity.isAuthenticRole(userid, domain, role.toString())) {
                authenticationResult.setSuccess();
            }
        }
        catch(Exception e) {
            //TODO
        }
        return authenticationResult;
    }

}
    \end{verbatim}
        
\subsection{IAuthenticationManager}
       
    Shown below is the interface which must be implemented by your
    authentication manager.
    
    \begin{verbatim}
package org.apache.uima.ducc.common.authentication;

public interface IAuthenticationManager {
    
    /**
     * This method is expected to return AuthenticationManager implementation version 
     * information.  It is nominally displayed by the DUCC webserver on the Login/Logout 
     * pages.
     * 
     * Example return value: Acme Authenticator 1.0
     * 
     * @return The version of the AuthenticationManager implementation.
     */
    public String getVersion();
    
    /**
     * This method is expected to return password checking information.  
     * It is nominally employed by the DUCC webserver to enable/disable
     *  password input area on the Login/Logout pages.
     * 
     * @return True if the AuthenticationManager implementation checks passwords; 
     * false otherwise.
     */
    public boolean isPasswordChecked();
    
    /**
     * This method is expected to perform authentication.
     * It is nominally employed by the DUCC webserver for submitted Login pages.
     * 
     * @param userid
     * @param domain
     * @param password
     * @return True if authentic userid+domain+password; false otherwise.
     */
    public IAuthenticationResult isAuthenticate(String userid, String domain, String password);
    
    /**
     * This method is expected to perform role validation.
     * It is nominally employed by the DUCC webserver for submitted Login pages.
     * 
     * @param userid
     * @param domain
     * @param role
     * @return True if authentic userid+domain+role; false otherwise.
     */
    public IAuthenticationResult isGroupMember(String userid, String domain, Role role);
    
    /**
     * The supported Roles
     */
    public enum Role {
        User,
        Admin
    }
}
    \end{verbatim}


\subsection{IAuthenticationResult}
    
    Shown below is the interface which must be returned by the required
    authentication methods in your authentication manager.
    
    \begin{verbatim}
package org.apache.uima.ducc.common.authentication;
    
public interface IAuthenticationResult {
    public void setSuccess();
    public void setFailure();
    public boolean isSuccess();
    public boolean isFailure();
    public void setCode(int code);
    public int getCode();
    public void setReason(String reason);
    public String getReason();
    public void setException(Exception exception);
    public Exception getException();
}
    \end{verbatim}
    
\subsection{Example ANT script to build jar}
    
    Shown below is an example ANT script to build a ducc-authenticator.jar file.
    The resulting jar file should be placed user DUCC's lib directory along with
    any dependency jars, and defined in ducc.properties file.
    
    \begin{verbatim}
<project name="uima-ducc-examples" default="build" basedir=".">
    
    <property name="TGT-LIB"                value="${basedir}/lib" />
    <property name="TGT-DUCC-AUTH-JAR"      value="${TGT-LIB}/ducc-authenticator.jar" />
    
    <target name="build" depends="clean, jar" />
    
    <target name="clean">
        <delete file="${TGT-DUCC-AUTH-JAR}" />
    </target>
    
    <target name="jar">
        <mkdir dir="${TGT-LIB}" />
        <jar destfile="${TGT-DUCC-AUTH-JAR}" basedir="${basedir}/target/classes/org/apache/uima/ducc/example/authentication/module"/>
    </target>
    
</project>
    \end{verbatim}
    
\subsection{Example ducc.properties entries}
    
    Shown here is a snippet of the ducc.properties file defining the class to be
    used for authentication and the administrator created folder
    {\em site-security}, which should contain the ducc-authenticator.jar you
    built plus any jar files upon which it depends.
    
    Note: the {\em site-security} directory must be located within DUCC's lib
    directory.
    
    \begin{verbatim}
# The class that performs authentication (for the WebServer)
ducc.authentication.implementer = org.apache.uima.ducc.example.authentication.module.AuthenticationManager

# Site specific jars: include all jars in directory site-security
ducc.local.jars = site-security/*
    \end{verbatim}
    
\subsection{Example ducc.administrators}
    
    Example contents of ducc.administrators file located within DUCC's resources
    directory. Only userids listed here can assume the Administrator role when 
    performing operations via the WebServer.
    
    \begin{verbatim}
jdoe
fred
hal9000
    \end{verbatim}


  
