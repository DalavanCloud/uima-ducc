% 
% Licensed to the Apache Software Foundation (ASF) under one
% or more contributor license agreements.  See the NOTICE file
% distributed with this work for additional information
% regarding copyright ownership.  The ASF licenses this file
% to you under the Apache License, Version 2.0 (the
% "License"); you may not use this file except in compliance
% with the License.  You may obtain a copy of the License at
% 
%   http://www.apache.org/licenses/LICENSE-2.0
% 
% Unless required by applicable law or agreed to in writing,
% software distributed under the License is distributed on an
% "AS IS" BASIS, WITHOUT WARRANTIES OR CONDITIONS OF ANY
% KIND, either express or implied.  See the License for the
% specific language governing permissions and limitations
% under the License.
% 
\section{Ducc Node Definitions}
\label{sec:admin-ducc.nodes}
    The DUCC node definitions are specified by default in the file {\em ducc.nodes}.

    The DUCC node list is used to configure the nodes used to run jobs and assign reservations. 
    When DUCC is started, the nodelist is read an a DUCC Agent is started on every node in the list.

    The node list can be composed of multiple node lists to assist organization of the DUCC cluster. 
    All the administrative commands operate upon node lists. By carefully organized these lists it is 
    possible to administer portions of a cluster independently. 

    In particular, it is highly recommended that the nodelists reflect the nodepool structure.  In
    this way, the configuration used to start DUCC is guaranteed to match the nodeppool definitions.

    Several types of records are permitted in nodelists:
    \begin{description}

      \item[Comments] A comment starts with the symbol ``$\#$''.  All text on the line
        following this symbol is ignored.

      \item[import] If a line starts with the symbol {\em import}, the next symbol on that line
        is expected to be the name of another node list.  This permits the DUCC cluster's
        nodes to be configured in a structured manner.

        For instance, the file {\em ducc.nodes} might consist entirely of {\em import} statements
        naming all of the nodepool files.
      \item[domain] This must be the first line of the file.  If specified, it should name
        the default domain to be used for all the nodes in this file, and the nodes named
        in imported files.  If not specified, then during start-up, nodes without domain names are assigned
        domain names according to the global domain name specified in the \hyperref[subsubsec:nodepool.configuration]{Resource Manager configuration}
        file, and if none is specified there, the domain name on the host starting DUCC is used.

      \item[nodename] This is a single token consisting of the name of a node on which an
        agent it to be started.

    \end{description}

    The example below shows a partial, hypothetical node configuration corresponding to the
    \hyperref[fig:nodepool.configuration]{nodepool configuration} above.

    \begin{figure}[H]

\begin{verbatim}
> cat ducc.nodes
# import all the nodes corresponding to my nodepools 
domain my.domain
import intel.nodes
import power.nodes
import jobdriver.nodes
import nightly-test.nodes
import timing-p7.nodes

> cat intel.nodes
# import the intel nodes, by frame
import intel-frame1.nodes
import intel-frame2.nodes
import intel-frame3.nodes

>cat intel-frame1.nodes
#import the specific nodes from frame1
r1s1node1
r1s1node2
r1s1node3
r1s1node4
\end{verbatim}
      \caption{Sample Node Configuration}
      \label{fig:node.configuration}

    \end{figure}    
    
