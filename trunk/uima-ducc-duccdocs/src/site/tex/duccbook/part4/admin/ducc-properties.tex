% 
% Licensed to the Apache Software Foundation (ASF) under one
% or more contributor license agreements.  See the NOTICE file
% distributed with this work for additional information
% regarding copyright ownership.  The ASF licenses this file
% to you under the Apache License, Version 2.0 (the
% "License"); you may not use this file except in compliance
% with the License.  You may obtain a copy of the License at
% 
%   http://www.apache.org/licenses/LICENSE-2.0
% 
% Unless required by applicable law or agreed to in writing,
% software distributed under the License is distributed on an
% "AS IS" BASIS, WITHOUT WARRANTIES OR CONDITIONS OF ANY
% KIND, either express or implied.  See the License for the
% specific language governing permissions and limitations
% under the License.
% 

\section{Properties}
 	
 	Public properties are in a primary configuration file is called ducc.properties 
	and always resides in the directory
    ducc\_runtime/resources.

	Private properties are in a secondary configuration file call ducc.private.properties
	and always resides in the directory
    ducc\_runtime/resources/private.

\section{Properties merging}
\label{sec:admin.properties-merge}
    
    With DUCC 2.0.0 the shipped DUCC properties file is designed to be read-only.  Installations
    create a local properties file which is automatically merged with the default properties file
    as part of system startup.

    The shipped DUCC properties file is called {\em default.ducc.properties}.  This file should
    never be edited or modified.

    The local site override properties file is called {\em site.ducc.properties}.  This is a 
    normal Java properties file containing override and additional properties.  An initial
    {\em site.ducc.properties} is created on installation of DUCC 2.0.0 by {\em ducc\_post\_install}.

    On startup 
    (\hyperref[subsec:admin.start-ducc]{\em start\_ducc}), 
    verification 
    (\hyperref[subsec:admin.check-ducc]{\em check\_ducc}),     
    and RM reconfiguration
    (\hyperref[subsec:admin.rm-reconfigure]{\em rm\_reconfigure}),     
    the two properties files are
    merged, with {\em site.ducc.properties} taking preference, to create the operational file,
    {\em ducc.properties}, which is used by all DUCC components.  This file should not be
    edited as it will be over-written whenever {\em start\_ducc} or {\em check\_ducc} is run.

\section{ducc.properties}
\label{sec:ducc.properties}
   
    Some of the properties in ducc.properties are intended as the "glue" that brings the various 
    DUCC components together and lets then run as a coherent whole. These types of properties should 
    be modified only by developers of DUCC itself.

    Some of the properties are tuning parameters: timeouts, heartbeat intervals, and so on. These
    may be modified by DUCC administrators, but only after experience is gained with DUCC, and only
    to solve specific performance problems. The default tuning parameters have been chosen by the
    DUCC system developers to provide "best" operation under most reasonable situations.

    Some of the properties describe the local cluster configuration: the location of the ActiveMQ
    broker, the location of the Java JRE, port numbers, etc. These should be modified by the DUCC
    administrators to configure DUCC to each individual installation.
   
\section{default.ducc.properties}
\label{sec:default.ducc.properties}   
	\verbatiminput{../../../../../src/main/resources/default.ducc.properties}
    