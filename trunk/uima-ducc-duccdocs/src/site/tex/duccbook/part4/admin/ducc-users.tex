% 
% Licensed to the Apache Software Foundation (ASF) under one
% or more contributor license agreements.  See the NOTICE file
% distributed with this work for additional information
% regarding copyright ownership.  The ASF licenses this file
% to you under the Apache License, Version 2.0 (the
% "License"); you may not use this file except in compliance
% with the License.  You may obtain a copy of the License at
% 
%   http://www.apache.org/licenses/LICENSE-2.0
% 
% Unless required by applicable law or agreed to in writing,
% software distributed under the License is distributed on an
% "AS IS" BASIS, WITHOUT WARRANTIES OR CONDITIONS OF ANY
% KIND, either express or implied.  See the License for the
% specific language governing permissions and limitations
% under the License.
% 
\section{Ducc User Definitions}
\label{sec:admin-ducc.users}
    The DUCC user registry provides user-specific overrides of various constraints
    DUCC might impose.  

    As of 2.0.0, the only constraint override is 
    \hyperref[sec:rm.allotment]{allotment} for non-preemptable requests.

    The syntax of the user registry is the same as that used in
    {\em ducc.classes}, and in fact, the user registry may be embedded directly
    in that file, rather than specified externally.

    The registry consists of multiple entries, one for affected user.  Any user
    of the system NOT in the registry acquires the system defaults.

    A user definition consists of the token ``User'' followed by the 
    id of the user, followed by a block delimited with ``curly'' braces \{ and \}.  This
    block contains the attributes of the nodepool as key/value pairs.
    Lineneds are ignored.  A semicolon ``$;$'' may optionally be used to
    delimit key/value pairs for readability, and an equals sign ``='' may optionally
    be used to delimit keys from values, also just for readability.  

    The attributes of a User entry are:
    \begin{description}
      \item[max-allotment] This overrides the maximum allotment for non-preemptable
        requests as defined in {\em ducc.properties}.  The override may be used to
        either increase, or decrease the user's allotment.  The units are in gigabytes.
    \end{description}

    The example below shows overrides for three users:
    \begin{itemize}
      \item Bob is allowed a non-preemptable allotment of 4000GB.
      \item May is allowed a non-preemptable allotment of 1000GB.
      \item Antoinette is allowed a non-preemptable allotment of 720GB.
    \end{itemize}    
        

    \begin{figure}[H]    
\begin{verbatim}
# --------------------- User Registry ---------------------
User bob         { max-allotment = 4000 }
User mary        { max-allotment = 1000 }
User antoinette  { max-allotment = 720  }
\end{verbatim}
          \caption{Sample User Registration}
      \label{fig:user.configuration}
    \end{figure}


