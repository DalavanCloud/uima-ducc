% 
% Licensed to the Apache Software Foundation (ASF) under one
% or more contributor license agreements.  See the NOTICE file
% distributed with this work for additional information
% regarding copyright ownership.  The ASF licenses this file
% to you under the Apache License, Version 2.0 (the
% "License"); you may not use this file except in compliance
% with the License.  You may obtain a copy of the License at
% 
%   http://www.apache.org/licenses/LICENSE-2.0
% 
% Unless required by applicable law or agreed to in writing,
% software distributed under the License is distributed on an
% "AS IS" BASIS, WITHOUT WARRANTIES OR CONDITIONS OF ANY
% KIND, either express or implied.  See the License for the
% specific language governing permissions and limitations
% under the License.
% 

\section{Administrative Tasks}

   The administrative tasks comprise one or several DUCC commands 
   and/or file editing to achieve a desired system state.
   
   It may be inconvenient to stop DUCC to make changes if there are,
   for example, services running that take a long time to initialize
   after a DUCC re-start.  These administrative tasks are performed
   with DUCC running, but carefully!  Making an invalid change may 
   cause the unexpected.  It is safer to shutdown DUCC, make changes,
   run check\_ducc to verify, then re-start DUCC.

\subsection{Add Node}
\label{subsec:admin.add-node}

	\subsubsection{{\em Description}}
    Persistently add a node to active service in the cluster.
    \begin{itemize}
      \item add the node to file resources/ducc.nodes
      \item add the node to file resources/ducc.classes (optional)
      \item run admin/start\_ducc -c agent@host.domain
      \item run admin/rm\_qoccupancy\textsuperscript{1}
    \end{itemize}

	The node should start appearing in the results from the
	rm\_qoccupancy\textsuperscript{1} command, perhaps with 
	some delay due to system latency.
	
\subsection{Remove Node}
\label{subsec:admin.remove-node}

    \subsubsection{{\em Description}}
    Persistently remove a node from active service in the cluster.
    \begin{itemize}
      \item remove the node from file resources/ducc.nodes
      \item remove the node from file resources/ducc.classes (if specified)
      \item run admin/stop\_ducc -c agent@host.domain
      \item run admin/rm\_reconfigure
      \item run admin/rm\_qoccupancy\textsuperscript{1}
    \end{itemize}

	The node should stop appearing in the results from the
	rm\_qoccupancy\textsuperscript{1} command, perhaps with
	some delay due to system latency.
	
	If running without the database, then stop and start the DUCC
	Web Server 	to cause the node to be removed from the Machines
	page by performing these additional steps:
	
	 \begin{itemize}
      \item run admin/stop\_ducc -c ws
      \item run admin/start\_ducc -c ws
    \end{itemize}
	
\subsection{Notes}
\label{subsec:admin.notes}

	\textsuperscript{1} rm\_qoccupancy is supported when system is configured
	to employ database only.